% Options for packages loaded elsewhere
\PassOptionsToPackage{unicode}{hyperref}
\PassOptionsToPackage{hyphens}{url}
%
\documentclass[
  ignorenonframetext,
]{beamer}
\usepackage{pgfpages}
\setbeamertemplate{caption}[numbered]
\setbeamertemplate{caption label separator}{: }
\setbeamercolor{caption name}{fg=normal text.fg}
\beamertemplatenavigationsymbolsempty
% Prevent slide breaks in the middle of a paragraph
\widowpenalties 1 10000
\raggedbottom
\setbeamertemplate{part page}{
  \centering
  \begin{beamercolorbox}[sep=16pt,center]{part title}
    \usebeamerfont{part title}\insertpart\par
  \end{beamercolorbox}
}
\setbeamertemplate{section page}{
  \centering
  \begin{beamercolorbox}[sep=12pt,center]{part title}
    \usebeamerfont{section title}\insertsection\par
  \end{beamercolorbox}
}
\setbeamertemplate{subsection page}{
  \centering
  \begin{beamercolorbox}[sep=8pt,center]{part title}
    \usebeamerfont{subsection title}\insertsubsection\par
  \end{beamercolorbox}
}
\AtBeginPart{
  \frame{\partpage}
}
\AtBeginSection{
  \ifbibliography
  \else
    \frame{\sectionpage}
  \fi
}
\AtBeginSubsection{
  \frame{\subsectionpage}
}
\usepackage{amsmath,amssymb}
\usepackage{lmodern}
\usepackage{ifxetex,ifluatex}
\ifnum 0\ifxetex 1\fi\ifluatex 1\fi=0 % if pdftex
  \usepackage[T1]{fontenc}
  \usepackage[utf8]{inputenc}
  \usepackage{textcomp} % provide euro and other symbols
\else % if luatex or xetex
  \usepackage{unicode-math}
  \defaultfontfeatures{Scale=MatchLowercase}
  \defaultfontfeatures[\rmfamily]{Ligatures=TeX,Scale=1}
  \setmainfont[BoldFont = SF Pro Rounded Semibold]{SF Pro Rounded}
  \setmathfont[]{STIX Two Math}
\fi
\usefonttheme{serif} % use mainfont rather than sansfont for slide text
% Use upquote if available, for straight quotes in verbatim environments
\IfFileExists{upquote.sty}{\usepackage{upquote}}{}
\IfFileExists{microtype.sty}{% use microtype if available
  \usepackage[]{microtype}
  \UseMicrotypeSet[protrusion]{basicmath} % disable protrusion for tt fonts
}{}
\makeatletter
\@ifundefined{KOMAClassName}{% if non-KOMA class
  \IfFileExists{parskip.sty}{%
    \usepackage{parskip}
  }{% else
    \setlength{\parindent}{0pt}
    \setlength{\parskip}{6pt plus 2pt minus 1pt}}
}{% if KOMA class
  \KOMAoptions{parskip=half}}
\makeatother
\usepackage{xcolor}
\IfFileExists{xurl.sty}{\usepackage{xurl}}{} % add URL line breaks if available
\IfFileExists{bookmark.sty}{\usepackage{bookmark}}{\usepackage{hyperref}}
\hypersetup{
  pdftitle={444 Lecture 8.4 - Many Player Stag Hunt},
  pdfauthor={Brian Weatherson},
  hidelinks,
  pdfcreator={LaTeX via pandoc}}
\urlstyle{same} % disable monospaced font for URLs
\newif\ifbibliography
\setlength{\emergencystretch}{3em} % prevent overfull lines
\providecommand{\tightlist}{%
  \setlength{\itemsep}{0pt}\setlength{\parskip}{0pt}}
\setcounter{secnumdepth}{-\maxdimen} % remove section numbering
\let\Tiny=\tiny

 \setbeamertemplate{navigation symbols}{} 

% \usetheme{Madrid}
 \usetheme[numbering=none, progressbar=foot]{metropolis}
 \usecolortheme{wolverine}
 \usepackage{color}
 \usepackage{MnSymbol}
% \usepackage{movie15}

\usepackage{amssymb}% http://ctan.org/pkg/amssymb
\usepackage{pifont}% http://ctan.org/pkg/pifont
\newcommand{\cmark}{\ding{51}}%
\newcommand{\xmark}{\ding{55}}%

\DeclareSymbolFont{symbolsC}{U}{txsyc}{m}{n}
\DeclareMathSymbol{\boxright}{\mathrel}{symbolsC}{128}
\DeclareMathAlphabet{\mathpzc}{OT1}{pzc}{m}{it}

\setlength{\parskip}{1ex plus 0.5ex minus 0.2ex}

\AtBeginSection[]
{
\begin{frame}
	\Huge{\color{darkblue} \insertsection}
\end{frame}
}

\renewenvironment*{quote}	
	{\list{}{\rightmargin   \leftmargin} \item } 	
	{\endlist }

\definecolor{darkgreen}{rgb}{0,0.7,0}
\definecolor{darkblue}{rgb}{0,0,0.8}

\usepackage[italic]{mathastext}
\usepackage{nicefrac}
\usepackage{istgame}

\setbeamertemplate{caption}{\raggedright\insertcaption}

%\def\toprule{}
%\def\bottomrule{}
%\def\midrule{}
\usepackage{etoolbox}
\AfterEndEnvironment{description}{\vspace{9pt}}
\AfterEndEnvironment{oltableau}{\vspace{9pt}}
\BeforeBeginEnvironment{oltableau}{\vspace{9pt}}
\AfterEndEnvironment{center}{\vspace{9pt}}
\BeforeBeginEnvironment{tabular}{\vspace{9pt}}
\AfterEndEnvironment{longtable}{\vspace{-6pt}}
\usepackage{booktabs}
\usepackage{longtable}
\usepackage{array}
\usepackage{multirow}
\usepackage{wrapfig}
\usepackage{float}
\usepackage{colortbl}
\usepackage{pdflscape}
\usepackage{tabu}
\usepackage{threeparttable} 
\usepackage{threeparttablex} 
\usepackage[normalem]{ulem} 
\usepackage{makecell}
\usepackage{xcolor}
\usepackage{ulem}

\setlength\heavyrulewidth{0ex}
\setlength\lightrulewidth{0.08ex}

\aboverulesep=0ex
\belowrulesep=0ex
\renewcommand{\arraystretch}{1.2}
\ifluatex
  \usepackage{selnolig}  % disable illegal ligatures
\fi

\title{444 Lecture 8.4 - Many Player Stag Hunt}
\author{Brian Weatherson}
\date{}

\begin{document}
\frame{\titlepage}

\begin{frame}{Generalising}
\protect\hypertarget{generalising}{}
\begin{itemize}
\tightlist
\item
  The world doesn't have many 2 player 2 option games.
\item
  If we're thinking of modelling real world situations, either as PD or
  Stag Hunt, we need something more general.
\end{itemize}
\end{frame}

\begin{frame}{Generalised Prisoners' Dilemma}
\protect\hypertarget{generalised-prisoners-dilemma}{}
\begin{itemize}
\tightlist
\item
  \(n > 2\) players each choose a number \(x\) in \([0, 1]\).
\item
  The mean of the choices is \(m\).
\item
  Payoff to each player is \(m - \frac{x}{r}\), for \(r\) between \(2\)
  and \(n\).
\end{itemize}
\end{frame}

\begin{frame}{General Pattern}
\protect\hypertarget{general-pattern}{}
\begin{itemize}
\tightlist
\item
  If everyone picks the same number, better for everyone that that
  number is higher. \pause
\item
  Holding fixed other players moves, it is always better to pick a lower
  number.
\end{itemize}
\end{frame}

\begin{frame}{Iteration}
\protect\hypertarget{iteration}{}
\begin{itemize}
\tightlist
\item
  It's really hard to do Axelrod-type stuff in these kinds of games.
\item
  Just having the chance to interact again is not enough to push people
  to cooperate.
\item
  There isn't enough freedom of movement; do you defect if 1 player out
  of 100 defects?
\end{itemize}
\end{frame}

\begin{frame}{Punishment}
\protect\hypertarget{punishment}{}
\begin{itemize}
\tightlist
\item
  Changing the payouts is a more effective move.
\item
  So what we see in these kinds of situations is what is called
  `altruistic punishment'.
\item
  One person makes themselves temporarily worse off, and the perpetrator
  much worse off, to send a signal that defection will not be tolerated.
\item
  Of course there is a free riding issue with who carries out the
  punishment, so \ldots{}
\end{itemize}
\end{frame}

\begin{frame}{Generalised Stag Hunt}
\protect\hypertarget{generalised-stag-hunt}{}
\begin{itemize}
\tightlist
\item
  \(n > 2\) players each choose a number \(x\) in \([0, 1]\).
\item
  The mean of the choices is \(m\). \pause
\item
  If a player chooses \(x \leq m\), their payout is \(x\). \pause
\item
  If they choosee \(x > m\), they receive \(m - r(x - m)\), where
  \(r > 1\) is some measure of how much one is penalised for leaving the
  equilibrium.
\end{itemize}
\end{frame}

\begin{frame}{General Pattern}
\protect\hypertarget{general-pattern-1}{}
\begin{itemize}
\tightlist
\item
  For any \(x\), everyone choosing \(x\) is a (strict) equilibrium.
\item
  The higher \(x\) is, the better this equilibrium is for everyone.
\item
  Choosing 0 minimises regret, and maximises expected return given some
  natural distributions of probability to the other player's moves.
\end{itemize}
\end{frame}

\begin{frame}{Real World}
\protect\hypertarget{real-world}{}
\begin{itemize}
\tightlist
\item
  For something to be a real world stag hunt, those are the features you
  (approximately) need.
\item
  The best thing to do is to do what everyone else does.
\item
  If everyone does the same thing, better that everyone does the more
  cooperative thing.
\item
  Given radical uncertainty about what others will do, best to do the
  uncooperative thing.
\end{itemize}
\end{frame}

\begin{frame}{For Next Time}
\protect\hypertarget{for-next-time}{}
\begin{itemize}
\tightlist
\item
  We'll talk through some examples of possible real life Stag Hunts.
\end{itemize}
\end{frame}

\end{document}
