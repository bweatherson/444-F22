% Options for packages loaded elsewhere
\PassOptionsToPackage{unicode}{hyperref}
\PassOptionsToPackage{hyphens}{url}
%
\documentclass[
  ignorenonframetext,
]{beamer}
\usepackage{pgfpages}
\setbeamertemplate{caption}[numbered]
\setbeamertemplate{caption label separator}{: }
\setbeamercolor{caption name}{fg=normal text.fg}
\beamertemplatenavigationsymbolsempty
% Prevent slide breaks in the middle of a paragraph
\widowpenalties 1 10000
\raggedbottom
\setbeamertemplate{part page}{
  \centering
  \begin{beamercolorbox}[sep=16pt,center]{part title}
    \usebeamerfont{part title}\insertpart\par
  \end{beamercolorbox}
}
\setbeamertemplate{section page}{
  \centering
  \begin{beamercolorbox}[sep=12pt,center]{part title}
    \usebeamerfont{section title}\insertsection\par
  \end{beamercolorbox}
}
\setbeamertemplate{subsection page}{
  \centering
  \begin{beamercolorbox}[sep=8pt,center]{part title}
    \usebeamerfont{subsection title}\insertsubsection\par
  \end{beamercolorbox}
}
\AtBeginPart{
  \frame{\partpage}
}
\AtBeginSection{
  \ifbibliography
  \else
    \frame{\sectionpage}
  \fi
}
\AtBeginSubsection{
  \frame{\subsectionpage}
}
\usepackage{amsmath,amssymb}
\usepackage{lmodern}
\usepackage{ifxetex,ifluatex}
\ifnum 0\ifxetex 1\fi\ifluatex 1\fi=0 % if pdftex
  \usepackage[T1]{fontenc}
  \usepackage[utf8]{inputenc}
  \usepackage{textcomp} % provide euro and other symbols
\else % if luatex or xetex
  \usepackage{unicode-math}
  \defaultfontfeatures{Scale=MatchLowercase}
  \defaultfontfeatures[\rmfamily]{Ligatures=TeX,Scale=1}
  \setmainfont[BoldFont = SF Pro Rounded Semibold]{SF Pro Rounded}
  \setmathfont[]{STIX Two Math}
\fi
\usefonttheme{serif} % use mainfont rather than sansfont for slide text
% Use upquote if available, for straight quotes in verbatim environments
\IfFileExists{upquote.sty}{\usepackage{upquote}}{}
\IfFileExists{microtype.sty}{% use microtype if available
  \usepackage[]{microtype}
  \UseMicrotypeSet[protrusion]{basicmath} % disable protrusion for tt fonts
}{}
\makeatletter
\@ifundefined{KOMAClassName}{% if non-KOMA class
  \IfFileExists{parskip.sty}{%
    \usepackage{parskip}
  }{% else
    \setlength{\parindent}{0pt}
    \setlength{\parskip}{6pt plus 2pt minus 1pt}}
}{% if KOMA class
  \KOMAoptions{parskip=half}}
\makeatother
\usepackage{xcolor}
\IfFileExists{xurl.sty}{\usepackage{xurl}}{} % add URL line breaks if available
\IfFileExists{bookmark.sty}{\usepackage{bookmark}}{\usepackage{hyperref}}
\hypersetup{
  pdftitle={444 Lecture 3.4 - Strategies in Dynamic Games},
  pdfauthor={Brian Weatherson},
  hidelinks,
  pdfcreator={LaTeX via pandoc}}
\urlstyle{same} % disable monospaced font for URLs
\newif\ifbibliography
\setlength{\emergencystretch}{3em} % prevent overfull lines
\providecommand{\tightlist}{%
  \setlength{\itemsep}{0pt}\setlength{\parskip}{0pt}}
\setcounter{secnumdepth}{-\maxdimen} % remove section numbering
\let\Tiny=\tiny

 \setbeamertemplate{navigation symbols}{} 

% \usetheme{Madrid}
 \usetheme[numbering=none, progressbar=foot]{metropolis}
 \usecolortheme{wolverine}
 \usepackage{color}
 \usepackage{MnSymbol}
% \usepackage{movie15}

\usepackage{amssymb}% http://ctan.org/pkg/amssymb
\usepackage{pifont}% http://ctan.org/pkg/pifont
\newcommand{\cmark}{\ding{51}}%
\newcommand{\xmark}{\ding{55}}%

\DeclareSymbolFont{symbolsC}{U}{txsyc}{m}{n}
\DeclareMathSymbol{\boxright}{\mathrel}{symbolsC}{128}
\DeclareMathAlphabet{\mathpzc}{OT1}{pzc}{m}{it}

\setlength{\parskip}{1ex plus 0.5ex minus 0.2ex}

\AtBeginSection[]
{
\begin{frame}
	\Huge{\color{darkblue} \insertsection}
\end{frame}
}

\renewenvironment*{quote}	
	{\list{}{\rightmargin   \leftmargin} \item } 	
	{\endlist }

\definecolor{darkgreen}{rgb}{0,0.7,0}
\definecolor{darkblue}{rgb}{0,0,0.8}

\usepackage[italic]{mathastext}
\usepackage{nicefrac}

\setbeamertemplate{caption}{\raggedright\insertcaption}

%\def\toprule{}
%\def\bottomrule{}
%\def\midrule{}
\usepackage{etoolbox}
\AfterEndEnvironment{description}{\vspace{9pt}}
\AfterEndEnvironment{oltableau}{\vspace{9pt}}
\BeforeBeginEnvironment{oltableau}{\vspace{9pt}}
\AfterEndEnvironment{center}{\vspace{9pt}}
\BeforeBeginEnvironment{tabular}{\vspace{9pt}}
\AfterEndEnvironment{longtable}{\vspace{-6pt}}
\usepackage{booktabs}
\usepackage{longtable}
\usepackage{array}
\usepackage{multirow}
\usepackage{wrapfig}
\usepackage{float}
\usepackage{colortbl}
\usepackage{pdflscape}
\usepackage{tabu}
\usepackage{threeparttable} 
\usepackage{threeparttablex} 
\usepackage[normalem]{ulem} 
\usepackage{makecell}
\usepackage{xcolor}
\usepackage{ulem}

\setlength\heavyrulewidth{0ex}
\setlength\lightrulewidth{0.08ex}

\aboverulesep=0ex
\belowrulesep=0ex
\renewcommand{\arraystretch}{1.2}
\ifluatex
  \usepackage{selnolig}  % disable illegal ligatures
\fi

\title{444 Lecture 3.4 - Strategies in Dynamic Games}
\author{Brian Weatherson}
\date{}

\begin{document}
\frame{\titlepage}

\begin{frame}{Plan}
\protect\hypertarget{plan}{}
\begin{itemize}
\tightlist
\item
  To relate game trees back to one-shot games.
\end{itemize}
\end{frame}

\begin{frame}{Reading}
\protect\hypertarget{reading}{}
\begin{itemize}
\tightlist
\item
  Bonanno, section 3.3.
\end{itemize}
\end{frame}

\begin{frame}{A Strategy}
\protect\hypertarget{a-strategy}{}
\begin{itemize}
\tightlist
\item
  Imagine that you're coaching someone who is going to play a dynamic
  game.
\item
  And imagine that you're a really really controlling coach, and they
  have a good memory.
\item
  Then you could tell them what to do in every possible situation.
\item
  That's a strategy.
\end{itemize}
\end{frame}

\begin{frame}{Control Freaks}
\protect\hypertarget{control-freaks}{}
\begin{itemize}
\tightlist
\item
  That metaphor might understate how controlling a strategy is supposed
  to be.
\item
  A strategy says what the player should do at every node in the tree.
\item
  That includes nodes that are ruled out by their earlier play.
\item
  So a strategy for chess might include the instructions ``Open with e4,
  then if the first two moves are d4-d5, follow with c4.''
\end{itemize}
\end{frame}

\begin{frame}{Why be a Control Freak?}
\protect\hypertarget{why-be-a-control-freak}{}
\begin{itemize}
\tightlist
\item
  Strategies serve two roles.
\item
  They get followed by one player.
\item
  And they get reasoned about by the other players, in order to create
  their own strategies.
\item
  And to play the latter role, sometimes you need these weird steps.
\end{itemize}
\end{frame}

\begin{frame}{Why be a Control Freak?}
\protect\hypertarget{why-be-a-control-freak-1}{}
\begin{itemize}
\tightlist
\item
  Sometimes in games we're interested in situations where there is a gap
  between giving an order and carrying it out.
\item
  When opposing generals are watching a battle play out, they can't
  assume that their instructions will be carried out to the letter, or
  that what they see on the battlefield is the result of the other side
  carrying out instructions properly.
\item
  In these cases, it is clear why a strategy should include ``What to do
  if you haven't done what I said you should do so far.''
\end{itemize}
\end{frame}

\begin{frame}{Back to Nash}
\protect\hypertarget{back-to-nash}{}
\begin{itemize}
\tightlist
\item
  Go back to the coaching metaphor.
\item
  And imagine the other player in the dynamic game also has a coach.
\item
  Then you and the other coach are playing a one-shot, simultaneous move
  game.
\item
  Each of you have a lot of options - but you get one shot to choose one
  of them, and the other player makes their choice at the same time
  (more or less) as you.
\end{itemize}
\end{frame}

\begin{frame}{Strategic (Normal) Form}
\protect\hypertarget{strategic-normal-form}{}
\begin{itemize}
\tightlist
\item
  For any two player game tree, we can write out all the strategies for
  each player.
\item
  I mean, we can in principle.
\item
  In chess there are probably more strategies than atoms in the
  universe, so in practice it would be hard, but in theory it can be
  done.
\item
  And from that we can form a table, and compute what would happen for
  each strategy pair.
\item
  And then we can do all the fun stuff from last week, like eliminating
  dominated strategies, finding best responses and Nash equilibria etc.
\end{itemize}
\end{frame}

\begin{frame}{The End of Time?}
\protect\hypertarget{the-end-of-time}{}
\begin{itemize}
\tightlist
\item
  So does that mean we don't need to think about dynamics at all?
\item
  No, and next time we'll look at why.
\item
  Some strategy pairs that make sense in a one-shot game don't seem to
  make sense in a dynamic game.
\end{itemize}
\end{frame}

\end{document}
