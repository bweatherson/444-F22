% Options for packages loaded elsewhere
\PassOptionsToPackage{unicode}{hyperref}
\PassOptionsToPackage{hyphens}{url}
%
\documentclass[
  ignorenonframetext,
]{beamer}
\usepackage{pgfpages}
\setbeamertemplate{caption}[numbered]
\setbeamertemplate{caption label separator}{: }
\setbeamercolor{caption name}{fg=normal text.fg}
\beamertemplatenavigationsymbolsempty
% Prevent slide breaks in the middle of a paragraph
\widowpenalties 1 10000
\raggedbottom
\setbeamertemplate{part page}{
  \centering
  \begin{beamercolorbox}[sep=16pt,center]{part title}
    \usebeamerfont{part title}\insertpart\par
  \end{beamercolorbox}
}
\setbeamertemplate{section page}{
  \centering
  \begin{beamercolorbox}[sep=12pt,center]{part title}
    \usebeamerfont{section title}\insertsection\par
  \end{beamercolorbox}
}
\setbeamertemplate{subsection page}{
  \centering
  \begin{beamercolorbox}[sep=8pt,center]{part title}
    \usebeamerfont{subsection title}\insertsubsection\par
  \end{beamercolorbox}
}
\AtBeginPart{
  \frame{\partpage}
}
\AtBeginSection{
  \ifbibliography
  \else
    \frame{\sectionpage}
  \fi
}
\AtBeginSubsection{
  \frame{\subsectionpage}
}
\usepackage{amsmath,amssymb}
\usepackage{lmodern}
\usepackage{ifxetex,ifluatex}
\ifnum 0\ifxetex 1\fi\ifluatex 1\fi=0 % if pdftex
  \usepackage[T1]{fontenc}
  \usepackage[utf8]{inputenc}
  \usepackage{textcomp} % provide euro and other symbols
\else % if luatex or xetex
  \usepackage{unicode-math}
  \defaultfontfeatures{Scale=MatchLowercase}
  \defaultfontfeatures[\rmfamily]{Ligatures=TeX,Scale=1}
  \setmainfont[BoldFont = SF Pro Rounded Semibold]{SF Pro Rounded}
  \setmathfont[]{STIX Two Math}
\fi
\usefonttheme{serif} % use mainfont rather than sansfont for slide text
% Use upquote if available, for straight quotes in verbatim environments
\IfFileExists{upquote.sty}{\usepackage{upquote}}{}
\IfFileExists{microtype.sty}{% use microtype if available
  \usepackage[]{microtype}
  \UseMicrotypeSet[protrusion]{basicmath} % disable protrusion for tt fonts
}{}
\makeatletter
\@ifundefined{KOMAClassName}{% if non-KOMA class
  \IfFileExists{parskip.sty}{%
    \usepackage{parskip}
  }{% else
    \setlength{\parindent}{0pt}
    \setlength{\parskip}{6pt plus 2pt minus 1pt}}
}{% if KOMA class
  \KOMAoptions{parskip=half}}
\makeatother
\usepackage{xcolor}
\IfFileExists{xurl.sty}{\usepackage{xurl}}{} % add URL line breaks if available
\IfFileExists{bookmark.sty}{\usepackage{bookmark}}{\usepackage{hyperref}}
\hypersetup{
  pdftitle={444 Lecture 9.1 - O'Connor Chapter 1},
  pdfauthor={Brian Weatherson},
  hidelinks,
  pdfcreator={LaTeX via pandoc}}
\urlstyle{same} % disable monospaced font for URLs
\newif\ifbibliography
\usepackage{longtable,booktabs,array}
\usepackage{calc} % for calculating minipage widths
\usepackage{caption}
% Make caption package work with longtable
\makeatletter
\def\fnum@table{\tablename~\thetable}
\makeatother
\setlength{\emergencystretch}{3em} % prevent overfull lines
\providecommand{\tightlist}{%
  \setlength{\itemsep}{0pt}\setlength{\parskip}{0pt}}
\setcounter{secnumdepth}{-\maxdimen} % remove section numbering
\let\Tiny=\tiny

 \setbeamertemplate{navigation symbols}{} 

% \usetheme{Madrid}
 \usetheme[numbering=none, progressbar=foot]{metropolis}
 \usecolortheme{wolverine}
 \usepackage{color}
 \usepackage{MnSymbol}
% \usepackage{movie15}

\usepackage{amssymb}% http://ctan.org/pkg/amssymb
\usepackage{pifont}% http://ctan.org/pkg/pifont
\newcommand{\cmark}{\ding{51}}%
\newcommand{\xmark}{\ding{55}}%

\DeclareSymbolFont{symbolsC}{U}{txsyc}{m}{n}
\DeclareMathSymbol{\boxright}{\mathrel}{symbolsC}{128}
\DeclareMathAlphabet{\mathpzc}{OT1}{pzc}{m}{it}

\setlength{\parskip}{1ex plus 0.5ex minus 0.2ex}

\AtBeginSection[]
{
\begin{frame}
	\Huge{\color{darkblue} \insertsection}
\end{frame}
}

\renewenvironment*{quote}	
	{\list{}{\rightmargin   \leftmargin} \item } 	
	{\endlist }

\definecolor{darkgreen}{rgb}{0,0.7,0}
\definecolor{darkblue}{rgb}{0,0,0.8}

\usepackage[italic]{mathastext}
\usepackage{nicefrac}
\usepackage{istgame}

\setbeamertemplate{caption}{\raggedright\insertcaption}

%\def\toprule{}
%\def\bottomrule{}
%\def\midrule{}
\usepackage{etoolbox}
\AfterEndEnvironment{description}{\vspace{9pt}}
\AfterEndEnvironment{oltableau}{\vspace{9pt}}
\BeforeBeginEnvironment{oltableau}{\vspace{9pt}}
\AfterEndEnvironment{center}{\vspace{9pt}}
\BeforeBeginEnvironment{tabular}{\vspace{9pt}}
\AfterEndEnvironment{longtable}{\vspace{-6pt}}
\usepackage{booktabs}
\usepackage{longtable}
\usepackage{array}
\usepackage{multirow}
\usepackage{wrapfig}
\usepackage{float}
\usepackage{colortbl}
\usepackage{pdflscape}
\usepackage{tabu}
\usepackage{threeparttable} 
\usepackage{threeparttablex} 
\usepackage[normalem]{ulem} 
\usepackage{makecell}
\usepackage{xcolor}
\usepackage{ulem}

\setlength\heavyrulewidth{0ex}
\setlength\lightrulewidth{0.08ex}

\aboverulesep=0ex
\belowrulesep=0ex
\renewcommand{\arraystretch}{1.2}
\AtBeginSection[]
{
    \begin{frame}
        \frametitle{Day Plan}
        \tableofcontents[currentsection]
    \end{frame}
}
\ifluatex
  \usepackage{selnolig}  % disable illegal ligatures
\fi

\title{444 Lecture 9.1 - O'Connor Chapter 1}
\author{Brian Weatherson}
\date{}

\begin{document}
\frame{\titlepage}

\begin{frame}{Day Plan}
\protect\hypertarget{day-plan}{}
\tableofcontents
\end{frame}

\hypertarget{gender-division}{%
\section{Gender Division}\label{gender-division}}

\begin{frame}{Genders in Society}
\protect\hypertarget{genders-in-society}{}
\begin{itemize}
\tightlist
\item
  The ubiquity of gender divisions in socities is really remarkable.
\item
  What other things are this ubiquitous? \pause
\item
  Definitely language use, which is itself remarkable.
\item
  And some age-related divisions, though with more variable
  manifestations. \pause
\item
  Language is the closest thing to gender; we see it in all societies,
  but the way we see it varies.
\end{itemize}
\end{frame}

\begin{frame}{Other Divisions}
\protect\hypertarget{other-divisions}{}
Obviously there are other divisions we see in societies.

\begin{itemize}
\tightlist
\item
  Race
\item
  Religion
\item
  Class \pause
\end{itemize}

But note two things about these.

\begin{enumerate}
\tightlist
\item
  Not nearly as ubiquitous.
\item
  They complement gender, not replace.
\end{enumerate}
\end{frame}

\hypertarget{types}{%
\section{Types}\label{types}}

\begin{frame}{Games with Types}
\protect\hypertarget{games-with-types}{}
\begin{itemize}
\tightlist
\item
  Games with types are sort of `between' two familiar kinds of games in
  complexity.
\item
  One is games where each player has a full awareness of the identity of
  who they are playing with, and can plan strategies that discriminate
  among these other players.
\item
  At the other end is where a player just has to pick a strategy in
  complete ignorance of who the other player is.
\item
  With types we get in between; you don't know who the other player is,
  but you know they are of type \(t\).
\end{itemize}
\end{frame}

\begin{frame}{Symmetry}
\protect\hypertarget{symmetry}{}
\begin{itemize}
\tightlist
\item
  The formal effect of this is to open up a new range of
  \textbf{symmetric} equilibria.
\item
  Without types, the only equilibrium in a complementary coordination
  game is really bad.
\end{itemize}

\begin{longtable}[]{@{}rcc@{}}
\toprule
& A & B \\
\midrule
\endhead
A & 0, 0 & 1, 1 \\
B & 1, 1 & 0, 0 \\
\bottomrule
\end{longtable}

\begin{itemize}
\tightlist
\item
  The only symmetric equilibrium is that we both play the mixed strategy
  half-A/half-B, with a return of 0.5.
\end{itemize}
\end{frame}

\begin{frame}{Types}
\protect\hypertarget{types-1}{}
\begin{longtable}[]{@{}rcc@{}}
\toprule
& A & B \\
\midrule
\endhead
A & 0, 0 & 1, 1 \\
B & 1, 1 & 0, 0 \\
\bottomrule
\end{longtable}

\begin{itemize}
\tightlist
\item
  Now imagine that we will first be assigned type-A or type-B, with
  probability 0.5 for each, and that types will be visible.
\item
  This opens up a new symmetric equilibrium: Play your type if the other
  is different, randomise if the other is same.
\item
  And this has an expected return of 0.75. \pause
\item
  This will be called a \textbf{population equilibrium} in chapter 2,
  and we'll return to it.
\end{itemize}
\end{frame}

\hypertarget{complementary-and-correlative-games}{%
\section{Complementary and Correlative
Games}\label{complementary-and-correlative-games}}

\begin{frame}{A Correlative Game}
\protect\hypertarget{a-correlative-game}{}
\begin{longtable}[]{@{}rcc@{}}
\toprule
& Bach & Stravinsky \\
\midrule
\endhead
Bach & 2, 1 & 0, 0 \\
Stravinsky & 0, 0 & 1, 2 \\
\bottomrule
\end{longtable}
\end{frame}

\begin{frame}{A Complementary Game}
\protect\hypertarget{a-complementary-game}{}
\begin{longtable}[]{@{}rcc@{}}
\toprule
& Favorite & Other \\
\midrule
\endhead
Favorite & 0, 0 & 2, 1 \\
Other & 1, 2 & 0, 0 \\
\bottomrule
\end{longtable}
\end{frame}

\begin{frame}{The Same Game?}
\protect\hypertarget{the-same-game}{}
\begin{itemize}
\tightlist
\item
  Aren't these the same game? \pause
\item
  Not necessarily; depending on how we set the game up.
\end{itemize}
\end{frame}

\begin{frame}{What makes something a move}
\protect\hypertarget{what-makes-something-a-move}{}
\begin{itemize}
\tightlist
\item
  The player must be physically capable of performing the move.
\item
  But they must also be capable of performing it under that description.
  \pause
\item
  Question: Can I call up Barack Obama? \pause
\item
  Positive answer: it's just a matter of dialing the right number, and I
  have the dexterity to hit the numbers. \pause
\item
  Negative answer: I don't know his phone number!
\item
  We're working with a system where the negative answer is the right
  one; which seems very natural.
\end{itemize}
\end{frame}

\begin{frame}{Correlative and Complementary}
\protect\hypertarget{correlative-and-complementary}{}
\begin{itemize}
\tightlist
\item
  Lewis: Correlative vs Complementary is just a matter of relabelling,
  it doesn't reflect a deep difference.
\item
  O'Connor: That relabelling might convert things that the player can do
  under that very description into things they cannot do under that
  description.
\end{itemize}
\end{frame}

\hypertarget{norms-and-conventions}{%
\section{Norms and Conventions}\label{norms-and-conventions}}

\begin{frame}{Norms and Conventions}
\protect\hypertarget{norms-and-conventions-1}{}
\begin{itemize}
\tightlist
\item
  The distinction here is tricky.
\item
  Conventions are things where everyone goes along because they'd expect
  to do worse as long as everyone else is following the convention.
\item
  Norms are things where everyone goes along because they'd expect to do
  worse as long as everyone else is endorsing the norm.
\item
  These seem really similar.
\end{itemize}
\end{frame}

\begin{frame}{Norms and Conventions}
\protect\hypertarget{norms-and-conventions-2}{}
\begin{itemize}
\tightlist
\item
  The difference is why they expect to do worse if they violate.
\item
  Roughly, social norms are where one expects to do worse because of
  punishment behavior by others.
\item
  Conventions are where you do worse because of a failure to coordinate.
  The restaurant won't punish me for knocking on the door at 3 in the
  morning to see if they are open, they just won't open up. (Because
  they aren't there!)
\end{itemize}
\end{frame}

\begin{frame}{Examples of Norms and Conventions}
\protect\hypertarget{examples-of-norms-and-conventions}{}
\begin{itemize}
\tightlist
\item
  I'm actually not convinced that driving on the socially approved side
  is a norm in this sense; the downsides are much more due to others
  behaving normally than due to punishment.
\item
  There is an intermediate case - where we internalise the convention,
  and violating it generates \emph{guilt}.
\item
  This should probably go with the punishment - not all punishments are
  by others.
\end{itemize}
\end{frame}

\hypertarget{hawk-dove}{%
\section{Hawk-Dove}\label{hawk-dove}}

\begin{frame}{Hawk-Dove}
\protect\hypertarget{hawk-dove-1}{}
\begin{itemize}
\tightlist
\item
  I haven't talked about this already, but I possibly should have.
\item
  When I talked last week about whether some games were Prisoners'
  Dilemma or Stag Hunt, there was a third option, Hawk-Dove.
\item
  And maybe that's the right model for some cases we discussed.
\end{itemize}
\end{frame}

\begin{frame}{What is Hawk-Dove}
\protect\hypertarget{what-is-hawk-dove}{}
\begin{enumerate}
\tightlist
\item
  Everyone wants the other person to be Dove; that's the cooperative
  move.
\item
  If everyone is a Hawk, it's a disaster. It's even worse than in PD.
\item
  But if everyone is a Dove, it's better to defect and play Hawk. That's
  like PD and unlike SH.
\end{enumerate}

Everyone wants to be the only defector.
\end{frame}

\begin{frame}{Simple Models}
\protect\hypertarget{simple-models}{}
\begin{itemize}
\tightlist
\item
  We will come back to Hawk-Dove in chapter 2, so we'll talk about it
  more then.
\item
  But for now it's good to have simple models in mind.
\item
  Don't think \emph{This coordination problem isn't PD, so must be SH}.
\item
  It could be Hawk-Dove instead.
\end{itemize}
\end{frame}

\hypertarget{notes}{%
\section{Notes}\label{notes}}

\begin{frame}{Military}
\protect\hypertarget{military}{}
\begin{itemize}
\tightlist
\item
  Just one small note this week.
\item
  Is it really true that militaries need strict hierarchies to be
  successful?
\item
  I think this turns on what you call a `military'.
\item
  The slave rebellion in what's now Haiti was pretty successful, without
  a particularly clear hierarchy.
\item
  The guerilla war the Spanish people waged against Napoleon was much
  more successful than anyone their own army did.
\item
  I don't think this makes a big difference to the story, but I was
  surprised to see this stated so categorically.
\end{itemize}
\end{frame}

\end{document}
