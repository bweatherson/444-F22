% Options for packages loaded elsewhere
\PassOptionsToPackage{unicode}{hyperref}
\PassOptionsToPackage{hyphens}{url}
%
\documentclass[
  ignorenonframetext,
]{beamer}
\usepackage{pgfpages}
\setbeamertemplate{caption}[numbered]
\setbeamertemplate{caption label separator}{: }
\setbeamercolor{caption name}{fg=normal text.fg}
\beamertemplatenavigationsymbolsempty
% Prevent slide breaks in the middle of a paragraph
\widowpenalties 1 10000
\raggedbottom
\setbeamertemplate{part page}{
  \centering
  \begin{beamercolorbox}[sep=16pt,center]{part title}
    \usebeamerfont{part title}\insertpart\par
  \end{beamercolorbox}
}
\setbeamertemplate{section page}{
  \centering
  \begin{beamercolorbox}[sep=12pt,center]{part title}
    \usebeamerfont{section title}\insertsection\par
  \end{beamercolorbox}
}
\setbeamertemplate{subsection page}{
  \centering
  \begin{beamercolorbox}[sep=8pt,center]{part title}
    \usebeamerfont{subsection title}\insertsubsection\par
  \end{beamercolorbox}
}
\AtBeginPart{
  \frame{\partpage}
}
\AtBeginSection{
  \ifbibliography
  \else
    \frame{\sectionpage}
  \fi
}
\AtBeginSubsection{
  \frame{\subsectionpage}
}
\usepackage{amsmath,amssymb}
\usepackage{lmodern}
\usepackage{ifxetex,ifluatex}
\ifnum 0\ifxetex 1\fi\ifluatex 1\fi=0 % if pdftex
  \usepackage[T1]{fontenc}
  \usepackage[utf8]{inputenc}
  \usepackage{textcomp} % provide euro and other symbols
\else % if luatex or xetex
  \usepackage{unicode-math}
  \defaultfontfeatures{Scale=MatchLowercase}
  \defaultfontfeatures[\rmfamily]{Ligatures=TeX,Scale=1}
  \setmainfont[BoldFont = SF Pro Rounded Semibold]{SF Pro Rounded}
  \setmathfont[]{STIX Two Math}
\fi
\usefonttheme{serif} % use mainfont rather than sansfont for slide text
% Use upquote if available, for straight quotes in verbatim environments
\IfFileExists{upquote.sty}{\usepackage{upquote}}{}
\IfFileExists{microtype.sty}{% use microtype if available
  \usepackage[]{microtype}
  \UseMicrotypeSet[protrusion]{basicmath} % disable protrusion for tt fonts
}{}
\makeatletter
\@ifundefined{KOMAClassName}{% if non-KOMA class
  \IfFileExists{parskip.sty}{%
    \usepackage{parskip}
  }{% else
    \setlength{\parindent}{0pt}
    \setlength{\parskip}{6pt plus 2pt minus 1pt}}
}{% if KOMA class
  \KOMAoptions{parskip=half}}
\makeatother
\usepackage{xcolor}
\IfFileExists{xurl.sty}{\usepackage{xurl}}{} % add URL line breaks if available
\IfFileExists{bookmark.sty}{\usepackage{bookmark}}{\usepackage{hyperref}}
\hypersetup{
  pdftitle={444 Lecture 13.2 - Private Vices and Public Virtues},
  pdfauthor={Brian Weatherson},
  hidelinks,
  pdfcreator={LaTeX via pandoc}}
\urlstyle{same} % disable monospaced font for URLs
\newif\ifbibliography
\setlength{\emergencystretch}{3em} % prevent overfull lines
\providecommand{\tightlist}{%
  \setlength{\itemsep}{0pt}\setlength{\parskip}{0pt}}
\setcounter{secnumdepth}{-\maxdimen} % remove section numbering
\let\Tiny=\tiny

 \setbeamertemplate{navigation symbols}{} 

% \usetheme{Madrid}
 \usetheme[numbering=none, progressbar=foot]{metropolis}
 \usecolortheme{wolverine}
 \usepackage{color}
 \usepackage{MnSymbol}
% \usepackage{movie15}

\usepackage{amssymb}% http://ctan.org/pkg/amssymb
\usepackage{pifont}% http://ctan.org/pkg/pifont
\newcommand{\cmark}{\ding{51}}%
\newcommand{\xmark}{\ding{55}}%

\DeclareSymbolFont{symbolsC}{U}{txsyc}{m}{n}
\DeclareMathSymbol{\boxright}{\mathrel}{symbolsC}{128}
\DeclareMathAlphabet{\mathpzc}{OT1}{pzc}{m}{it}

\setlength{\parskip}{1ex plus 0.5ex minus 0.2ex}

\AtBeginSection[]
{
\begin{frame}
	\Huge{\color{darkblue} \insertsection}
\end{frame}
}

\renewenvironment*{quote}	
	{\list{}{\rightmargin   \leftmargin} \item } 	
	{\endlist }

\definecolor{darkgreen}{rgb}{0,0.7,0}
\definecolor{darkblue}{rgb}{0,0,0.8}

\usepackage[italic]{mathastext}
\usepackage{nicefrac}
\usepackage{istgame}

\setbeamertemplate{caption}{\raggedright\insertcaption}

%\def\toprule{}
%\def\bottomrule{}
%\def\midrule{}
\usepackage{etoolbox}
\AfterEndEnvironment{description}{\vspace{9pt}}
\AfterEndEnvironment{oltableau}{\vspace{9pt}}
\BeforeBeginEnvironment{oltableau}{\vspace{9pt}}
\AfterEndEnvironment{center}{\vspace{9pt}}
\BeforeBeginEnvironment{tabular}{\vspace{9pt}}
\AfterEndEnvironment{longtable}{\vspace{-6pt}}
\usepackage{booktabs}
\usepackage{longtable}
\usepackage{array}
\usepackage{multirow}
\usepackage{wrapfig}
\usepackage{float}
\usepackage{colortbl}
\usepackage{pdflscape}
\usepackage{tabu}
\usepackage{threeparttable} 
\usepackage{threeparttablex} 
\usepackage[normalem]{ulem} 
\usepackage{makecell}
\usepackage{xcolor}
\usepackage{ulem}

\setlength\heavyrulewidth{0ex}
\setlength\lightrulewidth{0.08ex}

\aboverulesep=0ex
\belowrulesep=0ex
\renewcommand{\arraystretch}{1.2}
\AtBeginSection[]
{
    \begin{frame}
        \frametitle{Day Plan}
        \tableofcontents[currentsection]
    \end{frame}
}
\ifluatex
  \usepackage{selnolig}  % disable illegal ligatures
\fi

\title{444 Lecture 13.2 - Private Vices and Public Virtues}
\author{Brian Weatherson}
\date{}

\begin{document}
\frame{\titlepage}

\begin{frame}{Day Plan}
\protect\hypertarget{day-plan}{}
\tableofcontents
\end{frame}

\hypertarget{virtues-and-vices}{%
\section{Virtues and Vices}\label{virtues-and-vices}}

\begin{frame}{Epistemic Virtue}
\protect\hypertarget{epistemic-virtue}{}
The tradition they are working in traces back to Aristotle's ethics, and
its emphasis on virtue.

\begin{itemize}
\tightlist
\item
  Or, more precisely, it traces back to the mid-20th Century
  rediscovery/reinvention of Aristotelian ethics.
\item
  The very general picture is that ethics is abut character traits, not
  outcomes (like in consequentialism) or actions (as in deontology).
\end{itemize}
\end{frame}

\begin{frame}{Aristotelian Medians}
\protect\hypertarget{aristotelian-medians}{}
Aristotle had a particular take on virtue that we may, but don't have
to, adopt.

\begin{itemize}
\tightlist
\item
  Virtues are midpoints between two vices.
\item
  So courage is the midpoint between cowardice and foolishness.
\item
  Is this what virtue always consists in? I'm not sure; and the fact
  that we can ask that question means that we can sensibly talk about
  virtue theory independent of its Aristotelian origin.
\end{itemize}
\end{frame}

\hypertarget{virtue-epistemology}{%
\section{Virtue Epistemology}\label{virtue-epistemology}}

\begin{frame}{Dharmottara Cases}
\protect\hypertarget{dharmottara-cases}{}
\begin{itemize}
\tightlist
\item
  Virtue epistemology was originally introduced to solve Dharmottara
  cases.
\item
  Knowledge is true belief obtained via the exercise of an epistemic
  virtue.
\end{itemize}
\end{frame}

\begin{frame}{Other Applications}
\protect\hypertarget{other-applications}{}
But the theory is both more and less interesting than that.

\begin{itemize}
\tightlist
\item
  Less because like all other solutions to the puzzle raised by
  Dharmottara's example, it doesn't work.
\item
  But more because there are a lot of epistemic virtues that are
  interesting in their own right.
\end{itemize}
\end{frame}

\begin{frame}{Other Virtues}
\protect\hypertarget{other-virtues}{}
\begin{itemize}[<+->]
\tightlist
\item
  Open-mindedness
\item
  Inquisitiveness
\item
  Modesty
\item
  Curiosity
\end{itemize}
\end{frame}

\begin{frame}{Two Questions}
\protect\hypertarget{two-questions}{}
\begin{enumerate}[<+->]
\tightlist
\item
  Which of these are about belief, as opposed to inquiry?
\item
  Which of these are Aristotelian means?
\end{enumerate}
\end{frame}

\hypertarget{private-vices-as-public-goods}{%
\section{Private Vices as Public
Goods}\label{private-vices-as-public-goods}}

\begin{frame}{Historical Precedents}
\protect\hypertarget{historical-precedents}{}
\begin{itemize}
\tightlist
\item
  Mandeville's \emph{Fable of the Bees} (on one interpretation): saving
  is private virtue but public vice.
\item
  Being excessively confident, and going in for exploration or
  innovation, might be private vice but public virtue.
\end{itemize}
\end{frame}

\begin{frame}{Epistemic Equivalents?}
\protect\hypertarget{epistemic-equivalents}{}
\begin{enumerate}[<+->]
\tightlist
\item
  Cultural evolution - over imitation is irrational but useful for
  spreading new ideas.
\item
  Argumentation - It's good if people are over-invested in defending
  their own view/finding flaws in others
\end{enumerate}
\end{frame}

\hypertarget{objections}{%
\section{Objections}\label{objections}}

\begin{frame}{Two Obvious Objections}
\protect\hypertarget{two-obvious-objections}{}
\begin{enumerate}[<+->]
\tightlist
\item
  These aren't public goods. Or, at least, they aren't epistemic public
  goods.
\item
  These aren't private vices. I'll spend more time on this.
\end{enumerate}
\end{frame}

\begin{frame}{Imitation and Discrimination}
\protect\hypertarget{imitation-and-discrimination}{}
Someone who headbutts a light switch because they are imitating looks
fairly silly. But are they irrational?

\begin{itemize}
\tightlist
\item
  They aren't actually maximising utility, but that's too high a
  standard.
\item
  What does it really mean to say someone is ``over-imitating''?
\end{itemize}
\end{frame}

\begin{frame}{Imitation and Discrimination}
\protect\hypertarget{imitation-and-discrimination-1}{}
Remember there are very good reasons for imitation.

\begin{itemize}
\tightlist
\item
  If someone does something, and they subsequently get a good result,
  that is evidence that the `something' leads to a good result.
\item
  This is especially true because the `something' didn't just happen, it
  was an action.
\item
  It's a good general practice to assume that things you see people do
  are done for reasons.
\item
  That's especially true for the things you see successful people do.
\end{itemize}
\end{frame}

\begin{frame}{Second Order Evidence}
\protect\hypertarget{second-order-evidence}{}
Very often, we get a reason to believe that \(p\) not by being given
direct evidence of it, but by being given evidence that someone else has
evidence of it.

\begin{itemize}
\tightlist
\item
  The fact that someone who wanted to achieve outcome O did action A to
  get it, and got outcome O, is in fact two bits of evidence.
\item
  From the success, we get direct evidence that A leads to O.
\item
  From the fact they did it, we get indirect evidence that A is a good
  way to get to O.
\end{itemize}
\end{frame}

\begin{frame}{Vigilence}
\protect\hypertarget{vigilence}{}
But surely this is defeasible, and the claim is that ordinary people
don't let it get defeated often enough?

\begin{itemize}
\tightlist
\item
  True, but that doesn't mean we have to check for ourselves that each
  bit of imitation is worthwhile.
\item
  It certainly doesn't mean each bit of imitation has to be worthwhile.
\item
  It just means that we have to be \textbf{vigilant}.
\end{itemize}
\end{frame}

\begin{frame}{Vigilence}
\protect\hypertarget{vigilence-1}{}
To my mind, this is one of the most important ideas in recent
epistemology.

\begin{itemize}
\tightlist
\item
  Good epistemic activity requires vigilence.
\item
  Example: walking down crowded street.
\item
  Really hard question: how do you implement vigilence in an artificial
  system?
\end{itemize}
\end{frame}

\begin{frame}{Imitation with vigilence}
\protect\hypertarget{imitation-with-vigilence}{}
My take on these examples is that copying with vigilence is good, i.e.,
is virtuous.

\begin{itemize}
\tightlist
\item
  So this isn't a public virtue that's a private vice.
\item
  But maybe that's too optimistic a take on what the individuals are up
  to.
\end{itemize}
\end{frame}

\end{document}
