% Options for packages loaded elsewhere
\PassOptionsToPackage{unicode}{hyperref}
\PassOptionsToPackage{hyphens}{url}
%
\documentclass[
  ignorenonframetext,
]{beamer}
\usepackage{pgfpages}
\setbeamertemplate{caption}[numbered]
\setbeamertemplate{caption label separator}{: }
\setbeamercolor{caption name}{fg=normal text.fg}
\beamertemplatenavigationsymbolsempty
% Prevent slide breaks in the middle of a paragraph
\widowpenalties 1 10000
\raggedbottom
\setbeamertemplate{part page}{
  \centering
  \begin{beamercolorbox}[sep=16pt,center]{part title}
    \usebeamerfont{part title}\insertpart\par
  \end{beamercolorbox}
}
\setbeamertemplate{section page}{
  \centering
  \begin{beamercolorbox}[sep=12pt,center]{part title}
    \usebeamerfont{section title}\insertsection\par
  \end{beamercolorbox}
}
\setbeamertemplate{subsection page}{
  \centering
  \begin{beamercolorbox}[sep=8pt,center]{part title}
    \usebeamerfont{subsection title}\insertsubsection\par
  \end{beamercolorbox}
}
\AtBeginPart{
  \frame{\partpage}
}
\AtBeginSection{
  \ifbibliography
  \else
    \frame{\sectionpage}
  \fi
}
\AtBeginSubsection{
  \frame{\subsectionpage}
}
\usepackage{amsmath,amssymb}
\usepackage{lmodern}
\usepackage{ifxetex,ifluatex}
\ifnum 0\ifxetex 1\fi\ifluatex 1\fi=0 % if pdftex
  \usepackage[T1]{fontenc}
  \usepackage[utf8]{inputenc}
  \usepackage{textcomp} % provide euro and other symbols
\else % if luatex or xetex
  \usepackage{unicode-math}
  \defaultfontfeatures{Scale=MatchLowercase}
  \defaultfontfeatures[\rmfamily]{Ligatures=TeX,Scale=1}
  \setmainfont[BoldFont = SF Pro Rounded Semibold]{SF Pro Rounded}
  \setmathfont[]{STIX Two Math}
\fi
\usefonttheme{serif} % use mainfont rather than sansfont for slide text
% Use upquote if available, for straight quotes in verbatim environments
\IfFileExists{upquote.sty}{\usepackage{upquote}}{}
\IfFileExists{microtype.sty}{% use microtype if available
  \usepackage[]{microtype}
  \UseMicrotypeSet[protrusion]{basicmath} % disable protrusion for tt fonts
}{}
\makeatletter
\@ifundefined{KOMAClassName}{% if non-KOMA class
  \IfFileExists{parskip.sty}{%
    \usepackage{parskip}
  }{% else
    \setlength{\parindent}{0pt}
    \setlength{\parskip}{6pt plus 2pt minus 1pt}}
}{% if KOMA class
  \KOMAoptions{parskip=half}}
\makeatother
\usepackage{xcolor}
\IfFileExists{xurl.sty}{\usepackage{xurl}}{} % add URL line breaks if available
\IfFileExists{bookmark.sty}{\usepackage{bookmark}}{\usepackage{hyperref}}
\hypersetup{
  pdftitle={444 Lecture 2.10 - Nash Equilibrium},
  pdfauthor={Brian Weatherson},
  hidelinks,
  pdfcreator={LaTeX via pandoc}}
\urlstyle{same} % disable monospaced font for URLs
\newif\ifbibliography
\setlength{\emergencystretch}{3em} % prevent overfull lines
\providecommand{\tightlist}{%
  \setlength{\itemsep}{0pt}\setlength{\parskip}{0pt}}
\setcounter{secnumdepth}{-\maxdimen} % remove section numbering
\let\Tiny=\tiny

 \setbeamertemplate{navigation symbols}{} 

% \usetheme{Madrid}
 \usetheme[numbering=none, progressbar=foot]{metropolis}
 \usecolortheme{wolverine}
 \usepackage{color}
 \usepackage{MnSymbol}
% \usepackage{movie15}

\usepackage{amssymb}% http://ctan.org/pkg/amssymb
\usepackage{pifont}% http://ctan.org/pkg/pifont
\newcommand{\cmark}{\ding{51}}%
\newcommand{\xmark}{\ding{55}}%

\DeclareSymbolFont{symbolsC}{U}{txsyc}{m}{n}
\DeclareMathSymbol{\boxright}{\mathrel}{symbolsC}{128}
\DeclareMathAlphabet{\mathpzc}{OT1}{pzc}{m}{it}

\setlength{\parskip}{1ex plus 0.5ex minus 0.2ex}

\AtBeginSection[]
{
\begin{frame}
	\Huge{\color{darkblue} \insertsection}
\end{frame}
}

\renewenvironment*{quote}	
	{\list{}{\rightmargin   \leftmargin} \item } 	
	{\endlist }

\definecolor{darkgreen}{rgb}{0,0.7,0}
\definecolor{darkblue}{rgb}{0,0,0.8}

\usepackage[italic]{mathastext}
\usepackage{nicefrac}

\setbeamertemplate{caption}{\raggedright\insertcaption}

%\def\toprule{}
%\def\bottomrule{}
%\def\midrule{}
\usepackage{etoolbox}
\AfterEndEnvironment{description}{\vspace{9pt}}
\AfterEndEnvironment{oltableau}{\vspace{9pt}}
\BeforeBeginEnvironment{oltableau}{\vspace{9pt}}
\AfterEndEnvironment{center}{\vspace{9pt}}
\BeforeBeginEnvironment{tabular}{\vspace{9pt}}
\AfterEndEnvironment{longtable}{\vspace{-6pt}}
\usepackage{booktabs}
\usepackage{longtable}
\usepackage{array}
\usepackage{multirow}
\usepackage{wrapfig}
\usepackage{float}
\usepackage{colortbl}
\usepackage{pdflscape}
\usepackage{tabu}
\usepackage{threeparttable} 
\usepackage{threeparttablex} 
\usepackage[normalem]{ulem} 
\usepackage{makecell}
\usepackage{xcolor}
\usepackage{ulem}

\setlength\heavyrulewidth{0ex}
\setlength\lightrulewidth{0.08ex}

\aboverulesep=0ex
\belowrulesep=0ex
\renewcommand{\arraystretch}{1.2}
\ifluatex
  \usepackage{selnolig}  % disable illegal ligatures
\fi

\title{444 Lecture 2.10 - Nash Equilibrium}
\author{Brian Weatherson}
\date{}

\begin{document}
\frame{\titlepage}

\begin{frame}{Plan}
\protect\hypertarget{plan}{}
To think about why we should care about Nash Equilibrium.
\end{frame}

\begin{frame}{Reading}
\protect\hypertarget{reading}{}
Bonanno, section 2.6.
\end{frame}

\begin{frame}{A Philosophical Claim}
\protect\hypertarget{a-philosophical-claim}{}
In any game where it is common knowledge that all the players are
rational, every player will play a strategy that forms part of a Nash
Equilibrium.
\end{frame}

\begin{frame}{Status of Nash}
\protect\hypertarget{status-of-nash}{}
\begin{itemize}
\tightlist
\item
  I think most economists and political scientists accept something like
  this.
\item
  But I think philosophers who work on game theory more often do not
  accept it.
\end{itemize}
\end{frame}

\begin{frame}{Arguments for Nash}
\protect\hypertarget{arguments-for-nash}{}
\begin{itemize}
\tightlist
\item
  Oddly, it's hard to find canonical arguments for the importance of
  Nash.
\item
  It's so deeply embedded in game theory that it doesn't get discussed
  in research articles, more in textbooks.
\item
  Bonanno has a passage on page 40 that you could (perhaps uncharitably)
  count as a contribution to that genre.
\end{itemize}
\end{frame}

\begin{frame}{Transparency of Reason Interpretation}
\protect\hypertarget{transparency-of-reason-interpretation}{}
\begin{quote}
If players are all ``equally rational'' and Player 2 reaches the
conclusion that she should play y, then Player 1 must be able to
duplicate Player 2's reasoning process and come to the same conclusion;
it follows that Player 1's choice of strategy is not rational unless it
is a strategy x that is optimal against y. A similar argument applies to
Player 2's choice of strategy (y must be optimal against x) and thus
(x,y) is a Nash equilibrium.
\end{quote}
\end{frame}

\begin{frame}{Transparency of Reason Interpretation}
\protect\hypertarget{transparency-of-reason-interpretation-1}{}
\begin{itemize}
\tightlist
\item
  This doesn't look like a good argument for the Philosophical Claim.
\item
  All it shows is the weaker claim that if there is a uniquely rational
  play for each player, those plays will form a Nash Equilibrium.
\end{itemize}
\end{frame}

\begin{frame}{Viable recommendation interpretation}
\protect\hypertarget{viable-recommendation-interpretation}{}
\begin{quote}
Imagine that a third party makes a public recommendation to each player
on what strategy to play; then no player will have an incentive to
deviate from the recommendation (if she believes that the other players
will follow the recommendation) if and only if the recommended strategy
profile is a Nash equilibrium.
\end{quote}
\end{frame}

\begin{frame}{Viable recommendation interpretation}
\protect\hypertarget{viable-recommendation-interpretation-1}{}
\begin{itemize}
\tightlist
\item
  Again, this argument only works if the third party makes a unique
  recommendation.
\item
  If the third party says ``Do one of these three things'', there is no
  argument that all three have to be Nash.
\end{itemize}
\end{frame}

\begin{frame}{Self-enforcing agreement interpretation}
\protect\hypertarget{self-enforcing-agreement-interpretation}{}
\begin{quote}
Imagine that the players are able to communicate before playing the game
and reach a non-binding agreement expressed as a strategy profile s;
then no player will have an incentive to deviate from the agreement (if
she believes that the other player will follow the agreement) if and
only if s is a Nash equilibrium.
\end{quote}
\end{frame}

\begin{frame}{Self-enforcing agreement interpretation}
\protect\hypertarget{self-enforcing-agreement-interpretation-1}{}
\begin{itemize}
\tightlist
\item
  This is right as far as it goes, but doesn't help defend the
  philosophical claim in cases where no communication is possible.
\item
  And here it is particularly notable that Bonanno's purposes are not
  quite the same as mine.
\end{itemize}
\end{frame}

\begin{frame}{No regret interpretation}
\protect\hypertarget{no-regret-interpretation}{}
\begin{quote}
s is a Nash equilibrium if there is no player who, after observing the
opponent's choice, regrets his own choice (in the sense that he could
have done better with a different strategy of his, given the observed
strategy of the opponent).
\end{quote}

\begin{itemize}
\tightlist
\item
  This is a very good clear definition of what Nash is, but hard to see
  how it's a an argument for the importance of Nash.
\end{itemize}
\end{frame}

\begin{frame}{Other Arguments}
\protect\hypertarget{other-arguments}{}
\begin{itemize}[<+->]
\tightlist
\item
  You might be being spied on.
\item
  The other player might be a mind-reader.
\item
  You might be playing repeatedly, and so your strategy will be (more or
  less) revealed.
\end{itemize}
\end{frame}

\begin{frame}{Repeated Games}
\protect\hypertarget{repeated-games}{}
\begin{itemize}
\tightlist
\item
  The last one is, I think, the main reason in practice people care
  about Nash.
\item
  But it turns out for one important game, the Prisoners Dilemma, it is
  arguable that in the repeated game you should not play the Nash
  equilibrium.
\end{itemize}
\end{frame}

\begin{frame}{Prisoners' Dilemma}
\protect\hypertarget{prisoners-dilemma}{}
\begin{table}[!h]
\centering
\begin{tabular}[t]{>{}r|cc}
\toprule
 & Coop & Defect\\
\midrule
Coop & 3, 3 & 0, \fbox{5}\\
Defect & \fbox{5}, 0 & \fbox{1}, \fbox{1}\\
\bottomrule
\end{tabular}
\end{table}

\begin{itemize}
\tightlist
\item
  The only Nash equilinrium is both players defect.
\item
  And personally, I think in the one-shot game they should both defect.
\item
  But it is not at all obvious they should defect in the repeated game.
\item
  We will return to this point a lot in a few weeks.
\end{itemize}
\end{frame}

\begin{frame}{For Next Week}
\protect\hypertarget{for-next-week}{}
\begin{itemize}
\tightlist
\item
  We will start looking at chapter 3, on games that have sequential
  moves.
\end{itemize}
\end{frame}

\end{document}
