% Options for packages loaded elsewhere
\PassOptionsToPackage{unicode}{hyperref}
\PassOptionsToPackage{hyphens}{url}
%
\documentclass[
  ignorenonframetext,
]{beamer}
\usepackage{pgfpages}
\setbeamertemplate{caption}[numbered]
\setbeamertemplate{caption label separator}{: }
\setbeamercolor{caption name}{fg=normal text.fg}
\beamertemplatenavigationsymbolsempty
% Prevent slide breaks in the middle of a paragraph
\widowpenalties 1 10000
\raggedbottom
\setbeamertemplate{part page}{
  \centering
  \begin{beamercolorbox}[sep=16pt,center]{part title}
    \usebeamerfont{part title}\insertpart\par
  \end{beamercolorbox}
}
\setbeamertemplate{section page}{
  \centering
  \begin{beamercolorbox}[sep=12pt,center]{part title}
    \usebeamerfont{section title}\insertsection\par
  \end{beamercolorbox}
}
\setbeamertemplate{subsection page}{
  \centering
  \begin{beamercolorbox}[sep=8pt,center]{part title}
    \usebeamerfont{subsection title}\insertsubsection\par
  \end{beamercolorbox}
}
\AtBeginPart{
  \frame{\partpage}
}
\AtBeginSection{
  \ifbibliography
  \else
    \frame{\sectionpage}
  \fi
}
\AtBeginSubsection{
  \frame{\subsectionpage}
}
\usepackage{amsmath,amssymb}
\usepackage{lmodern}
\usepackage{ifxetex,ifluatex}
\ifnum 0\ifxetex 1\fi\ifluatex 1\fi=0 % if pdftex
  \usepackage[T1]{fontenc}
  \usepackage[utf8]{inputenc}
  \usepackage{textcomp} % provide euro and other symbols
\else % if luatex or xetex
  \usepackage{unicode-math}
  \defaultfontfeatures{Scale=MatchLowercase}
  \defaultfontfeatures[\rmfamily]{Ligatures=TeX,Scale=1}
  \setmainfont[BoldFont = SF Pro Rounded Semibold]{SF Pro Rounded}
  \setmathfont[]{STIX Two Math}
\fi
\usefonttheme{serif} % use mainfont rather than sansfont for slide text
% Use upquote if available, for straight quotes in verbatim environments
\IfFileExists{upquote.sty}{\usepackage{upquote}}{}
\IfFileExists{microtype.sty}{% use microtype if available
  \usepackage[]{microtype}
  \UseMicrotypeSet[protrusion]{basicmath} % disable protrusion for tt fonts
}{}
\makeatletter
\@ifundefined{KOMAClassName}{% if non-KOMA class
  \IfFileExists{parskip.sty}{%
    \usepackage{parskip}
  }{% else
    \setlength{\parindent}{0pt}
    \setlength{\parskip}{6pt plus 2pt minus 1pt}}
}{% if KOMA class
  \KOMAoptions{parskip=half}}
\makeatother
\usepackage{xcolor}
\IfFileExists{xurl.sty}{\usepackage{xurl}}{} % add URL line breaks if available
\IfFileExists{bookmark.sty}{\usepackage{bookmark}}{\usepackage{hyperref}}
\hypersetup{
  pdftitle={444 Lecture 7.4 - Beer and Quiche},
  pdfauthor={Brian Weatherson},
  hidelinks,
  pdfcreator={LaTeX via pandoc}}
\urlstyle{same} % disable monospaced font for URLs
\newif\ifbibliography
\setlength{\emergencystretch}{3em} % prevent overfull lines
\providecommand{\tightlist}{%
  \setlength{\itemsep}{0pt}\setlength{\parskip}{0pt}}
\setcounter{secnumdepth}{-\maxdimen} % remove section numbering
\let\Tiny=\tiny

 \setbeamertemplate{navigation symbols}{} 

% \usetheme{Madrid}
 \usetheme[numbering=none, progressbar=foot]{metropolis}
 \usecolortheme{wolverine}
 \usepackage{color}
 \usepackage{MnSymbol}
% \usepackage{movie15}

\usepackage{amssymb}% http://ctan.org/pkg/amssymb
\usepackage{pifont}% http://ctan.org/pkg/pifont
\newcommand{\cmark}{\ding{51}}%
\newcommand{\xmark}{\ding{55}}%

\DeclareSymbolFont{symbolsC}{U}{txsyc}{m}{n}
\DeclareMathSymbol{\boxright}{\mathrel}{symbolsC}{128}
\DeclareMathAlphabet{\mathpzc}{OT1}{pzc}{m}{it}

\setlength{\parskip}{1ex plus 0.5ex minus 0.2ex}

\AtBeginSection[]
{
\begin{frame}
	\Huge{\color{darkblue} \insertsection}
\end{frame}
}

\renewenvironment*{quote}	
	{\list{}{\rightmargin   \leftmargin} \item } 	
	{\endlist }

\definecolor{darkgreen}{rgb}{0,0.7,0}
\definecolor{darkblue}{rgb}{0,0,0.8}

\usepackage[italic]{mathastext}
\usepackage{nicefrac}
\usepackage{istgame}

\setbeamertemplate{caption}{\raggedright\insertcaption}

%\def\toprule{}
%\def\bottomrule{}
%\def\midrule{}
\usepackage{etoolbox}
\AfterEndEnvironment{description}{\vspace{9pt}}
\AfterEndEnvironment{oltableau}{\vspace{9pt}}
\BeforeBeginEnvironment{oltableau}{\vspace{9pt}}
\AfterEndEnvironment{center}{\vspace{9pt}}
\BeforeBeginEnvironment{tabular}{\vspace{9pt}}
\AfterEndEnvironment{longtable}{\vspace{-6pt}}
\usepackage{booktabs}
\usepackage{longtable}
\usepackage{array}
\usepackage{multirow}
\usepackage{wrapfig}
\usepackage{float}
\usepackage{colortbl}
\usepackage{pdflscape}
\usepackage{tabu}
\usepackage{threeparttable} 
\usepackage{threeparttablex} 
\usepackage[normalem]{ulem} 
\usepackage{makecell}
\usepackage{xcolor}
\usepackage{ulem}

\setlength\heavyrulewidth{0ex}
\setlength\lightrulewidth{0.08ex}

\aboverulesep=0ex
\belowrulesep=0ex
\renewcommand{\arraystretch}{1.2}
\ifluatex
  \usepackage{selnolig}  % disable illegal ligatures
\fi

\title{444 Lecture 7.4 - Beer and Quiche}
\author{Brian Weatherson}
\date{}

\begin{document}
\frame{\titlepage}

\begin{frame}{The Beer-Quiche Game}
\protect\hypertarget{the-beer-quiche-game}{}
\begin{itemize}
\tightlist
\item
  Sender's car breaks down on the way to work, so he walks into a bar to
  wait somewhere while the repair truck comes. (I think in the 1985
  version he's looking for a phone.)
\item
  He quickly realises this is a rougher bar than he expected, and the
  patrons are all staring at him.
\item
  Sender is smart, and he quickly realises that the patrons are both
  bullies and cowards. They're bullies, so they are looking for a fight,
  but cowards, so they won't fight a Tough Guy. And they think it's
  about 60\% likely that he's a Tough Guy.
\item
  Sender really wants to avoid a fight (whether or not he's a Tough
  Guy).
\item
  He knows that if he just tries to leave, they will conclude that he
  too is a Wimp, so he better order something
\end{itemize}
\end{frame}

\begin{frame}{The Beer-Quiche Game}
\protect\hypertarget{the-beer-quiche-game-1}{}
\begin{itemize}
\tightlist
\item
  His choices are beer or quiche.
\item
  He knows that the patrons believe, correctly, that if he's a Tough
  Guy, he'd prefer beer, and if he's a Wimp, he'd prefer quiche.
\item
  And while they can't read his character, they can hear his order.
\item
  But he would also prefer not to get in a fight either way. Even Tough
  Guys have better things to do at 8 in the morning.
\item
  What to do?
\end{itemize}
\end{frame}

\begin{frame}
\begin{center}
\begin{istgame}[scale=1.3]
   \xtdistance{20mm}{20mm}
   \istroot(0)[chance node]{$c$}
     \istb<grow=left>{0.6}[a]
     \istb<grow=right>{0.4}[a]
     \endist
   \xtdistance{10mm}{20mm}
   \istroot(1)(0-1)<180>{1}
     \istb<grow=north>{Beer}[l]
     \istb<grow=south>{Quiche}[l]
     \endist
   \istroot(2)(0-2)<0>{1}
     \istb<grow=north>{Beer}[r]
     \istb<grow=south>{Quiche}[r]
     \endist
   \istroot'[north](a1)(1-1)
     \istb{F}[bl]{1,-1}
     \istb{N}[br]{4,0}
     \endist
   \istroot(b1)(1-2)
     \istb{F}[al]{-1,-1}
     \istb{N}[ar]{2,0}
     \endist
   \istroot(a2)(2-2)
     \istb{F}[al]{-1,1}
     \istb{N}[ar]{4,0}
     \endist
   \istroot'[north](b2)(2-1)
     \istb{F}[bl]{-1,1}
     \istb{N}[br]{2,0}
     \endist
   \xtInfoset(a1)(b2){2}
   \xtInfoset(b1)(a2){2}
   \end{istgame}
\end{center}
\end{frame}

\begin{frame}
\begin{center}
\begin{istgame}[scale=0.9]
   \xtdistance{20mm}{20mm}
   \istroot(0)[chance node]{$c$}
     \istb<grow=left>{0.6}[a]
     \istb<grow=right>{0.4}[a]
     \endist
   \xtdistance{10mm}{20mm}
   \istroot(1)(0-1)<180>{1}
     \istb<grow=north>{Beer}[l]
     \istb<grow=south>{Quiche}[l]
     \endist
   \istroot(2)(0-2)<0>{1}
     \istb<grow=north>{Beer}[r]
     \istb<grow=south>{Quiche}[r]
     \endist
   \istroot'[north](a1)(1-1)
     \istb{F}[bl]{1,-1}
     \istb{N}[br]{4,0}
     \endist
   \istroot(b1)(1-2)
     \istb{F}[al]{-1,-1}
     \istb{N}[ar]{2,0}
     \endist
   \istroot(a2)(2-2)
     \istb{F}[al]{-1,1}
     \istb{N}[ar]{4,0}
     \endist
   \istroot'[north](b2)(2-1)
     \istb{F}[bl]{-1,1}
     \istb{N}[br]{2,0}
     \endist
   \xtInfoset(a1)(b2){2}
   \xtInfoset(b1)(a2){2}
   \end{istgame}
\end{center}

Sender gets

\begin{itemize}
\tightlist
\item
  3 points for avoiding fight;
\item
  +1 for liked order, -1 for disliked order.
\end{itemize}
\end{frame}

\begin{frame}
\begin{center}
\begin{istgame}[scale=0.9]
   \xtdistance{20mm}{20mm}
   \istroot(0)[chance node]{$c$}
     \istb<grow=left>{0.6}[a]
     \istb<grow=right>{0.4}[a]
     \endist
   \xtdistance{10mm}{20mm}
   \istroot(1)(0-1)<180>{1}
     \istb<grow=north>{Beer}[l]
     \istb<grow=south>{Quiche}[l]
     \endist
   \istroot(2)(0-2)<0>{1}
     \istb<grow=north>{Beer}[r]
     \istb<grow=south>{Quiche}[r]
     \endist
   \istroot'[north](a1)(1-1)
     \istb{F}[bl]{1,-1}
     \istb{N}[br]{4,0}
     \endist
   \istroot(b1)(1-2)
     \istb{F}[al]{-1,-1}
     \istb{N}[ar]{2,0}
     \endist
   \istroot(a2)(2-2)
     \istb{F}[al]{-1,1}
     \istb{N}[ar]{4,0}
     \endist
   \istroot'[north](b2)(2-1)
     \istb{F}[bl]{-1,1}
     \istb{N}[br]{2,0}
     \endist
   \xtInfoset(a1)(b2){2}
   \xtInfoset(b1)(a2){2}
   \end{istgame}
\end{center}

Hearer gets

\begin{itemize}
\tightlist
\item
  1 point for fighting Wimp;
\item
  -1 point for fighting Tough Guy
\end{itemize}
\end{frame}

\begin{frame}
\begin{center}
\begin{istgame}[scale=0.9]
   \xtdistance{20mm}{20mm}
   \istroot(0)[chance node]{$c$}
     \istb<grow=left>{0.6}[a]
     \istb<grow=right>{0.4}[a]
     \endist
   \xtdistance{10mm}{20mm}
   \istroot(1)(0-1)<180>{1}
     \istb<grow=north>{Beer}[l]
     \istb<grow=south>{Quiche}[l]
     \endist
   \istroot(2)(0-2)<0>{1}
     \istb<grow=north>{Beer}[r]
     \istb<grow=south>{Quiche}[r]
     \endist
   \istroot'[north](a1)(1-1)
     \istb{F}[bl]{1,-1}
     \istb{N}[br]{4,0}
     \endist
   \istroot(b1)(1-2)
     \istb{F}[al]{-1,-1}
     \istb{N}[ar]{2,0}
     \endist
   \istroot(a2)(2-2)
     \istb{F}[al]{-1,1}
     \istb{N}[ar]{4,0}
     \endist
   \istroot'[north](b2)(2-1)
     \istb{F}[bl]{-1,1}
     \istb{N}[br]{2,0}
     \endist
   \xtInfoset(a1)(b2){2}
   \xtInfoset(b1)(a2){2}
   \end{istgame}
\end{center}

Obvious Equilibrium

\begin{itemize}
\tightlist
\item
  Sender orders Beer if either Tough Guy or Wimp.
\item
  Hearer doesn't fight if Beer, fights if Quiche.
\end{itemize}
\end{frame}

\begin{frame}
\begin{center}
\begin{istgame}[scale=0.9]
   \xtdistance{20mm}{20mm}
   \istroot(0)[chance node]{$c$}
     \istb<grow=left>{0.6}[a]
     \istb<grow=right>{0.4}[a]
     \endist
   \xtdistance{10mm}{20mm}
   \istroot(1)(0-1)<180>{1}
     \istb<grow=north>{Beer}[l]
     \istb<grow=south>{Quiche}[l]
     \endist
   \istroot(2)(0-2)<0>{1}
     \istb<grow=north>{Beer}[r]
     \istb<grow=south>{Quiche}[r]
     \endist
   \istroot'[north](a1)(1-1)
     \istb{F}[bl]{1,-1}
     \istb{N}[br]{4,0}
     \endist
   \istroot(b1)(1-2)
     \istb{F}[al]{-1,-1}
     \istb{N}[ar]{2,0}
     \endist
   \istroot(a2)(2-2)
     \istb{F}[al]{-1,1}
     \istb{N}[ar]{4,0}
     \endist
   \istroot'[north](b2)(2-1)
     \istb{F}[bl]{-1,1}
     \istb{N}[br]{2,0}
     \endist
   \xtInfoset(a1)(b2){2}
   \xtInfoset(b1)(a2){2}
   \end{istgame}
\end{center}

Non-Obvious Equilibrium

\begin{itemize}
\tightlist
\item
  Sender orders Quiche if either Tough Guy or Wimp.
\item
  Hearer doesn't fight if Quiche, fights if Beer.
\end{itemize}
\end{frame}

\begin{frame}{Mathematical Puzzle}
\protect\hypertarget{mathematical-puzzle}{}
\begin{itemize}
\tightlist
\item
  What constraints on equilibrium selection can rule out the non-obvious
  explanation?
\item
  Really fun puzzle if you like puzzles, but not for us.
\item
  The initial statement of the puzzle, and an idea for a solution, is in
  Cho and Kreps, \emph{Signaling Games and Stable Equilibrium}, QJE
  1987.
\item
  If you like puzzles in this area, I highly recommend that paper.
\end{itemize}
\end{frame}

\begin{frame}{Our Lessons}
\protect\hypertarget{our-lessons}{}
\begin{itemize}
\tightlist
\item
  Nature may provide something like a `character', or what Harsanyi
  called a `type', to Sender.
\item
  You don't have to think of this as some random event that occurs at a
  particular time, like the whimsical assignment of characters to the
  pre-infants in \emph{Soul}.
\item
  All that matters is that there is some feature of Sender that Sender
  knows and Hearer doesn't.
\item
  Well, and that Hearer's probability distribution over the possible
  types of Sender is common knowledge; this game gets nasty if the
  initial probability for Tough Guy is under 0.5.
\end{itemize}
\end{frame}

\begin{frame}{Our Lessons (cont)}
\protect\hypertarget{our-lessons-cont}{}
\begin{itemize}
\tightlist
\item
  This is also a good example of a non-cooperative, but positive-sum,
  signaling game.
\item
  And that's the kind of game that we're going to spend more time
  looking at in future lectures.
\end{itemize}
\end{frame}

\end{document}
