% Options for packages loaded elsewhere
\PassOptionsToPackage{unicode}{hyperref}
\PassOptionsToPackage{hyphens}{url}
%
\documentclass[
  ignorenonframetext,
]{beamer}
\usepackage{pgfpages}
\setbeamertemplate{caption}[numbered]
\setbeamertemplate{caption label separator}{: }
\setbeamercolor{caption name}{fg=normal text.fg}
\beamertemplatenavigationsymbolsempty
% Prevent slide breaks in the middle of a paragraph
\widowpenalties 1 10000
\raggedbottom
\setbeamertemplate{part page}{
  \centering
  \begin{beamercolorbox}[sep=16pt,center]{part title}
    \usebeamerfont{part title}\insertpart\par
  \end{beamercolorbox}
}
\setbeamertemplate{section page}{
  \centering
  \begin{beamercolorbox}[sep=12pt,center]{part title}
    \usebeamerfont{section title}\insertsection\par
  \end{beamercolorbox}
}
\setbeamertemplate{subsection page}{
  \centering
  \begin{beamercolorbox}[sep=8pt,center]{part title}
    \usebeamerfont{subsection title}\insertsubsection\par
  \end{beamercolorbox}
}
\AtBeginPart{
  \frame{\partpage}
}
\AtBeginSection{
  \ifbibliography
  \else
    \frame{\sectionpage}
  \fi
}
\AtBeginSubsection{
  \frame{\subsectionpage}
}
\usepackage{amsmath,amssymb}
\usepackage{lmodern}
\usepackage{ifxetex,ifluatex}
\ifnum 0\ifxetex 1\fi\ifluatex 1\fi=0 % if pdftex
  \usepackage[T1]{fontenc}
  \usepackage[utf8]{inputenc}
  \usepackage{textcomp} % provide euro and other symbols
\else % if luatex or xetex
  \usepackage{unicode-math}
  \defaultfontfeatures{Scale=MatchLowercase}
  \defaultfontfeatures[\rmfamily]{Ligatures=TeX,Scale=1}
  \setmainfont[BoldFont = SF Pro Rounded Semibold]{SF Pro Rounded}
  \setmathfont[]{STIX Two Math}
\fi
\usefonttheme{serif} % use mainfont rather than sansfont for slide text
% Use upquote if available, for straight quotes in verbatim environments
\IfFileExists{upquote.sty}{\usepackage{upquote}}{}
\IfFileExists{microtype.sty}{% use microtype if available
  \usepackage[]{microtype}
  \UseMicrotypeSet[protrusion]{basicmath} % disable protrusion for tt fonts
}{}
\makeatletter
\@ifundefined{KOMAClassName}{% if non-KOMA class
  \IfFileExists{parskip.sty}{%
    \usepackage{parskip}
  }{% else
    \setlength{\parindent}{0pt}
    \setlength{\parskip}{6pt plus 2pt minus 1pt}}
}{% if KOMA class
  \KOMAoptions{parskip=half}}
\makeatother
\usepackage{xcolor}
\IfFileExists{xurl.sty}{\usepackage{xurl}}{} % add URL line breaks if available
\IfFileExists{bookmark.sty}{\usepackage{bookmark}}{\usepackage{hyperref}}
\hypersetup{
  pdftitle={444 Lecture 5.1 - Cardinal Games},
  pdfauthor={Brian Weatherson},
  hidelinks,
  pdfcreator={LaTeX via pandoc}}
\urlstyle{same} % disable monospaced font for URLs
\newif\ifbibliography
\setlength{\emergencystretch}{3em} % prevent overfull lines
\providecommand{\tightlist}{%
  \setlength{\itemsep}{0pt}\setlength{\parskip}{0pt}}
\setcounter{secnumdepth}{-\maxdimen} % remove section numbering
\let\Tiny=\tiny

 \setbeamertemplate{navigation symbols}{} 

% \usetheme{Madrid}
 \usetheme[numbering=none, progressbar=foot]{metropolis}
 \usecolortheme{wolverine}
 \usepackage{color}
 \usepackage{MnSymbol}
% \usepackage{movie15}

\usepackage{amssymb}% http://ctan.org/pkg/amssymb
\usepackage{pifont}% http://ctan.org/pkg/pifont
\newcommand{\cmark}{\ding{51}}%
\newcommand{\xmark}{\ding{55}}%

\DeclareSymbolFont{symbolsC}{U}{txsyc}{m}{n}
\DeclareMathSymbol{\boxright}{\mathrel}{symbolsC}{128}
\DeclareMathAlphabet{\mathpzc}{OT1}{pzc}{m}{it}

\setlength{\parskip}{1ex plus 0.5ex minus 0.2ex}

\AtBeginSection[]
{
\begin{frame}
	\Huge{\color{darkblue} \insertsection}
\end{frame}
}

\renewenvironment*{quote}	
	{\list{}{\rightmargin   \leftmargin} \item } 	
	{\endlist }

\definecolor{darkgreen}{rgb}{0,0.7,0}
\definecolor{darkblue}{rgb}{0,0,0.8}

\usepackage[italic]{mathastext}
\usepackage{nicefrac}

\setbeamertemplate{caption}{\raggedright\insertcaption}

%\def\toprule{}
%\def\bottomrule{}
%\def\midrule{}
\usepackage{etoolbox}
\AfterEndEnvironment{description}{\vspace{9pt}}
\AfterEndEnvironment{oltableau}{\vspace{9pt}}
\BeforeBeginEnvironment{oltableau}{\vspace{9pt}}
\AfterEndEnvironment{center}{\vspace{9pt}}
\BeforeBeginEnvironment{tabular}{\vspace{9pt}}
\AfterEndEnvironment{longtable}{\vspace{-6pt}}
\usepackage{booktabs}
\usepackage{longtable}
\usepackage{array}
\usepackage{multirow}
\usepackage{wrapfig}
\usepackage{float}
\usepackage{colortbl}
\usepackage{pdflscape}
\usepackage{tabu}
\usepackage{threeparttable} 
\usepackage{threeparttablex} 
\usepackage[normalem]{ulem} 
\usepackage{makecell}
\usepackage{xcolor}
\usepackage{ulem}

\setlength\heavyrulewidth{0ex}
\setlength\lightrulewidth{0.08ex}

\aboverulesep=0ex
\belowrulesep=0ex
\renewcommand{\arraystretch}{1.2}
\ifluatex
  \usepackage{selnolig}  % disable illegal ligatures
\fi

\title{444 Lecture 5.1 - Cardinal Games}
\author{Brian Weatherson}
\date{}

\begin{document}
\frame{\titlepage}

\begin{frame}{Plan}
\protect\hypertarget{plan}{}
Talk about why we might care about having cardinal payouts in games.
\end{frame}

\begin{frame}{Reading}
\protect\hypertarget{reading}{}
Bonanno, section 6.1
\end{frame}

\begin{frame}{Games with Lotteries}
\protect\hypertarget{games-with-lotteries}{}
Here is one thing we can do with cardinal utilities - include lotteries
in the payoffs.

\begin{itemize}
\tightlist
\item
  We can treat the lottery ticket as having a value equal to the
  \textbf{expected value} of the lottery.
\end{itemize}
\end{frame}

\begin{frame}{Games with Lotteries}
\protect\hypertarget{games-with-lotteries-1}{}
Bonanno illustrates this with a game that involves an actual lottery -
an auction where tied bids are resolved by a chance mechanism.

\begin{itemize}
\tightlist
\item
  But philosophically, lots of things in life look like lottery ticket.
\item
  How much is \$1 million worth? \pause
\item
  It depends a bit on whether there is lots of inflation in the near
  future. \pause
\item
  It also depends on whether there is a revolution soon and millionaires
  are in danger.
\end{itemize}
\end{frame}

\begin{frame}{Everything's a Gamble}
\protect\hypertarget{everythings-a-gamble}{}
The orthodox treatment of these questions, which I totally endorse, is
that a quantity of money is just as much a gamble as a lottery ticket.

\begin{itemize}
\tightlist
\item
  It's a relatively safe gamble; there hasn't been hyperinflation or
  anti-capitialist revolution in America in a long time.
\item
  But it's a gamble.
\item
  So even games with monetary payouts are gambles - gambles on the
  future value of money.
\end{itemize}
\end{frame}

\begin{frame}{Chicken 1}
\protect\hypertarget{chicken-1}{}
Here is a version of chicken using ordinal utility.

\begin{table}[!h]
\centering
\begin{tabular}[t]{>{}r|cc}
\toprule
 & swerve & drive\\
\midrule
Swerve & 3, 3 & 2, 4\\
Drive & 4, 2 & 1, 1\\
\bottomrule
\end{tabular}
\end{table}
\end{frame}

\begin{frame}{Chicken 2}
\protect\hypertarget{chicken-2}{}
\begin{table}[!h]
\centering
\begin{tabular}[t]{>{}r|cc}
\toprule
 & swerve & drive\\
\midrule
Swerve & 1, 1 & 0, 2\\
Drive & 2, 0 & -5, -5\\
\bottomrule
\end{tabular}
\end{table}

I guess you mostly swerve in this game, but you think about driving.
\end{frame}

\begin{frame}{Chicken 3}
\protect\hypertarget{chicken-3}{}
\begin{table}[!h]
\centering
\begin{tabular}[t]{>{}r|cc}
\toprule
 & swerve & drive\\
\midrule
Swerve & 1, 1 & 0, 2\\
Drive & 2, 0 & -5000, -5000\\
\bottomrule
\end{tabular}
\end{table}

Please swerve! \pause

\begin{itemize}
\tightlist
\item
  But (Swerve, swerve) is not Nash. We'll come back to this.
\end{itemize}
\end{frame}

\begin{frame}{Cardinal Utility Matters}
\protect\hypertarget{cardinal-utility-matters}{}
\begin{itemize}
\tightlist
\item
  The last two games were alike in ordinal utility.
\item
  But they were unlike in how you should play them.
\item
  So more than ordinal utility matters for how you should play.
\end{itemize}
\end{frame}

\begin{frame}{For Next Time}
\protect\hypertarget{for-next-time}{}
We will introduce an important means for solving games like Chicken -
the mixed strategy.
\end{frame}

\end{document}
