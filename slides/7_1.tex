% Options for packages loaded elsewhere
\PassOptionsToPackage{unicode}{hyperref}
\PassOptionsToPackage{hyphens}{url}
%
\documentclass[
  ignorenonframetext,
]{beamer}
\usepackage{pgfpages}
\setbeamertemplate{caption}[numbered]
\setbeamertemplate{caption label separator}{: }
\setbeamercolor{caption name}{fg=normal text.fg}
\beamertemplatenavigationsymbolsempty
% Prevent slide breaks in the middle of a paragraph
\widowpenalties 1 10000
\raggedbottom
\setbeamertemplate{part page}{
  \centering
  \begin{beamercolorbox}[sep=16pt,center]{part title}
    \usebeamerfont{part title}\insertpart\par
  \end{beamercolorbox}
}
\setbeamertemplate{section page}{
  \centering
  \begin{beamercolorbox}[sep=12pt,center]{part title}
    \usebeamerfont{section title}\insertsection\par
  \end{beamercolorbox}
}
\setbeamertemplate{subsection page}{
  \centering
  \begin{beamercolorbox}[sep=8pt,center]{part title}
    \usebeamerfont{subsection title}\insertsubsection\par
  \end{beamercolorbox}
}
\AtBeginPart{
  \frame{\partpage}
}
\AtBeginSection{
  \ifbibliography
  \else
    \frame{\sectionpage}
  \fi
}
\AtBeginSubsection{
  \frame{\subsectionpage}
}
\usepackage{amsmath,amssymb}
\usepackage{lmodern}
\usepackage{ifxetex,ifluatex}
\ifnum 0\ifxetex 1\fi\ifluatex 1\fi=0 % if pdftex
  \usepackage[T1]{fontenc}
  \usepackage[utf8]{inputenc}
  \usepackage{textcomp} % provide euro and other symbols
\else % if luatex or xetex
  \usepackage{unicode-math}
  \defaultfontfeatures{Scale=MatchLowercase}
  \defaultfontfeatures[\rmfamily]{Ligatures=TeX,Scale=1}
  \setmainfont[BoldFont = SF Pro Rounded Semibold]{SF Pro Rounded}
  \setmathfont[]{STIX Two Math}
\fi
\usefonttheme{serif} % use mainfont rather than sansfont for slide text
% Use upquote if available, for straight quotes in verbatim environments
\IfFileExists{upquote.sty}{\usepackage{upquote}}{}
\IfFileExists{microtype.sty}{% use microtype if available
  \usepackage[]{microtype}
  \UseMicrotypeSet[protrusion]{basicmath} % disable protrusion for tt fonts
}{}
\makeatletter
\@ifundefined{KOMAClassName}{% if non-KOMA class
  \IfFileExists{parskip.sty}{%
    \usepackage{parskip}
  }{% else
    \setlength{\parindent}{0pt}
    \setlength{\parskip}{6pt plus 2pt minus 1pt}}
}{% if KOMA class
  \KOMAoptions{parskip=half}}
\makeatother
\usepackage{xcolor}
\IfFileExists{xurl.sty}{\usepackage{xurl}}{} % add URL line breaks if available
\IfFileExists{bookmark.sty}{\usepackage{bookmark}}{\usepackage{hyperref}}
\hypersetup{
  pdftitle={444 Lecture 7.1 - Bayesian Equilibrium},
  pdfauthor={Brian Weatherson},
  hidelinks,
  pdfcreator={LaTeX via pandoc}}
\urlstyle{same} % disable monospaced font for URLs
\newif\ifbibliography
\setlength{\emergencystretch}{3em} % prevent overfull lines
\providecommand{\tightlist}{%
  \setlength{\itemsep}{0pt}\setlength{\parskip}{0pt}}
\setcounter{secnumdepth}{-\maxdimen} % remove section numbering
\let\Tiny=\tiny

 \setbeamertemplate{navigation symbols}{} 

% \usetheme{Madrid}
 \usetheme[numbering=none, progressbar=foot]{metropolis}
 \usecolortheme{wolverine}
 \usepackage{color}
 \usepackage{MnSymbol}
% \usepackage{movie15}

\usepackage{amssymb}% http://ctan.org/pkg/amssymb
\usepackage{pifont}% http://ctan.org/pkg/pifont
\newcommand{\cmark}{\ding{51}}%
\newcommand{\xmark}{\ding{55}}%

\DeclareSymbolFont{symbolsC}{U}{txsyc}{m}{n}
\DeclareMathSymbol{\boxright}{\mathrel}{symbolsC}{128}
\DeclareMathAlphabet{\mathpzc}{OT1}{pzc}{m}{it}

\setlength{\parskip}{1ex plus 0.5ex minus 0.2ex}

\AtBeginSection[]
{
\begin{frame}
	\Huge{\color{darkblue} \insertsection}
\end{frame}
}

\renewenvironment*{quote}	
	{\list{}{\rightmargin   \leftmargin} \item } 	
	{\endlist }

\definecolor{darkgreen}{rgb}{0,0.7,0}
\definecolor{darkblue}{rgb}{0,0,0.8}

\usepackage[italic]{mathastext}
\usepackage{nicefrac}
\usepackage{istgame}

\setbeamertemplate{caption}{\raggedright\insertcaption}

%\def\toprule{}
%\def\bottomrule{}
%\def\midrule{}
\usepackage{etoolbox}
\AfterEndEnvironment{description}{\vspace{9pt}}
\AfterEndEnvironment{oltableau}{\vspace{9pt}}
\BeforeBeginEnvironment{oltableau}{\vspace{9pt}}
\AfterEndEnvironment{center}{\vspace{9pt}}
\BeforeBeginEnvironment{tabular}{\vspace{9pt}}
\AfterEndEnvironment{longtable}{\vspace{-6pt}}
\usepackage{booktabs}
\usepackage{longtable}
\usepackage{array}
\usepackage{multirow}
\usepackage{wrapfig}
\usepackage{float}
\usepackage{colortbl}
\usepackage{pdflscape}
\usepackage{tabu}
\usepackage{threeparttable} 
\usepackage{threeparttablex} 
\usepackage[normalem]{ulem} 
\usepackage{makecell}
\usepackage{xcolor}
\usepackage{ulem}

\setlength\heavyrulewidth{0ex}
\setlength\lightrulewidth{0.08ex}

\aboverulesep=0ex
\belowrulesep=0ex
\renewcommand{\arraystretch}{1.2}
\ifluatex
  \usepackage{selnolig}  % disable illegal ligatures
\fi

\title{444 Lecture 7.1 - Bayesian Equilibrium}
\author{Brian Weatherson}
\date{}

\begin{document}
\frame{\titlepage}

\begin{frame}{What is Rational Here?}
\protect\hypertarget{what-is-rational-here}{}
\begin{center}
\begin{istgame}
%\setistgrowdirection'{east}
\xtdistance{15mm}{30mm}
\istroot(0){Alice}
  \istb{A}[al]{(1,1)}
  \istb{B}[r]
  \istb{C}[ar]
  \endist
\xtdistance{10mm}{20mm}
\istroot(1)(0-2)
  \istb{l}[al]{(2,2)}
  \istb{r}[ar]{(0,0)}
  \endist
\istroot(2)(0-3)
  \istb{l}[al]{(2,2)}
  \istb{r}[ar]{(0,0)}
  \endist
\xtInfoset(1)(2){Billie}
\end{istgame}
\end{center}

\begin{itemize}[<+->]
\tightlist
\item
  Intuitively we should end up with one of the 2,2 outcomes.
\item
  But how theoretically can we get that?
\end{itemize}
\end{frame}

\begin{frame}{Strategy Table}
\protect\hypertarget{strategy-table}{}
\begin{table}[!h]
\centering
\begin{tabular}[t]{>{}r|cc}
\toprule
 & l & r\\
\midrule
A & 1, 1 & 1, 1\\
B & 2, 2 & 0, 0\\
C & 2, 2 & 0, 0\\
\bottomrule
\end{tabular}
\end{table}

Note that \(\langle A, r \rangle\) is a Nash equilibrium.
\end{frame}

\begin{frame}{Subgame Perfection}
\protect\hypertarget{subgame-perfection}{}
\begin{center}
\begin{istgame}
%\setistgrowdirection'{east}
\xtdistance{15mm}{30mm}
\istroot(0){Alice}
  \istb{A}[al]{(1,1)}
  \istb{B}[r]
  \istb{C}[ar]
  \endist
\xtdistance{10mm}{20mm}
\istroot(1)(0-2)
  \istb{l}[al]{(2,2)}
  \istb{r}[ar]{(0,0)}
  \endist
\istroot(2)(0-3)
  \istb{l}[al]{(2,2)}
  \istb{r}[ar]{(0,0)}
  \endist
\xtInfoset(1)(2){Billie}
\end{istgame}
\end{center}

\begin{itemize}
\tightlist
\item
  There are no subgames (think about why) - so \(\langle A, r \rangle\)
  is also subgame perfect.
\end{itemize}
\end{frame}

\begin{frame}{Intuition}
\protect\hypertarget{intuition}{}
\begin{itemize}
\tightlist
\item
  It is absurd for Billie to play r if it gets that far.
\item
  We need a theory that says this is absurd.
\item
  This is just the kind of thing subgame perfect equilibrium was
  introduced for, but it isn't working for technical reasons about the
  definition of subgames.
\item
  I'm not sure if there is a completely standard solution here, but I
  wanted to set out an approach that's consistent with current
  philosophy.
\end{itemize}
\end{frame}

\begin{frame}{Bayesian Equilibrium}
\protect\hypertarget{bayesian-equilibrium}{}
\begin{itemize}
\tightlist
\item
  An equilibrium is a pair of behavioral dispositions.
\item
  Each behavioral disposition gives a probability of each choice at each
  node the player may have to choose at.
\item
  In equilibrium, each player has their own disposition from the pair,
  and believes (with certainty) that the other player has the other
  disposition from the pair.
\item
  If there is a move made by Nature, each player has the correct
  probability for each of Nature's possible moves.
\item
  At every stage, each Player maximises expected utility given their
  beliefs about the other.
\end{itemize}
\end{frame}

\begin{frame}{Re-formulation}
\protect\hypertarget{re-formulation}{}
This is more or less equivalent.

\begin{itemize}
\tightlist
\item
  Each player starts with a probability distribution over outcomes of
  the game.
\item
  In equilibrium, these are the same, and the players have correct
  probabilities about the moves nature will make.
\item
  When something happens (a move is revealed), the players update by
  conditionalisation if the event has positive probability.
\item
  When something unexpected (probability zero) happens, the players just
  pick a new probability.
\item
  In equilibrium, both players know how both players are disposed to
  react in each of these cases.
\item
  Everyone is always maximising expected utility.
\end{itemize}
\end{frame}

\begin{frame}{Back to the Game}
\protect\hypertarget{back-to-the-game}{}
\begin{center}
\begin{istgame}
%\setistgrowdirection'{east}
\xtdistance{15mm}{30mm}
\istroot(0){Alice}
  \istb{A}[al]{(1,1)}
  \istb{B}[r]
  \istb{C}[ar]
  \endist
\xtdistance{10mm}{20mm}
\istroot(1)(0-2)
  \istb{l}[al]{(2,2)}
  \istb{r}[ar]{(0,0)}
  \endist
\istroot(2)(0-3)
  \istb{l}[al]{(2,2)}
  \istb{r}[ar]{(0,0)}
  \endist
\xtInfoset(1)(2){Billie}
\end{istgame}
\end{center}

\begin{itemize}[<+->]
\tightlist
\item
  At the B/C information set, Billie must have some probability between
  B and C.
\item
  Whatever it is, l has higher expected utility that r.
\item
  Alice knows this, so will choose B or C rather than A.
\end{itemize}
\end{frame}

\begin{frame}{For Next Time}
\protect\hypertarget{for-next-time}{}
\begin{itemize}
\tightlist
\item
  I'll look at two puzzle cases.
\item
  If you're pressed for time, skip ahead to the following lecture.
\end{itemize}
\end{frame}

\end{document}
