% Options for packages loaded elsewhere
\PassOptionsToPackage{unicode}{hyperref}
\PassOptionsToPackage{hyphens}{url}
%
\documentclass[
  ignorenonframetext,
]{beamer}
\usepackage{pgfpages}
\setbeamertemplate{caption}[numbered]
\setbeamertemplate{caption label separator}{: }
\setbeamercolor{caption name}{fg=normal text.fg}
\beamertemplatenavigationsymbolsempty
% Prevent slide breaks in the middle of a paragraph
\widowpenalties 1 10000
\raggedbottom
\setbeamertemplate{part page}{
  \centering
  \begin{beamercolorbox}[sep=16pt,center]{part title}
    \usebeamerfont{part title}\insertpart\par
  \end{beamercolorbox}
}
\setbeamertemplate{section page}{
  \centering
  \begin{beamercolorbox}[sep=12pt,center]{part title}
    \usebeamerfont{section title}\insertsection\par
  \end{beamercolorbox}
}
\setbeamertemplate{subsection page}{
  \centering
  \begin{beamercolorbox}[sep=8pt,center]{part title}
    \usebeamerfont{subsection title}\insertsubsection\par
  \end{beamercolorbox}
}
\AtBeginPart{
  \frame{\partpage}
}
\AtBeginSection{
  \ifbibliography
  \else
    \frame{\sectionpage}
  \fi
}
\AtBeginSubsection{
  \frame{\subsectionpage}
}
\usepackage{amsmath,amssymb}
\usepackage{lmodern}
\usepackage{ifxetex,ifluatex}
\ifnum 0\ifxetex 1\fi\ifluatex 1\fi=0 % if pdftex
  \usepackage[T1]{fontenc}
  \usepackage[utf8]{inputenc}
  \usepackage{textcomp} % provide euro and other symbols
\else % if luatex or xetex
  \usepackage{unicode-math}
  \defaultfontfeatures{Scale=MatchLowercase}
  \defaultfontfeatures[\rmfamily]{Ligatures=TeX,Scale=1}
  \setmainfont[BoldFont = SF Pro Rounded Semibold]{SF Pro Rounded}
  \setmathfont[]{STIX Two Math}
\fi
\usefonttheme{serif} % use mainfont rather than sansfont for slide text
% Use upquote if available, for straight quotes in verbatim environments
\IfFileExists{upquote.sty}{\usepackage{upquote}}{}
\IfFileExists{microtype.sty}{% use microtype if available
  \usepackage[]{microtype}
  \UseMicrotypeSet[protrusion]{basicmath} % disable protrusion for tt fonts
}{}
\makeatletter
\@ifundefined{KOMAClassName}{% if non-KOMA class
  \IfFileExists{parskip.sty}{%
    \usepackage{parskip}
  }{% else
    \setlength{\parindent}{0pt}
    \setlength{\parskip}{6pt plus 2pt minus 1pt}}
}{% if KOMA class
  \KOMAoptions{parskip=half}}
\makeatother
\usepackage{xcolor}
\IfFileExists{xurl.sty}{\usepackage{xurl}}{} % add URL line breaks if available
\IfFileExists{bookmark.sty}{\usepackage{bookmark}}{\usepackage{hyperref}}
\hypersetup{
  pdftitle={444 Lecture 7.3 - Signaling Game},
  pdfauthor={Brian Weatherson},
  hidelinks,
  pdfcreator={LaTeX via pandoc}}
\urlstyle{same} % disable monospaced font for URLs
\newif\ifbibliography
\setlength{\emergencystretch}{3em} % prevent overfull lines
\providecommand{\tightlist}{%
  \setlength{\itemsep}{0pt}\setlength{\parskip}{0pt}}
\setcounter{secnumdepth}{-\maxdimen} % remove section numbering
\let\Tiny=\tiny

 \setbeamertemplate{navigation symbols}{} 

% \usetheme{Madrid}
 \usetheme[numbering=none, progressbar=foot]{metropolis}
 \usecolortheme{wolverine}
 \usepackage{color}
 \usepackage{MnSymbol}
% \usepackage{movie15}

\usepackage{amssymb}% http://ctan.org/pkg/amssymb
\usepackage{pifont}% http://ctan.org/pkg/pifont
\newcommand{\cmark}{\ding{51}}%
\newcommand{\xmark}{\ding{55}}%

\DeclareSymbolFont{symbolsC}{U}{txsyc}{m}{n}
\DeclareMathSymbol{\boxright}{\mathrel}{symbolsC}{128}
\DeclareMathAlphabet{\mathpzc}{OT1}{pzc}{m}{it}

\setlength{\parskip}{1ex plus 0.5ex minus 0.2ex}

\AtBeginSection[]
{
\begin{frame}
	\Huge{\color{darkblue} \insertsection}
\end{frame}
}

\renewenvironment*{quote}	
	{\list{}{\rightmargin   \leftmargin} \item } 	
	{\endlist }

\definecolor{darkgreen}{rgb}{0,0.7,0}
\definecolor{darkblue}{rgb}{0,0,0.8}

\usepackage[italic]{mathastext}
\usepackage{nicefrac}
\usepackage{istgame}

\setbeamertemplate{caption}{\raggedright\insertcaption}

%\def\toprule{}
%\def\bottomrule{}
%\def\midrule{}
\usepackage{etoolbox}
\AfterEndEnvironment{description}{\vspace{9pt}}
\AfterEndEnvironment{oltableau}{\vspace{9pt}}
\BeforeBeginEnvironment{oltableau}{\vspace{9pt}}
\AfterEndEnvironment{center}{\vspace{9pt}}
\BeforeBeginEnvironment{tabular}{\vspace{9pt}}
\AfterEndEnvironment{longtable}{\vspace{-6pt}}
\usepackage{booktabs}
\usepackage{longtable}
\usepackage{array}
\usepackage{multirow}
\usepackage{wrapfig}
\usepackage{float}
\usepackage{colortbl}
\usepackage{pdflscape}
\usepackage{tabu}
\usepackage{threeparttable} 
\usepackage{threeparttablex} 
\usepackage[normalem]{ulem} 
\usepackage{makecell}
\usepackage{xcolor}
\usepackage{ulem}

\setlength\heavyrulewidth{0ex}
\setlength\lightrulewidth{0.08ex}

\aboverulesep=0ex
\belowrulesep=0ex
\renewcommand{\arraystretch}{1.2}
\ifluatex
  \usepackage{selnolig}  % disable illegal ligatures
\fi

\title{444 Lecture 7.3 - Signaling Game}
\author{Brian Weatherson}
\date{}

\begin{document}
\frame{\titlepage}

\begin{frame}
\begin{center}
\begin{istgame}[scale=0.9]
   \xtdistance{20mm}{20mm}
   \istroot(0)[chance node]{$c$}
     \istb<grow=left>{0.6}[a]
     \istb<grow=right>{0.4}[a]
     \endist
   \xtdistance{10mm}{20mm}
   \istroot(1)(0-1)<180>{1}
     \istb<grow=north>{a}[l]
     \istb<grow=south>{b}[l]
     \endist
   \istroot(2)(0-2)<0>{1}
     \istb<grow=north>{a}[r]
     \istb<grow=south>{b}[r]
     \endist
   \istroot'[north](a1)(1-1)
     \istb{L}[bl]{1,1}
     \istb{R}[br]{0,0}
     \endist
   \istroot(b1)(1-2)
     \istb{L}[al]{1,1}
     \istb{R}[ar]{0,0}
     \endist
   \istroot(a2)(2-2)
     \istb{L}[al]{0,0}
     \istb{R}[ar]{1,1}
     \endist
   \istroot'[north](b2)(2-1)
     \istb{L}[bl]{0,0}
     \istb{R}[br]{1,1}
     \endist
   \xtInfoset(a1)(b2){2}
   \xtInfoset(b1)(a2){2}
   \end{istgame}
\end{center}

This is a basic signaling game.
\end{frame}

\begin{frame}
\begin{center}
\begin{istgame}[scale=0.9]
   \xtdistance{20mm}{20mm}
   \istroot(0)[chance node]{$c$}
     \istb<grow=left>{0.6}[a]
     \istb<grow=right>{0.4}[a]
     \endist
   \xtdistance{10mm}{20mm}
   \istroot(1)(0-1)<180>{1}
     \istb<grow=north>{a}[l]
     \istb<grow=south>{b}[l]
     \endist
   \istroot(2)(0-2)<0>{1}
     \istb<grow=north>{a}[r]
     \istb<grow=south>{b}[r]
     \endist
   \istroot'[north](a1)(1-1)
     \istb{L}[bl]{1,1}
     \istb{R}[br]{0,0}
     \endist
   \istroot(b1)(1-2)
     \istb{L}[al]{1,1}
     \istb{R}[ar]{0,0}
     \endist
   \istroot(a2)(2-2)
     \istb{L}[al]{0,0}
     \istb{R}[ar]{1,1}
     \endist
   \istroot'[north](b2)(2-1)
     \istb{L}[bl]{0,0}
     \istb{R}[br]{1,1}
     \endist
   \xtInfoset(a1)(b2){2}
   \xtInfoset(b1)(a2){2}
   \end{istgame}
\end{center}

\begin{enumerate}
\tightlist
\item
  Nature moves left or right, and reveals this to Sender (1).
\item
  Sender moves up or down, and reveals this to Hearer (2).
\item
  Hearer moves left or right.
\end{enumerate}
\end{frame}

\begin{frame}
\begin{center}
\begin{istgame}[scale=0.9]
   \xtdistance{20mm}{20mm}
   \istroot(0)[chance node]{$c$}
     \istb<grow=left>{0.6}[a]
     \istb<grow=right>{0.4}[a]
     \endist
   \xtdistance{10mm}{20mm}
   \istroot(1)(0-1)<180>{1}
     \istb<grow=north>{a}[l]
     \istb<grow=south>{b}[l]
     \endist
   \istroot(2)(0-2)<0>{1}
     \istb<grow=north>{a}[r]
     \istb<grow=south>{b}[r]
     \endist
   \istroot'[north](a1)(1-1)
     \istb{L}[bl]{1,1}
     \istb{R}[br]{0,0}
     \endist
   \istroot(b1)(1-2)
     \istb{L}[al]{1,1}
     \istb{R}[ar]{0,0}
     \endist
   \istroot(a2)(2-2)
     \istb{L}[al]{0,0}
     \istb{R}[ar]{1,1}
     \endist
   \istroot'[north](b2)(2-1)
     \istb{L}[bl]{0,0}
     \istb{R}[br]{1,1}
     \endist
   \xtInfoset(a1)(b2){2}
   \xtInfoset(b1)(a2){2}
   \end{istgame}
\end{center}

\begin{itemize}
\tightlist
\item
  This is a purely cooperative version.
\item
  Each player gets 1 if Hearer makes same move as Nature, and 0
  otherwise.
\end{itemize}
\end{frame}

\begin{frame}{Four Equilibria}
\protect\hypertarget{four-equilibria}{}
\begin{enumerate}[<+->]
\tightlist
\item
  Sender does Up if Left, Down if Right. Hearer does what Nature does.
\item
  Sender does Up if Right, Down if Left. Hearer does what Nature does.
\item
  Sender does the same thing no matter what. Hearer does Left no matter
  what.
\item
  Sender speaks randomly. Hearer does Left no matter what.
\end{enumerate}
\end{frame}

\begin{frame}{Equilibria Types}
\protect\hypertarget{equilibria-types}{}
\begin{itemize}
\tightlist
\item
  The first two are \textbf{separating} equilibria. This is what we hope
  happens!
\item
  The third is a \textbf{pooling} equilibria. This is not great.
\item
  The fourth is a \textbf{babbling} equilibria. This might be worse.
\end{itemize}

When nature has only two possible states, there isn't much conceptual
difference between pooling and babbling. But in the more general case,
we want to distinguish between partially pooling equilibria and
partially babbling equilibria.
\end{frame}

\begin{frame}{Conceptual Load}
\protect\hypertarget{conceptual-load}{}
\begin{itemize}
\tightlist
\item
  We're normally interested in games where each player is perfectly
  rational, and is basically a genius game player.
\item
  But it's important for the purposes that signaling games are put that
  you can sustain a separating equilibrium when the players are really
  not very sophisticated.
\item
  Sender needs to be able to detect what Nature is doing, and
  differentially respond.
\item
  Hearer needs to differentially respond to Sender's response - so they
  also need to be able to detect what Sender is doing, and use that
  detection to guide action.
\item
  That's not nothing - two rocks couldn't do it - but it's not a lot.
\end{itemize}
\end{frame}

\begin{frame}{Tiny Players}
\protect\hypertarget{tiny-players}{}
\begin{itemize}
\tightlist
\item
  Cells, for example, can detect some things and differentially respond.
\item
  Certainly animals can do this.
\item
  And probably a lot of plants can as well, at least for some possible
  signals.
\item
  So you can use signaling games to explain all sorts of biological
  interactions, both between organisms and within them.
\item
  But this isn't a (philosophy of) biology class, so I'll leave that to
  another day.
\end{itemize}
\end{frame}

\end{document}
