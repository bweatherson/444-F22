% Options for packages loaded elsewhere
\PassOptionsToPackage{unicode}{hyperref}
\PassOptionsToPackage{hyphens}{url}
%
\documentclass[
  ignorenonframetext,
]{beamer}
\usepackage{pgfpages}
\setbeamertemplate{caption}[numbered]
\setbeamertemplate{caption label separator}{: }
\setbeamercolor{caption name}{fg=normal text.fg}
\beamertemplatenavigationsymbolsempty
% Prevent slide breaks in the middle of a paragraph
\widowpenalties 1 10000
\raggedbottom
\setbeamertemplate{part page}{
  \centering
  \begin{beamercolorbox}[sep=16pt,center]{part title}
    \usebeamerfont{part title}\insertpart\par
  \end{beamercolorbox}
}
\setbeamertemplate{section page}{
  \centering
  \begin{beamercolorbox}[sep=12pt,center]{part title}
    \usebeamerfont{section title}\insertsection\par
  \end{beamercolorbox}
}
\setbeamertemplate{subsection page}{
  \centering
  \begin{beamercolorbox}[sep=8pt,center]{part title}
    \usebeamerfont{subsection title}\insertsubsection\par
  \end{beamercolorbox}
}
\AtBeginPart{
  \frame{\partpage}
}
\AtBeginSection{
  \ifbibliography
  \else
    \frame{\sectionpage}
  \fi
}
\AtBeginSubsection{
  \frame{\subsectionpage}
}
\usepackage{amsmath,amssymb}
\usepackage{lmodern}
\usepackage{ifxetex,ifluatex}
\ifnum 0\ifxetex 1\fi\ifluatex 1\fi=0 % if pdftex
  \usepackage[T1]{fontenc}
  \usepackage[utf8]{inputenc}
  \usepackage{textcomp} % provide euro and other symbols
\else % if luatex or xetex
  \usepackage{unicode-math}
  \defaultfontfeatures{Scale=MatchLowercase}
  \defaultfontfeatures[\rmfamily]{Ligatures=TeX,Scale=1}
  \setmainfont[BoldFont = SF Pro Rounded Semibold]{SF Pro Rounded}
  \setmathfont[]{STIX Two Math}
\fi
\usefonttheme{serif} % use mainfont rather than sansfont for slide text
% Use upquote if available, for straight quotes in verbatim environments
\IfFileExists{upquote.sty}{\usepackage{upquote}}{}
\IfFileExists{microtype.sty}{% use microtype if available
  \usepackage[]{microtype}
  \UseMicrotypeSet[protrusion]{basicmath} % disable protrusion for tt fonts
}{}
\makeatletter
\@ifundefined{KOMAClassName}{% if non-KOMA class
  \IfFileExists{parskip.sty}{%
    \usepackage{parskip}
  }{% else
    \setlength{\parindent}{0pt}
    \setlength{\parskip}{6pt plus 2pt minus 1pt}}
}{% if KOMA class
  \KOMAoptions{parskip=half}}
\makeatother
\usepackage{xcolor}
\IfFileExists{xurl.sty}{\usepackage{xurl}}{} % add URL line breaks if available
\IfFileExists{bookmark.sty}{\usepackage{bookmark}}{\usepackage{hyperref}}
\hypersetup{
  pdftitle={444 Lecture 12.2 - Bratman on Group Action},
  pdfauthor={Brian Weatherson},
  hidelinks,
  pdfcreator={LaTeX via pandoc}}
\urlstyle{same} % disable monospaced font for URLs
\newif\ifbibliography
\setlength{\emergencystretch}{3em} % prevent overfull lines
\providecommand{\tightlist}{%
  \setlength{\itemsep}{0pt}\setlength{\parskip}{0pt}}
\setcounter{secnumdepth}{-\maxdimen} % remove section numbering
\let\Tiny=\tiny

 \setbeamertemplate{navigation symbols}{} 

% \usetheme{Madrid}
 \usetheme[numbering=none, progressbar=foot]{metropolis}
 \usecolortheme{wolverine}
 \usepackage{color}
 \usepackage{MnSymbol}
% \usepackage{movie15}

\usepackage{amssymb}% http://ctan.org/pkg/amssymb
\usepackage{pifont}% http://ctan.org/pkg/pifont
\newcommand{\cmark}{\ding{51}}%
\newcommand{\xmark}{\ding{55}}%

\DeclareSymbolFont{symbolsC}{U}{txsyc}{m}{n}
\DeclareMathSymbol{\boxright}{\mathrel}{symbolsC}{128}
\DeclareMathAlphabet{\mathpzc}{OT1}{pzc}{m}{it}

\setlength{\parskip}{1ex plus 0.5ex minus 0.2ex}

\AtBeginSection[]
{
\begin{frame}
	\Huge{\color{darkblue} \insertsection}
\end{frame}
}

\renewenvironment*{quote}	
	{\list{}{\rightmargin   \leftmargin} \item } 	
	{\endlist }

\definecolor{darkgreen}{rgb}{0,0.7,0}
\definecolor{darkblue}{rgb}{0,0,0.8}

\usepackage[italic]{mathastext}
\usepackage{nicefrac}
\usepackage{istgame}

\setbeamertemplate{caption}{\raggedright\insertcaption}

%\def\toprule{}
%\def\bottomrule{}
%\def\midrule{}
\usepackage{etoolbox}
\AfterEndEnvironment{description}{\vspace{9pt}}
\AfterEndEnvironment{oltableau}{\vspace{9pt}}
\BeforeBeginEnvironment{oltableau}{\vspace{9pt}}
\AfterEndEnvironment{center}{\vspace{9pt}}
\BeforeBeginEnvironment{tabular}{\vspace{9pt}}
\AfterEndEnvironment{longtable}{\vspace{-6pt}}
\usepackage{booktabs}
\usepackage{longtable}
\usepackage{array}
\usepackage{multirow}
\usepackage{wrapfig}
\usepackage{float}
\usepackage{colortbl}
\usepackage{pdflscape}
\usepackage{tabu}
\usepackage{threeparttable} 
\usepackage{threeparttablex} 
\usepackage[normalem]{ulem} 
\usepackage{makecell}
\usepackage{xcolor}
\usepackage{ulem}

\setlength\heavyrulewidth{0ex}
\setlength\lightrulewidth{0.08ex}

\aboverulesep=0ex
\belowrulesep=0ex
\renewcommand{\arraystretch}{1.2}
\AtBeginSection[]
{
    \begin{frame}
        \frametitle{Day Plan}
        \tableofcontents[currentsection]
    \end{frame}
}
\ifluatex
  \usepackage{selnolig}  % disable illegal ligatures
\fi

\title{444 Lecture 12.2 - Bratman on Group Action}
\author{Brian Weatherson}
\date{}

\begin{document}
\frame{\titlepage}

\begin{frame}{Strategy}
\protect\hypertarget{strategy}{}
\begin{itemize}
\tightlist
\item
  Like Gilbert, Bratman starts with a simple case, and builds up.
\item
  But for Bratman, the building isn't an impressionistic picture; it's a
  set of conditions.
\item
  Of course, these conditions don't quite work in puzzle cases, so we
  add complications.
\item
  The history of philosophy suggests this path does not have a great
  record of success.
\end{itemize}
\end{frame}

\begin{frame}{Three Conditions}
\protect\hypertarget{three-conditions}{}
\begin{enumerate}
\tightlist
\item
  Mutual Responsiveness
\item
  Commitment to joint activity; i.e., we both intend to do this very
  activity, under something like this description.
\item
  Commitment to mutual support; i.e., we both intend to help the other
  should they falter, and not claim all the glory.
\end{enumerate}

The last condition is a strengthening of the idea that cooperative
activity is not side-by-side activity.
\end{frame}

\begin{frame}{Many Person Groups}
\protect\hypertarget{many-person-groups}{}
\begin{itemize}
\tightlist
\item
  Like with Gilbert, you might worry about the generalisation of these
  to many person groups.
\item
  In a large group, I can intend that this group activity happen without
  having any commitment to being part of it.
\item
  That can't happen in a two person group; if I leave, the group ceases
  to exist.
\item
  So there is this tricky question about what kind of commitment is
  needed on the part of each individual for a large group activity to
  persist.
\item
  Thinking about two person cases is unlikely to help clarify that.
\end{itemize}
\end{frame}

\begin{frame}{Extra-Personal Intentions}
\protect\hypertarget{extra-personal-intentions}{}
\begin{itemize}
\tightlist
\item
  One of the central moves Bratman makes is that each person
  individually intends that the group does something.
\item
  You might think this is odd; I can only intend that I do things.
  \pause
\item
  But really there are lots of cases where I intend something not
  entirely in my control.
\item
  I can intend to spend a sunny day at the beach, without intending the
  sunshine. \pause
\item
  I can even, I think, do it without being 100\% sure of the sunshine.
  \pause
\item
  Another example: I can intend to holiday in Paris, although I can't
  control all the aspects of my getting to Paris.
\end{itemize}
\end{frame}

\begin{frame}{Mesh}
\protect\hypertarget{mesh}{}
\begin{itemize}
\tightlist
\item
  Bratman's idea that plans should mesh is, I think, a really nice way
  of splitting the difference between the views that our plans must
  match, and that there is no constraint on mutual plans.
\item
  Matching plans is too strong; I don't need to have views about what
  you do.
\item
  No constraints is too weak; it isn't a joint activity if I don't have
  some kinds of vetos.
\item
  Mesh is a nice attempt to get something between these.
\end{itemize}
\end{frame}

\begin{frame}{Problems for Mesh}
\protect\hypertarget{problems-for-mesh}{}
\begin{itemize}
\tightlist
\item
  But as stated it feels too strong.
\item
  Imagine that your job is to get the paint.
\item
  I have views about where to get the paint from (as in Bratman's
  example), but also how to drive there.
\item
  This feels like it shouldn't matter; it's your job to get the paint.
\end{itemize}
\end{frame}

\begin{frame}{Mesh and Counterfactuals}
\protect\hypertarget{mesh-and-counterfactuals}{}
From the other direction, it's fascinating to think through cases that
turn on how counterfactually resilient mesh must be.

\begin{itemize}
\tightlist
\item
  Some level of resiliency is needed. If we are working side by side on
  different projects, the fact that they happen to mesh doesn't make
  them joint projects.
\item
  But total counterfactual resiliency isn't needed either. I can be in a
  group with you, but be disposed to leave if you insist on singing
  arias while we work.
\item
  I suspect there will be some vague cases in the middle here.
\end{itemize}
\end{frame}

\begin{frame}{Coercion}
\protect\hypertarget{coercion}{}
\begin{itemize}
\tightlist
\item
  Assuming that everyone plans to stay in the group, and to be
  cooperative, it feels we should give each other some flexibility in
  how they do their jobs.
\item
  There is something vaguely coercive about even having views about how
  you should get to the store to buy the paint.
\item
  Of course, it's fine to be helpful, and there isn't really anything
  wrong with having views about what is better and worse.
\item
  I don't really know to balance these considerations.
\end{itemize}
\end{frame}

\begin{frame}{What Counts as Support}
\protect\hypertarget{what-counts-as-support}{}
\begin{itemize}
\tightlist
\item
  The single possible kind of support feels really weak.
\item
  What if there is a kind of thing I can't stand seeing anyone suffer
  through?
\item
  Feels like we need a generic here not an existential.
\end{itemize}
\end{frame}

\begin{frame}{Are Competitive Games SCAs, or Group Actions}
\protect\hypertarget{are-competitive-games-scas-or-group-actions}{}
\begin{itemize}
\tightlist
\item
  Last case, because this is both a problem for Bratman and an
  interesting puzzle that tells us something about the stakes of these
  questions.
\item
  Is playing chess with a friend a shared cooperative activity?
\item
  It doesn't satisfy mesh, or support. I want to upset your plans.
\item
  But in some ways it is; a chess game involves a fair amount of
  coordination and cooperation. \pause
\item
  What turns on calling it a shared activity or not?
\end{itemize}
\end{frame}

\end{document}
