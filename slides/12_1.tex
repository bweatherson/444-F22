% Options for packages loaded elsewhere
\PassOptionsToPackage{unicode}{hyperref}
\PassOptionsToPackage{hyphens}{url}
%
\documentclass[
  ignorenonframetext,
]{beamer}
\usepackage{pgfpages}
\setbeamertemplate{caption}[numbered]
\setbeamertemplate{caption label separator}{: }
\setbeamercolor{caption name}{fg=normal text.fg}
\beamertemplatenavigationsymbolsempty
% Prevent slide breaks in the middle of a paragraph
\widowpenalties 1 10000
\raggedbottom
\setbeamertemplate{part page}{
  \centering
  \begin{beamercolorbox}[sep=16pt,center]{part title}
    \usebeamerfont{part title}\insertpart\par
  \end{beamercolorbox}
}
\setbeamertemplate{section page}{
  \centering
  \begin{beamercolorbox}[sep=12pt,center]{part title}
    \usebeamerfont{section title}\insertsection\par
  \end{beamercolorbox}
}
\setbeamertemplate{subsection page}{
  \centering
  \begin{beamercolorbox}[sep=8pt,center]{part title}
    \usebeamerfont{subsection title}\insertsubsection\par
  \end{beamercolorbox}
}
\AtBeginPart{
  \frame{\partpage}
}
\AtBeginSection{
  \ifbibliography
  \else
    \frame{\sectionpage}
  \fi
}
\AtBeginSubsection{
  \frame{\subsectionpage}
}
\usepackage{amsmath,amssymb}
\usepackage{lmodern}
\usepackage{ifxetex,ifluatex}
\ifnum 0\ifxetex 1\fi\ifluatex 1\fi=0 % if pdftex
  \usepackage[T1]{fontenc}
  \usepackage[utf8]{inputenc}
  \usepackage{textcomp} % provide euro and other symbols
\else % if luatex or xetex
  \usepackage{unicode-math}
  \defaultfontfeatures{Scale=MatchLowercase}
  \defaultfontfeatures[\rmfamily]{Ligatures=TeX,Scale=1}
  \setmainfont[BoldFont = SF Pro Rounded Semibold]{SF Pro Rounded}
  \setmathfont[]{STIX Two Math}
\fi
\usefonttheme{serif} % use mainfont rather than sansfont for slide text
% Use upquote if available, for straight quotes in verbatim environments
\IfFileExists{upquote.sty}{\usepackage{upquote}}{}
\IfFileExists{microtype.sty}{% use microtype if available
  \usepackage[]{microtype}
  \UseMicrotypeSet[protrusion]{basicmath} % disable protrusion for tt fonts
}{}
\makeatletter
\@ifundefined{KOMAClassName}{% if non-KOMA class
  \IfFileExists{parskip.sty}{%
    \usepackage{parskip}
  }{% else
    \setlength{\parindent}{0pt}
    \setlength{\parskip}{6pt plus 2pt minus 1pt}}
}{% if KOMA class
  \KOMAoptions{parskip=half}}
\makeatother
\usepackage{xcolor}
\IfFileExists{xurl.sty}{\usepackage{xurl}}{} % add URL line breaks if available
\IfFileExists{bookmark.sty}{\usepackage{bookmark}}{\usepackage{hyperref}}
\hypersetup{
  pdftitle={444 Lecture 12.1 - Gilbert on Group Action},
  pdfauthor={Brian Weatherson},
  hidelinks,
  pdfcreator={LaTeX via pandoc}}
\urlstyle{same} % disable monospaced font for URLs
\newif\ifbibliography
\setlength{\emergencystretch}{3em} % prevent overfull lines
\providecommand{\tightlist}{%
  \setlength{\itemsep}{0pt}\setlength{\parskip}{0pt}}
\setcounter{secnumdepth}{-\maxdimen} % remove section numbering
\let\Tiny=\tiny

 \setbeamertemplate{navigation symbols}{} 

% \usetheme{Madrid}
 \usetheme[numbering=none, progressbar=foot]{metropolis}
 \usecolortheme{wolverine}
 \usepackage{color}
 \usepackage{MnSymbol}
% \usepackage{movie15}

\usepackage{amssymb}% http://ctan.org/pkg/amssymb
\usepackage{pifont}% http://ctan.org/pkg/pifont
\newcommand{\cmark}{\ding{51}}%
\newcommand{\xmark}{\ding{55}}%

\DeclareSymbolFont{symbolsC}{U}{txsyc}{m}{n}
\DeclareMathSymbol{\boxright}{\mathrel}{symbolsC}{128}
\DeclareMathAlphabet{\mathpzc}{OT1}{pzc}{m}{it}

\setlength{\parskip}{1ex plus 0.5ex minus 0.2ex}

\AtBeginSection[]
{
\begin{frame}
	\Huge{\color{darkblue} \insertsection}
\end{frame}
}

\renewenvironment*{quote}	
	{\list{}{\rightmargin   \leftmargin} \item } 	
	{\endlist }

\definecolor{darkgreen}{rgb}{0,0.7,0}
\definecolor{darkblue}{rgb}{0,0,0.8}

\usepackage[italic]{mathastext}
\usepackage{nicefrac}
\usepackage{istgame}

\setbeamertemplate{caption}{\raggedright\insertcaption}

%\def\toprule{}
%\def\bottomrule{}
%\def\midrule{}
\usepackage{etoolbox}
\AfterEndEnvironment{description}{\vspace{9pt}}
\AfterEndEnvironment{oltableau}{\vspace{9pt}}
\BeforeBeginEnvironment{oltableau}{\vspace{9pt}}
\AfterEndEnvironment{center}{\vspace{9pt}}
\BeforeBeginEnvironment{tabular}{\vspace{9pt}}
\AfterEndEnvironment{longtable}{\vspace{-6pt}}
\usepackage{booktabs}
\usepackage{longtable}
\usepackage{array}
\usepackage{multirow}
\usepackage{wrapfig}
\usepackage{float}
\usepackage{colortbl}
\usepackage{pdflscape}
\usepackage{tabu}
\usepackage{threeparttable} 
\usepackage{threeparttablex} 
\usepackage[normalem]{ulem} 
\usepackage{makecell}
\usepackage{xcolor}
\usepackage{ulem}

\setlength\heavyrulewidth{0ex}
\setlength\lightrulewidth{0.08ex}

\aboverulesep=0ex
\belowrulesep=0ex
\renewcommand{\arraystretch}{1.2}
\AtBeginSection[]
{
    \begin{frame}
        \frametitle{Day Plan}
        \tableofcontents[currentsection]
    \end{frame}
}
\ifluatex
  \usepackage{selnolig}  % disable illegal ligatures
\fi

\title{444 Lecture 12.1 - Gilbert on Group Action}
\author{Brian Weatherson}
\date{}

\begin{document}
\frame{\titlepage}

\begin{frame}{Very Big Picture}
\protect\hypertarget{very-big-picture}{}
We talk all the time about groups as if they are agents.

\begin{itemize}[<+->]
\tightlist
\item
  They have doxastic states: beliefs, suspicions, knowledge.
\item
  They have desire-like states: hopes, plans, intentions.
\item
  They do things.
\end{itemize}
\end{frame}

\begin{frame}{Three Big Questions}
\protect\hypertarget{three-big-questions}{}
\begin{enumerate}
\tightlist
\item
  Are these claims literally true, or are they just figures of speech?
\item
  If they are true, when are they true? If they are not, when are they
  appropriate?
\item
  What turns on the answers to 1 and 2?
\end{enumerate}
\end{frame}

\begin{frame}{Gilbert on Group Action}
\protect\hypertarget{gilbert-on-group-action}{}
\begin{itemize}
\tightlist
\item
  Start with a picture of a very simple group - two people walking
  together.
\item
  Leverage that into a picture of what it is for groups to act.
\item
  The picture will eventually include groups having other states - the
  group will be an agent.
\item
  But that's not quite how Gilbert builds things up.
\end{itemize}
\end{frame}

\begin{frame}{Two Big Questions about Gilbert}
\protect\hypertarget{two-big-questions-about-gilbert}{}
\begin{enumerate}
\tightlist
\item
  Does Gilbert have the right analysis of ``walking together'', or other
  small group activities?
\item
  Is it the right model for larger group activities?
\end{enumerate}
\end{frame}

\begin{frame}{A Traditional Way of Thinking About Problem}
\protect\hypertarget{a-traditional-way-of-thinking-about-problem}{}
\begin{enumerate}
\tightlist
\item
  What makes some people a \emph{group}, as opposed to merely some
  people?
\item
  What makes it the case that that group is engaged in a group action,
  shares a group intention, and so on?
\end{enumerate}

Gilbert's view is that this is the wrong way to look at things. Rather,
these two questions should be answered simultaneously.
\end{frame}

\begin{frame}{Two Theories of Group Action}
\protect\hypertarget{two-theories-of-group-action}{}
\begin{description}
\tightlist
\item[Weak Shared Plan]
All the people in the group have the same plan.
\item[Strong Shared Plan]
All the people in the group have the same plan, and this is common
knowledge.
\end{description}
\end{frame}

\begin{frame}{Argument Against Weak Shared Plan}
\protect\hypertarget{argument-against-weak-shared-plan}{}
\begin{enumerate}
\tightlist
\item
  If each person is trying to do X, and thinks they are the only one
  trying to do X, then there is no group action of trying to do X.
\item
  If \textbf{Weak Shared Plan} is true, then in such a situation there
  is a group action of trying to do X.
\item
  So \textbf{Weak Shared Plan} is false.
\end{enumerate}
\end{frame}

\begin{frame}{Argument Against Strong Shared Plan}
\protect\hypertarget{argument-against-strong-shared-plan}{}
\begin{enumerate}
\tightlist
\item
  If \textbf{Strong Shared Plan} is true, then the members of the group
  have no obligation to the others to continue with the plan if they
  lose interest in it.
\item
  In cases of group action, members of the group do have an obligation
  to the others to continue with the plan even if they lose interest in
  it.
\item
  So \textbf{Strong Shared Plan} is false.
\end{enumerate}

Both parts of this are controversial. Gilbert spends time on each, first
defending 1, then clarifying 2.
\end{frame}

\begin{frame}{Trust and Reliance}
\protect\hypertarget{trust-and-reliance}{}
\begin{itemize}
\tightlist
\item
  Gilbert's objection is that \textbf{Strong Shared Plan} gives you
  mutual reliance, but it doesn't give you trust.
\item
  The distinction between reliance and trust is hard to state precisely,
  but there are very intuitive examples of reliance without trust.
\item
  Note in particular that how you can criticise someone who betrays your
  trust is very different to how you can criticise someone who you
  mistakenly relied on.
\end{itemize}
\end{frame}

\begin{frame}{Gilbert on Obligation}
\protect\hypertarget{gilbert-on-obligation}{}
\begin{enumerate}
\tightlist
\item
  If \textbf{Strong Shared Plan} was true, then members of a group could
  properly rely on each other to continue the group's operation, but
  they couldn't properly trust each other to continue the group's
  operation.
\item
  When someone abandons a group project, the criticisms we can make of
  them are more like the criticisms of people who betray a trust than
  people who let us down even though we relied on them.
\item
  So, \textbf{Strong Shared Plan} is false.
\end{enumerate}

I'm personally somewhat sceptical of 2, at least as a universal claim
about group projects.
\end{frame}

\begin{frame}{What is the Obligation to Continue}
\protect\hypertarget{what-is-the-obligation-to-continue}{}
It's not a moral obligation. Here is Gilbert's argument.

\begin{enumerate}
\tightlist
\item
  You can have shared plan between people with no concept of moral
  obligation.
\item
  If the obligation is moral obligation, that's impossible.
\item
  So the obligation is not moral obligation. \pause
\end{enumerate}

This is, I think, a bad argument. 1 is only true for psychopaths, and
not clear they can engage in group action.
\end{frame}

\begin{frame}{What is the Obligation to Continue}
\protect\hypertarget{what-is-the-obligation-to-continue-1}{}
Here is a better argument for the same conclusion.

\begin{enumerate}
\tightlist
\item
  You can have a shared plan to do an immoral thing.
\item
  You don't have moral obligations to do immoral things.
\item
  So the obligation is not a moral obligation.
\end{enumerate}
\end{frame}

\begin{frame}{What is the Obligation to Continue}
\protect\hypertarget{what-is-the-obligation-to-continue-2}{}
\begin{itemize}
\tightlist
\item
  It's also not a prudential obligation.
\item
  This should be clear, but Gilbert spends a bit of time on it. \pause
\end{itemize}

So what kind of weird sui generis obligation is it? This is a big
question for Gilbert to answer.
\end{frame}

\begin{frame}{Gilbert's Positive View}
\protect\hypertarget{gilberts-positive-view}{}
\begin{itemize}
\tightlist
\item
  That there is a group action when (and only when) the people form a
  plural subject.
\item
  So, what is a plural subject.
\end{itemize}
\end{frame}

\begin{frame}{First Person Plural}
\protect\hypertarget{first-person-plural}{}
One account is that we have a plural subject when the plurality can
literally be the subject of a sentence.

\begin{itemize}
\tightlist
\item
  This can't be right.
\item
  ``We are about to start killing each other for food'' is a well-formed
  English sentence, but the `we' there does not pick out a Gilbert-style
  group.
\end{itemize}
\end{frame}

\begin{frame}{Distributive/Collective}
\protect\hypertarget{distributivecollective}{}
There is an important distinction between distributive and collective
readings of plural sentences. The distinction turns on whether this
inference is valid.

\begin{enumerate}
\tightlist
\item
  Group G is F.
\item
  a is in group G.
\item
  Therefore, a is F.
\end{enumerate}
\end{frame}

\begin{frame}{Distributive Reading}
\protect\hypertarget{distributive-reading}{}
\begin{enumerate}
\tightlist
\item
  Our class has an exam tomorrow.
\item
  I am in the class.
\item
  Therefore, I have an exam tomorrow.
\end{enumerate}

This is the distributive reading; the group has the property because
everyone has the property.
\end{frame}

\begin{frame}{Collective Reading}
\protect\hypertarget{collective-reading}{}
\begin{enumerate}
\tightlist
\item
  Our class is surrounding the building.
\item
  I am in the class.
\item
  Therefore, I am surrounding the building.
\end{enumerate}

This is the collective reading; the group does not surround the building
in virtue of each individual surrounding the building.
\end{frame}

\begin{frame}{Plural Subjects}
\protect\hypertarget{plural-subjects}{}
A better version of Gilbert's view on pronouns is that a group is a
plural subject when they can be referred to by a first person plural
pronoun understood collectively, not distributively.

\begin{itemize}
\tightlist
\item
  This rules out really bad cases - that we have an exam tomorrow
  doesn't make us a group.
\item
  But it doesn't do enough - the example I gave earlier is collective
  not distributive.
\end{itemize}
\end{frame}

\begin{frame}{Authority and Doing}
\protect\hypertarget{authority-and-doing}{}
\begin{itemize}
\tightlist
\item
  I think at the heart of Gilbert's view is a really fascinating
  phenomena about the emergence of authority.
\item
  In some cases, the fact that a person is giving instructions and other
  people are following them gives that person a kind of authority.
\item
  By that I don't just mean the descriptive claim that their
  instructions will be followed.
\item
  I mean that some others should (in some sense) follow these
  instructions; they are doing the wrong thing if they don't.
\end{itemize}
\end{frame}

\begin{frame}{Authority and Gilbert Groups}
\protect\hypertarget{authority-and-gilbert-groups}{}
\begin{itemize}
\tightlist
\item
  If a person has this kind of authority, then some others should
  follow.
\item
  And that sort of sufficies for the people involved to form a group in
  Gilbert's sense.
\end{itemize}
\end{frame}

\begin{frame}{Duos and Gilbert Groups}
\protect\hypertarget{duos-and-gilbert-groups}{}
But there is something very special about the two person groups Gilbert
considers.

\begin{itemize}
\tightlist
\item
  No one can leave without the group dissolving.
\end{itemize}
\end{frame}

\begin{frame}{Many Person Groups}
\protect\hypertarget{many-person-groups}{}
Imagine that you're in a group, and that you have (somehow) the
followiing obligations.

\begin{enumerate}
\tightlist
\item
  To not do something that would constitute the dissolution of the
  group.
\item
  To follow the rules of the group conditional on being in the group.
\end{enumerate}

In a two person group, these will entail an obligation to continue
following the group rules. But not in larger groups.
\end{frame}

\begin{frame}{Puzzle Cases}
\protect\hypertarget{puzzle-cases}{}
\begin{enumerate}
\tightlist
\item
  Large groups
\item
  Immoral group activities
\item
  Explicit disavowal of long term commitment
\end{enumerate}
\end{frame}

\begin{frame}{Large Groups}
\protect\hypertarget{large-groups}{}
\begin{itemize}
\tightlist
\item
  If I'm in a large group (e.g., a protest rally), how much obligation
  do I have to continue from the fact that I've joined?
\item
  Intuitively, not much.
\item
  And maybe the fact that I wouldn't dissolve the group by leaving
  matters here.
\end{itemize}
\end{frame}

\begin{frame}{Immoral Groups}
\protect\hypertarget{immoral-groups}{}
\begin{itemize}
\tightlist
\item
  Gilbert knows this, but it's a challenge for her to say what the
  obligation is in cases where the group is, say, robbing a bank.
\item
  To be fair, this is kind of a problem for everyone.
\item
  If I join a bank robbing group, and promise to do my part in a plan,
  then just bail when the plan is in operation, there is a sense in
  which I've done something wrong.
\item
  This sense can persist even if it would be all things considered worse
  to continue in the group.
\item
  Perhaps Gilbert can offer resources to explain what's going on here.
\end{itemize}
\end{frame}

\begin{frame}{Explicit Disavowal}
\protect\hypertarget{explicit-disavowal}{}
Sometimes a person can join a group and explicitly say they have no long
term commitment to it.

\begin{itemize}
\tightlist
\item
  We're watching football in a common area, and someone we know comes
  by.
\item
  We invite them to join us, and they say ``Sure, but I might have to go
  if I get a call.''
\item
  They are in the group, even though they do nothing at all wrong if the
  call comes and they leave.
\end{itemize}
\end{frame}

\begin{frame}{For Next Time}
\protect\hypertarget{for-next-time}{}
Bratman's very different picture of group action.
\end{frame}

\end{document}
