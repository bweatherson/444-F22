% Options for packages loaded elsewhere
\PassOptionsToPackage{unicode}{hyperref}
\PassOptionsToPackage{hyphens}{url}
%
\documentclass[
  ignorenonframetext,
]{beamer}
\usepackage{pgfpages}
\setbeamertemplate{caption}[numbered]
\setbeamertemplate{caption label separator}{: }
\setbeamercolor{caption name}{fg=normal text.fg}
\beamertemplatenavigationsymbolsempty
% Prevent slide breaks in the middle of a paragraph
\widowpenalties 1 10000
\raggedbottom
\setbeamertemplate{part page}{
  \centering
  \begin{beamercolorbox}[sep=16pt,center]{part title}
    \usebeamerfont{part title}\insertpart\par
  \end{beamercolorbox}
}
\setbeamertemplate{section page}{
  \centering
  \begin{beamercolorbox}[sep=12pt,center]{part title}
    \usebeamerfont{section title}\insertsection\par
  \end{beamercolorbox}
}
\setbeamertemplate{subsection page}{
  \centering
  \begin{beamercolorbox}[sep=8pt,center]{part title}
    \usebeamerfont{subsection title}\insertsubsection\par
  \end{beamercolorbox}
}
\AtBeginPart{
  \frame{\partpage}
}
\AtBeginSection{
  \ifbibliography
  \else
    \frame{\sectionpage}
  \fi
}
\AtBeginSubsection{
  \frame{\subsectionpage}
}
\usepackage{amsmath,amssymb}
\usepackage{lmodern}
\usepackage{ifxetex,ifluatex}
\ifnum 0\ifxetex 1\fi\ifluatex 1\fi=0 % if pdftex
  \usepackage[T1]{fontenc}
  \usepackage[utf8]{inputenc}
  \usepackage{textcomp} % provide euro and other symbols
\else % if luatex or xetex
  \usepackage{unicode-math}
  \defaultfontfeatures{Scale=MatchLowercase}
  \defaultfontfeatures[\rmfamily]{Ligatures=TeX,Scale=1}
  \setmainfont[BoldFont = SF Pro Rounded Semibold]{SF Pro Rounded}
  \setmathfont[]{STIX Two Math}
\fi
\usefonttheme{serif} % use mainfont rather than sansfont for slide text
% Use upquote if available, for straight quotes in verbatim environments
\IfFileExists{upquote.sty}{\usepackage{upquote}}{}
\IfFileExists{microtype.sty}{% use microtype if available
  \usepackage[]{microtype}
  \UseMicrotypeSet[protrusion]{basicmath} % disable protrusion for tt fonts
}{}
\makeatletter
\@ifundefined{KOMAClassName}{% if non-KOMA class
  \IfFileExists{parskip.sty}{%
    \usepackage{parskip}
  }{% else
    \setlength{\parindent}{0pt}
    \setlength{\parskip}{6pt plus 2pt minus 1pt}}
}{% if KOMA class
  \KOMAoptions{parskip=half}}
\makeatother
\usepackage{xcolor}
\IfFileExists{xurl.sty}{\usepackage{xurl}}{} % add URL line breaks if available
\IfFileExists{bookmark.sty}{\usepackage{bookmark}}{\usepackage{hyperref}}
\hypersetup{
  pdftitle={444 Lecture 5.6 - Best Responses},
  pdfauthor={Brian Weatherson},
  hidelinks,
  pdfcreator={LaTeX via pandoc}}
\urlstyle{same} % disable monospaced font for URLs
\newif\ifbibliography
\setlength{\emergencystretch}{3em} % prevent overfull lines
\providecommand{\tightlist}{%
  \setlength{\itemsep}{0pt}\setlength{\parskip}{0pt}}
\setcounter{secnumdepth}{-\maxdimen} % remove section numbering
\let\Tiny=\tiny

 \setbeamertemplate{navigation symbols}{} 

% \usetheme{Madrid}
 \usetheme[numbering=none, progressbar=foot]{metropolis}
 \usecolortheme{wolverine}
 \usepackage{color}
 \usepackage{MnSymbol}
% \usepackage{movie15}

\usepackage{amssymb}% http://ctan.org/pkg/amssymb
\usepackage{pifont}% http://ctan.org/pkg/pifont
\newcommand{\cmark}{\ding{51}}%
\newcommand{\xmark}{\ding{55}}%

\DeclareSymbolFont{symbolsC}{U}{txsyc}{m}{n}
\DeclareMathSymbol{\boxright}{\mathrel}{symbolsC}{128}
\DeclareMathAlphabet{\mathpzc}{OT1}{pzc}{m}{it}

\setlength{\parskip}{1ex plus 0.5ex minus 0.2ex}

\AtBeginSection[]
{
\begin{frame}
	\Huge{\color{darkblue} \insertsection}
\end{frame}
}

\renewenvironment*{quote}	
	{\list{}{\rightmargin   \leftmargin} \item } 	
	{\endlist }

\definecolor{darkgreen}{rgb}{0,0.7,0}
\definecolor{darkblue}{rgb}{0,0,0.8}

\usepackage[italic]{mathastext}
\usepackage{nicefrac}

\setbeamertemplate{caption}{\raggedright\insertcaption}

%\def\toprule{}
%\def\bottomrule{}
%\def\midrule{}
\usepackage{etoolbox}
\AfterEndEnvironment{description}{\vspace{9pt}}
\AfterEndEnvironment{oltableau}{\vspace{9pt}}
\BeforeBeginEnvironment{oltableau}{\vspace{9pt}}
\AfterEndEnvironment{center}{\vspace{9pt}}
\BeforeBeginEnvironment{tabular}{\vspace{9pt}}
\AfterEndEnvironment{longtable}{\vspace{-6pt}}
\usepackage{booktabs}
\usepackage{longtable}
\usepackage{array}
\usepackage{multirow}
\usepackage{wrapfig}
\usepackage{float}
\usepackage{colortbl}
\usepackage{pdflscape}
\usepackage{tabu}
\usepackage{threeparttable} 
\usepackage{threeparttablex} 
\usepackage[normalem]{ulem} 
\usepackage{makecell}
\usepackage{xcolor}
\usepackage{ulem}

\setlength\heavyrulewidth{0ex}
\setlength\lightrulewidth{0.08ex}

\aboverulesep=0ex
\belowrulesep=0ex
\renewcommand{\arraystretch}{1.2}
\ifluatex
  \usepackage{selnolig}  % disable illegal ligatures
\fi

\title{444 Lecture 5.6 - Best Responses}
\author{Brian Weatherson}
\date{}

\begin{document}
\frame{\titlepage}

\begin{frame}{Plan}
\protect\hypertarget{plan}{}
Discuss how to think about best responses work once mixed strategies are
on the table.
\end{frame}

\begin{frame}{Reading}
\protect\hypertarget{reading}{}
Bonanno, Section 6.4.
\end{frame}

\begin{frame}{An example}
\protect\hypertarget{an-example}{}
\begin{table}[!h]
\centering
\begin{tabular}[t]{>{}r|cc}
\toprule
 & Left & Right\\
\midrule
Up & 3, 0 & 0, 0\\
Middle & 2, 0 & 2, 0\\
Down & 0, 0 & 3, 0\\
\bottomrule
\end{tabular}
\end{table}

\begin{itemize}
\tightlist
\item
  Up is the best response to Left.
\item
  Down is the best response to Right.
\item
  Is Middle the best response to anything?
\end{itemize}
\end{frame}

\begin{frame}{Best Responses}
\protect\hypertarget{best-responses}{}
\begin{table}[!h]
\centering
\begin{tabular}[t]{>{}r|cc}
\toprule
 & Left & Right\\
\midrule
Up & 3, 0 & 0, 0\\
Middle & 2, 0 & 2, 0\\
Down & 0, 0 & 3, 0\\
\bottomrule
\end{tabular}
\end{table}

Yes!

\begin{itemize}
\tightlist
\item
  Middle is the best response to the mixed strategy Left with
  probability 0.5, Right with probability 0.5.
\item
  It gets 2, the other options have an expected return of 1.5.
\end{itemize}
\end{frame}

\begin{frame}{Varieties of Mixed Strategies}
\protect\hypertarget{varieties-of-mixed-strategies}{}
\begin{table}[!h]
\centering
\begin{tabular}[t]{>{}r|cc}
\toprule
 & Left & Right\\
\midrule
Up & 3, 0 & 0, 0\\
Middle & 2, 0 & 2, 0\\
Down & 0, 0 & 3, 0\\
\bottomrule
\end{tabular}
\end{table}

\begin{itemize}
\tightlist
\item
  Middle is the best thing to do if you know Column is going to flip a
  coin to decide what to do. \pause
\item
  But it's also the best thing to do if you have no idea what Column is
  going to do, and the best you can do is say it's 50/50 what they are
  going to do.
\item
  So it's actually pretty easy to think of situations where Middle is
  the smart play.
\end{itemize}
\end{frame}

\begin{frame}{Best Response}
\protect\hypertarget{best-response}{}
\begin{itemize}
\tightlist
\item
  A strategy \(S\) is a \textbf{best response} just in case\ldots{}
\item
  There is some probability distribution over the other player's
  strategies and \ldots{}
\item
  No strategy has a higher expected return than \(S\) given that
  probability distribution. \pause
\item
  Note that this allows for ties.
\item
  Weakly dominated strategies can even be best responses in this sense.
  \pause
\item
  This definition also covers mixed strategies; they can also be best
  responses.
\end{itemize}
\end{frame}

\begin{frame}{A Surprising Theorem}
\protect\hypertarget{a-surprising-theorem}{}
\begin{itemize}
\tightlist
\item
  Say a strategy is \textbf{undominated} if no other strategy, pure or
  mixed, strongly dominates it. \pause
\item
  And it is a \textbf{best response} if it does as well as anything,
  given at least one probability distribution.\pause
\item
  Here's the surprising theorem:
\end{itemize}

\begin{quote}
The strategies that are best responses are just the same strategies as
those that are undominated.
\end{quote}
\end{frame}

\begin{frame}{Best Reponses}
\protect\hypertarget{best-reponses}{}
\begin{itemize}
\tightlist
\item
  This relates back to something I was saying in the last lecture.
\item
  The strategies that are dominated by mixtures didn't seem to make
  sense - you could just play the mixtures.
\item
  But here's another property that they have - they are never best
  responses.
\item
  And if they are not best responses, no one can play them while
  maximising expected utility.
\item
  Whatever probability you give to the other player's play, if you
  maximise expected utility you will play a best response.
\item
  And you should maximise expected utility.
\end{itemize}
\end{frame}

\begin{frame}{For Next Time}
\protect\hypertarget{for-next-time}{}
\begin{itemize}
\tightlist
\item
  I'll introduce a new solution concept, one that say playing a best
  response is not just necessary for rationality, it is also sufficient.
\end{itemize}
\end{frame}

\end{document}
