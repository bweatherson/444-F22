% Options for packages loaded elsewhere
\PassOptionsToPackage{unicode}{hyperref}
\PassOptionsToPackage{hyphens}{url}
%
\documentclass[
  ignorenonframetext,
]{beamer}
\usepackage{pgfpages}
\setbeamertemplate{caption}[numbered]
\setbeamertemplate{caption label separator}{: }
\setbeamercolor{caption name}{fg=normal text.fg}
\beamertemplatenavigationsymbolsempty
% Prevent slide breaks in the middle of a paragraph
\widowpenalties 1 10000
\raggedbottom
\setbeamertemplate{part page}{
  \centering
  \begin{beamercolorbox}[sep=16pt,center]{part title}
    \usebeamerfont{part title}\insertpart\par
  \end{beamercolorbox}
}
\setbeamertemplate{section page}{
  \centering
  \begin{beamercolorbox}[sep=12pt,center]{part title}
    \usebeamerfont{section title}\insertsection\par
  \end{beamercolorbox}
}
\setbeamertemplate{subsection page}{
  \centering
  \begin{beamercolorbox}[sep=8pt,center]{part title}
    \usebeamerfont{subsection title}\insertsubsection\par
  \end{beamercolorbox}
}
\AtBeginPart{
  \frame{\partpage}
}
\AtBeginSection{
  \ifbibliography
  \else
    \frame{\sectionpage}
  \fi
}
\AtBeginSubsection{
  \frame{\subsectionpage}
}
\usepackage{amsmath,amssymb}
\usepackage{lmodern}
\usepackage{ifxetex,ifluatex}
\ifnum 0\ifxetex 1\fi\ifluatex 1\fi=0 % if pdftex
  \usepackage[T1]{fontenc}
  \usepackage[utf8]{inputenc}
  \usepackage{textcomp} % provide euro and other symbols
\else % if luatex or xetex
  \usepackage{unicode-math}
  \defaultfontfeatures{Scale=MatchLowercase}
  \defaultfontfeatures[\rmfamily]{Ligatures=TeX,Scale=1}
  \setmainfont[BoldFont = SF Pro Rounded Semibold]{SF Pro Rounded}
  \setmathfont[]{STIX Two Math}
\fi
\usefonttheme{serif} % use mainfont rather than sansfont for slide text
% Use upquote if available, for straight quotes in verbatim environments
\IfFileExists{upquote.sty}{\usepackage{upquote}}{}
\IfFileExists{microtype.sty}{% use microtype if available
  \usepackage[]{microtype}
  \UseMicrotypeSet[protrusion]{basicmath} % disable protrusion for tt fonts
}{}
\makeatletter
\@ifundefined{KOMAClassName}{% if non-KOMA class
  \IfFileExists{parskip.sty}{%
    \usepackage{parskip}
  }{% else
    \setlength{\parindent}{0pt}
    \setlength{\parskip}{6pt plus 2pt minus 1pt}}
}{% if KOMA class
  \KOMAoptions{parskip=half}}
\makeatother
\usepackage{xcolor}
\IfFileExists{xurl.sty}{\usepackage{xurl}}{} % add URL line breaks if available
\IfFileExists{bookmark.sty}{\usepackage{bookmark}}{\usepackage{hyperref}}
\hypersetup{
  pdftitle={444 Lecture 10.1 - O'Connor Chapter 4},
  pdfauthor={Brian Weatherson},
  hidelinks,
  pdfcreator={LaTeX via pandoc}}
\urlstyle{same} % disable monospaced font for URLs
\newif\ifbibliography
\setlength{\emergencystretch}{3em} % prevent overfull lines
\providecommand{\tightlist}{%
  \setlength{\itemsep}{0pt}\setlength{\parskip}{0pt}}
\setcounter{secnumdepth}{-\maxdimen} % remove section numbering
\let\Tiny=\tiny

 \setbeamertemplate{navigation symbols}{} 

% \usetheme{Madrid}
 \usetheme[numbering=none, progressbar=foot]{metropolis}
 \usecolortheme{wolverine}
 \usepackage{color}
 \usepackage{MnSymbol}
% \usepackage{movie15}

\usepackage{amssymb}% http://ctan.org/pkg/amssymb
\usepackage{pifont}% http://ctan.org/pkg/pifont
\newcommand{\cmark}{\ding{51}}%
\newcommand{\xmark}{\ding{55}}%

\DeclareSymbolFont{symbolsC}{U}{txsyc}{m}{n}
\DeclareMathSymbol{\boxright}{\mathrel}{symbolsC}{128}
\DeclareMathAlphabet{\mathpzc}{OT1}{pzc}{m}{it}

\setlength{\parskip}{1ex plus 0.5ex minus 0.2ex}

\AtBeginSection[]
{
\begin{frame}
	\Huge{\color{darkblue} \insertsection}
\end{frame}
}

\renewenvironment*{quote}	
	{\list{}{\rightmargin   \leftmargin} \item } 	
	{\endlist }

\definecolor{darkgreen}{rgb}{0,0.7,0}
\definecolor{darkblue}{rgb}{0,0,0.8}

\usepackage[italic]{mathastext}
\usepackage{nicefrac}
\usepackage{istgame}

\setbeamertemplate{caption}{\raggedright\insertcaption}

%\def\toprule{}
%\def\bottomrule{}
%\def\midrule{}
\usepackage{etoolbox}
\AfterEndEnvironment{description}{\vspace{9pt}}
\AfterEndEnvironment{oltableau}{\vspace{9pt}}
\BeforeBeginEnvironment{oltableau}{\vspace{9pt}}
\AfterEndEnvironment{center}{\vspace{9pt}}
\BeforeBeginEnvironment{tabular}{\vspace{9pt}}
\AfterEndEnvironment{longtable}{\vspace{-6pt}}
\usepackage{booktabs}
\usepackage{longtable}
\usepackage{array}
\usepackage{multirow}
\usepackage{wrapfig}
\usepackage{float}
\usepackage{colortbl}
\usepackage{pdflscape}
\usepackage{tabu}
\usepackage{threeparttable} 
\usepackage{threeparttablex} 
\usepackage[normalem]{ulem} 
\usepackage{makecell}
\usepackage{xcolor}
\usepackage{ulem}

\setlength\heavyrulewidth{0ex}
\setlength\lightrulewidth{0.08ex}

\aboverulesep=0ex
\belowrulesep=0ex
\renewcommand{\arraystretch}{1.2}
\AtBeginSection[]
{
    \begin{frame}
        \frametitle{Day Plan}
        \tableofcontents[currentsection]
    \end{frame}
}
\ifluatex
  \usepackage{selnolig}  % disable illegal ligatures
\fi

\title{444 Lecture 10.1 - O'Connor Chapter 4}
\author{Brian Weatherson}
\date{}

\begin{document}
\frame{\titlepage}

\begin{frame}{Day Plan}
\protect\hypertarget{day-plan}{}
\tableofcontents
\end{frame}

\hypertarget{conventionality}{%
\section{Conventionality}\label{conventionality}}

\begin{frame}{Degrees of Conventionality}
\protect\hypertarget{degrees-of-conventionality}{}
This is a really good point - we shouldn't think of social outcomes as
being either purely conventional, or explained by the superiority of the
ultimate outcome.

\begin{itemize}
\tightlist
\item
  There are in between cases.
\item
  And there are cases that are closer to one end than the other.
\item
  We need a scale.
\end{itemize}
\end{frame}

\begin{frame}{Language}
\protect\hypertarget{language}{}
To see this, think about how we ended up with the distribution of
languages we have today.

\begin{itemize}
\tightlist
\item
  Why are English, Spanish and Mandarin Chinese spoken so widely, and
  other languages less so?
\end{itemize}
\end{frame}

\begin{frame}{Non-Conventional Stories}
\protect\hypertarget{non-conventional-stories}{}
\begin{itemize}
\tightlist
\item
  Some languages are easier to learn than others.
\item
  They don't have large dictionaries (though English does);
\item
  They don't make you learn messy things like gender or formality
  (though Spanish does);
\item
  They don't make you produce sounds that are hard to acquire in
  adulthood (I'm not sure if any of these three do)
\item
  These are all reasons why it would be surprising for Welsh or Xhosa to
  become widespread.
\end{itemize}
\end{frame}

\begin{frame}{Conventional Stories}
\protect\hypertarget{conventional-stories}{}
\begin{itemize}
\tightlist
\item
  But none of these factors explain why English is more widely spoken
  than, say, Dutch.
\item
  The reason for that seems more connected to how New Amsterdam became
  New York than to anything about language.
\item
  It happened that various English-speaking settlements in the Americas
  thrived, Dutch-speaking ones did not, and here we are.
\end{itemize}
\end{frame}

\begin{frame}{Measuring Conventionality}
\protect\hypertarget{measuring-conventionality}{}
\begin{itemize}
\tightlist
\item
  O'Connor offers a measure of how conventional an outcome is.
\item
  I don't love it as a measure; as she notes, it varies depending on how
  finely we divide up the alternative options.
\item
  Better I think to do pairwise comparisons directly.
\item
  English is more widely spoken than Welsh for somewhat functional
  reasons.
\item
  English is more widely spoken than Dutch for almost entirely
  conventional reasons.
\end{itemize}
\end{frame}

\hypertarget{pairing}{%
\section{Pairing}\label{pairing}}

\begin{frame}{Normal Households}
\protect\hypertarget{normal-households}{}
Here's a quote from page 101.

\begin{quote}
Especially in recent Western history, the vast majority of households
involved one man and one woman.
\end{quote}

And I don't really know if this is true, especially in the time frame
needed to make the explanation work.
\end{frame}

\begin{frame}{Post War America}
\protect\hypertarget{post-war-america}{}
In post-war America, you see the following three factors.

\begin{enumerate}
\tightlist
\item
  Few households with multiple adult generations.
\item
  High marriage rates and low divorce rates.
\item
  Low rates of mid-life death.
\end{enumerate}

You really need all three to get ``the vast majority of households
involved one man and one woman'', and I don't really know how many
times/places have all three.
\end{frame}

\begin{frame}{Household Structure}
\protect\hypertarget{household-structure}{}
So open question.

\begin{itemize}
\tightlist
\item
  In what societies (other than mid-20th-century
  America/Canada/Australia) was it true that the vast majority of
  households involved one man and one woman?
\end{itemize}

My guess is not that many. And since gender divisions are
\emph{everywhere}, this can't really be the story.
\end{frame}

\hypertarget{mfeo-games}{%
\section{MFEO Games}\label{mfeo-games}}

\begin{frame}{How Common are MFEO Games?}
\protect\hypertarget{how-common-are-mfeo-games}{}
\begin{itemize}
\tightlist
\item
  We need specialisation to make modern society run.
\item
  But question: how many real world situations are there where we need
  1/2 the (adult) population to do X, and the other half to do Y?
\item
  A lot of specialisation tasks are more than we need some single digit
  percentage of the population to do them.
\item
  Why don't we get more types correlating with those needed
  specialisations.
\end{itemize}
\end{frame}

\begin{frame}{Households and MFEO}
\protect\hypertarget{households-and-mfeo}{}
\begin{itemize}
\tightlist
\item
  This is really a version of the previous question.
\item
  If it is really common to have 1M/1F households, then sure there will
  be lots of MFEO games where it is useful to have 50/50 splits.
\item
  But without that, I don't really see them being frequent enough for
  this analysis.
\end{itemize}
\end{frame}

\begin{frame}{Child Making}
\protect\hypertarget{child-making}{}
Now there's one thing that does require 1M/1F, namely child making.

\begin{itemize}
\tightlist
\item
  If you have a society where the two biological parents of a child have
  a distinctive role in child rearing, then sure, that will lead to
  these two player games.
\item
  But this leads us back to questions of household composition, and I'm
  not sure there is a good answer.
\end{itemize}
\end{frame}

\begin{frame}{Functionality Given Pregnancy}
\protect\hypertarget{functionality-given-pregnancy}{}
Can you get more out of thinking about reproduction?

\begin{itemize}
\tightlist
\item
  Some jobs are hard to do while pregnant or nursing.
\item
  Now `hard' doesn't mean impossible, but maybe there is a story here.
\item
  But do you need 1/2 the population to engage in those jobs?
\item
  Some of this might turn on hard questions about the prevalence of big
  game hunting in human societies.
\end{itemize}
\end{frame}

\hypertarget{multiple-games}{%
\section{Multiple Games}\label{multiple-games}}

\begin{frame}{Hawk-Dove and MFEO}
\protect\hypertarget{hawk-dove-and-mfeo}{}
\begin{itemize}
\tightlist
\item
  This I thought was one of the best parts of the chapter.
\item
  It seems really crucial to understanding what's going on.
\item
  The point of gender roles is not that there is this one thing where
  men do one thing and women do another.
\item
  It's that there is a systematic pattern of differential behavior
  across a huge range of parts of life.
\item
  Thinking about how games interact could be a big part of the story.
\end{itemize}
\end{frame}

\begin{frame}{To Be Explained}
\protect\hypertarget{to-be-explained}{}
One challenge in this chapter is to go beyond explaining why typing
occurs to explaining why gender is universal.

\begin{itemize}
\tightlist
\item
  There is an attempt at this by noting the benefits of the equal
  division.
\item
  But this is, I think, an artifact of the game chosen.
\item
  It comes back to whether it is useful to have 50\% specialise in some
  task.
\item
  I think it's more likely that thinking about interlocking games is
  more promising.
\end{itemize}
\end{frame}

\end{document}
