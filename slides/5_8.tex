% Options for packages loaded elsewhere
\PassOptionsToPackage{unicode}{hyperref}
\PassOptionsToPackage{hyphens}{url}
%
\documentclass[
  ignorenonframetext,
]{beamer}
\usepackage{pgfpages}
\setbeamertemplate{caption}[numbered]
\setbeamertemplate{caption label separator}{: }
\setbeamercolor{caption name}{fg=normal text.fg}
\beamertemplatenavigationsymbolsempty
% Prevent slide breaks in the middle of a paragraph
\widowpenalties 1 10000
\raggedbottom
\setbeamertemplate{part page}{
  \centering
  \begin{beamercolorbox}[sep=16pt,center]{part title}
    \usebeamerfont{part title}\insertpart\par
  \end{beamercolorbox}
}
\setbeamertemplate{section page}{
  \centering
  \begin{beamercolorbox}[sep=12pt,center]{part title}
    \usebeamerfont{section title}\insertsection\par
  \end{beamercolorbox}
}
\setbeamertemplate{subsection page}{
  \centering
  \begin{beamercolorbox}[sep=8pt,center]{part title}
    \usebeamerfont{subsection title}\insertsubsection\par
  \end{beamercolorbox}
}
\AtBeginPart{
  \frame{\partpage}
}
\AtBeginSection{
  \ifbibliography
  \else
    \frame{\sectionpage}
  \fi
}
\AtBeginSubsection{
  \frame{\subsectionpage}
}
\usepackage{amsmath,amssymb}
\usepackage{lmodern}
\usepackage{ifxetex,ifluatex}
\ifnum 0\ifxetex 1\fi\ifluatex 1\fi=0 % if pdftex
  \usepackage[T1]{fontenc}
  \usepackage[utf8]{inputenc}
  \usepackage{textcomp} % provide euro and other symbols
\else % if luatex or xetex
  \usepackage{unicode-math}
  \defaultfontfeatures{Scale=MatchLowercase}
  \defaultfontfeatures[\rmfamily]{Ligatures=TeX,Scale=1}
  \setmainfont[BoldFont = SF Pro Rounded Semibold]{SF Pro Rounded}
  \setmathfont[]{STIX Two Math}
\fi
\usefonttheme{serif} % use mainfont rather than sansfont for slide text
% Use upquote if available, for straight quotes in verbatim environments
\IfFileExists{upquote.sty}{\usepackage{upquote}}{}
\IfFileExists{microtype.sty}{% use microtype if available
  \usepackage[]{microtype}
  \UseMicrotypeSet[protrusion]{basicmath} % disable protrusion for tt fonts
}{}
\makeatletter
\@ifundefined{KOMAClassName}{% if non-KOMA class
  \IfFileExists{parskip.sty}{%
    \usepackage{parskip}
  }{% else
    \setlength{\parindent}{0pt}
    \setlength{\parskip}{6pt plus 2pt minus 1pt}}
}{% if KOMA class
  \KOMAoptions{parskip=half}}
\makeatother
\usepackage{xcolor}
\IfFileExists{xurl.sty}{\usepackage{xurl}}{} % add URL line breaks if available
\IfFileExists{bookmark.sty}{\usepackage{bookmark}}{\usepackage{hyperref}}
\hypersetup{
  pdftitle={444 Lecture 5.8 - Why Nash?},
  pdfauthor={Brian Weatherson},
  hidelinks,
  pdfcreator={LaTeX via pandoc}}
\urlstyle{same} % disable monospaced font for URLs
\newif\ifbibliography
\setlength{\emergencystretch}{3em} % prevent overfull lines
\providecommand{\tightlist}{%
  \setlength{\itemsep}{0pt}\setlength{\parskip}{0pt}}
\setcounter{secnumdepth}{-\maxdimen} % remove section numbering
\let\Tiny=\tiny

 \setbeamertemplate{navigation symbols}{} 

% \usetheme{Madrid}
 \usetheme[numbering=none, progressbar=foot]{metropolis}
 \usecolortheme{wolverine}
 \usepackage{color}
 \usepackage{MnSymbol}
% \usepackage{movie15}

\usepackage{amssymb}% http://ctan.org/pkg/amssymb
\usepackage{pifont}% http://ctan.org/pkg/pifont
\newcommand{\cmark}{\ding{51}}%
\newcommand{\xmark}{\ding{55}}%

\DeclareSymbolFont{symbolsC}{U}{txsyc}{m}{n}
\DeclareMathSymbol{\boxright}{\mathrel}{symbolsC}{128}
\DeclareMathAlphabet{\mathpzc}{OT1}{pzc}{m}{it}

\setlength{\parskip}{1ex plus 0.5ex minus 0.2ex}

\AtBeginSection[]
{
\begin{frame}
	\Huge{\color{darkblue} \insertsection}
\end{frame}
}

\renewenvironment*{quote}	
	{\list{}{\rightmargin   \leftmargin} \item } 	
	{\endlist }

\definecolor{darkgreen}{rgb}{0,0.7,0}
\definecolor{darkblue}{rgb}{0,0,0.8}

\usepackage[italic]{mathastext}
\usepackage{nicefrac}

\setbeamertemplate{caption}{\raggedright\insertcaption}

%\def\toprule{}
%\def\bottomrule{}
%\def\midrule{}
\usepackage{etoolbox}
\AfterEndEnvironment{description}{\vspace{9pt}}
\AfterEndEnvironment{oltableau}{\vspace{9pt}}
\BeforeBeginEnvironment{oltableau}{\vspace{9pt}}
\AfterEndEnvironment{center}{\vspace{9pt}}
\BeforeBeginEnvironment{tabular}{\vspace{9pt}}
\AfterEndEnvironment{longtable}{\vspace{-6pt}}
\usepackage{booktabs}
\usepackage{longtable}
\usepackage{array}
\usepackage{multirow}
\usepackage{wrapfig}
\usepackage{float}
\usepackage{colortbl}
\usepackage{pdflscape}
\usepackage{tabu}
\usepackage{threeparttable} 
\usepackage{threeparttablex} 
\usepackage[normalem]{ulem} 
\usepackage{makecell}
\usepackage{xcolor}
\usepackage{ulem}

\setlength\heavyrulewidth{0ex}
\setlength\lightrulewidth{0.08ex}

\aboverulesep=0ex
\belowrulesep=0ex
\renewcommand{\arraystretch}{1.2}
\ifluatex
  \usepackage{selnolig}  % disable illegal ligatures
\fi

\title{444 Lecture 5.8 - Why Nash?}
\author{Brian Weatherson}
\date{}

\begin{document}
\frame{\titlepage}

\begin{frame}{Plan}
\protect\hypertarget{plan}{}
To look at the philosophical significance of Nash Equilibria.
\end{frame}

\begin{frame}{Reading}
\protect\hypertarget{reading}{}
Bonanno, sections 2.6 (which we discussed earlier) and 6.4.
\end{frame}

\begin{frame}{Two Conjectures}
\protect\hypertarget{two-conjectures}{}
\begin{enumerate}
\tightlist
\item
  It is rational to play any rationalizable strategy.
\item
  It is only rational to play Nash Equilibrium strategies
\end{enumerate}

I'm going to end this week talking a bit about why people might prefer 2
over 1.
\end{frame}

\begin{frame}{One Intuitive Idea}
\protect\hypertarget{one-intuitive-idea}{}
\begin{itemize}
\tightlist
\item
  Don't just play Rock - the other person will figure it out.
\item
  Rock every time is rationalizable.
\item
  But you shouldn't do it.
\item
  Therefore principle 1 must be false.
\end{itemize}
\end{frame}

\begin{frame}{Response 1}
\protect\hypertarget{response-1}{}
\begin{itemize}
\tightlist
\item
  Yeah, you shouldn't play Rock every single time, that's dumb.
\item
  But on any given occasion, it's fine.
\item
  And we know, from e.g., Prisoners' Dilemma, that we shouldn't infer
  what to do in a single shot game from what happens in the repeated
  game.
\end{itemize}
\end{frame}

\begin{frame}{Response 2}
\protect\hypertarget{response-2}{}
\begin{itemize}
\tightlist
\item
  The orthodox solution (i.e., principle 2) actually doesn't give you
  the result you might want here.
\item
  It is possible that the randomising device will come up Rock every
  single time.
\item
  So if you think it's always irrational to play Rock repeatedly, you
  have to think both of these are wrong.
\end{itemize}
\end{frame}

\begin{frame}{Response 3}
\protect\hypertarget{response-3}{}
\begin{itemize}
\tightlist
\item
  If principle 2 is right, all rational players will randomise every
  time.
\item
  So the expected return of Rock is just the same as the expected return
  of randomisation.
\item
  So it can't be wrong to play it.
\end{itemize}
\end{frame}

\begin{frame}{Mixing Response 2 and Response 3}
\protect\hypertarget{mixing-response-2-and-response-3}{}
\begin{itemize}
\tightlist
\item
  If principle 2 is right, and it's common knowledge that the players
  are rational, then the rational way to interpret the other player
  playing Rock every time is ``Wow, their random device is having a
  freaky run.''
\item
  But if that's right, there isn't anything wrong with playing Rock
  every time.
\end{itemize}
\end{frame}

\begin{frame}{Other Direction}
\protect\hypertarget{other-direction}{}
\begin{itemize}
\tightlist
\item
  As we'll see when we get to O'Connor's book, you mostly see people
  wanting more restrictions on moves than Nash.
\item
  But Bonanno ends chapter 6 with an interesting reason for thinking
  even rationalisability (i.e., IDSDS) is too strong. \pause
\item
  It's really incredibly unrealistic to know the utility function that
  the other player has.
\item
  You might know the physical outcomes of the game, but knowing what
  utility each player gets is a huge assumption.
\item
  So in practice, you should probably not rely too heavily on theories
  or policies that rely on this knowledge.
\end{itemize}
\end{frame}

\begin{frame}{For Next Time}
\protect\hypertarget{for-next-time}{}
\begin{itemize}
\tightlist
\item
  Next week we will do a bit of revision of probability theory.
\item
  It's completely optional; if you want a week off, take a week off.
\item
  After that, we'll look at how game theorists think about signals and
  messages.
\end{itemize}
\end{frame}

\end{document}
