% Options for packages loaded elsewhere
\PassOptionsToPackage{unicode}{hyperref}
\PassOptionsToPackage{hyphens}{url}
%
\documentclass[
  ignorenonframetext,
]{beamer}
\usepackage{pgfpages}
\setbeamertemplate{caption}[numbered]
\setbeamertemplate{caption label separator}{: }
\setbeamercolor{caption name}{fg=normal text.fg}
\beamertemplatenavigationsymbolsempty
% Prevent slide breaks in the middle of a paragraph
\widowpenalties 1 10000
\raggedbottom
\setbeamertemplate{part page}{
  \centering
  \begin{beamercolorbox}[sep=16pt,center]{part title}
    \usebeamerfont{part title}\insertpart\par
  \end{beamercolorbox}
}
\setbeamertemplate{section page}{
  \centering
  \begin{beamercolorbox}[sep=12pt,center]{part title}
    \usebeamerfont{section title}\insertsection\par
  \end{beamercolorbox}
}
\setbeamertemplate{subsection page}{
  \centering
  \begin{beamercolorbox}[sep=8pt,center]{part title}
    \usebeamerfont{subsection title}\insertsubsection\par
  \end{beamercolorbox}
}
\AtBeginPart{
  \frame{\partpage}
}
\AtBeginSection{
  \ifbibliography
  \else
    \frame{\sectionpage}
  \fi
}
\AtBeginSubsection{
  \frame{\subsectionpage}
}
\usepackage{amsmath,amssymb}
\usepackage{lmodern}
\usepackage{ifxetex,ifluatex}
\ifnum 0\ifxetex 1\fi\ifluatex 1\fi=0 % if pdftex
  \usepackage[T1]{fontenc}
  \usepackage[utf8]{inputenc}
  \usepackage{textcomp} % provide euro and other symbols
\else % if luatex or xetex
  \usepackage{unicode-math}
  \defaultfontfeatures{Scale=MatchLowercase}
  \defaultfontfeatures[\rmfamily]{Ligatures=TeX,Scale=1}
  \setmainfont[BoldFont = SF Pro Rounded Semibold]{SF Pro Rounded}
  \setmathfont[]{STIX Two Math}
\fi
\usefonttheme{serif} % use mainfont rather than sansfont for slide text
% Use upquote if available, for straight quotes in verbatim environments
\IfFileExists{upquote.sty}{\usepackage{upquote}}{}
\IfFileExists{microtype.sty}{% use microtype if available
  \usepackage[]{microtype}
  \UseMicrotypeSet[protrusion]{basicmath} % disable protrusion for tt fonts
}{}
\makeatletter
\@ifundefined{KOMAClassName}{% if non-KOMA class
  \IfFileExists{parskip.sty}{%
    \usepackage{parskip}
  }{% else
    \setlength{\parindent}{0pt}
    \setlength{\parskip}{6pt plus 2pt minus 1pt}}
}{% if KOMA class
  \KOMAoptions{parskip=half}}
\makeatother
\usepackage{xcolor}
\IfFileExists{xurl.sty}{\usepackage{xurl}}{} % add URL line breaks if available
\IfFileExists{bookmark.sty}{\usepackage{bookmark}}{\usepackage{hyperref}}
\hypersetup{
  pdftitle={444 Lecture 8.2 - Axelrod Tournament},
  pdfauthor={Brian Weatherson},
  hidelinks,
  pdfcreator={LaTeX via pandoc}}
\urlstyle{same} % disable monospaced font for URLs
\newif\ifbibliography
\setlength{\emergencystretch}{3em} % prevent overfull lines
\providecommand{\tightlist}{%
  \setlength{\itemsep}{0pt}\setlength{\parskip}{0pt}}
\setcounter{secnumdepth}{-\maxdimen} % remove section numbering
\let\Tiny=\tiny

 \setbeamertemplate{navigation symbols}{} 

% \usetheme{Madrid}
 \usetheme[numbering=none, progressbar=foot]{metropolis}
 \usecolortheme{wolverine}
 \usepackage{color}
 \usepackage{MnSymbol}
% \usepackage{movie15}

\usepackage{amssymb}% http://ctan.org/pkg/amssymb
\usepackage{pifont}% http://ctan.org/pkg/pifont
\newcommand{\cmark}{\ding{51}}%
\newcommand{\xmark}{\ding{55}}%

\DeclareSymbolFont{symbolsC}{U}{txsyc}{m}{n}
\DeclareMathSymbol{\boxright}{\mathrel}{symbolsC}{128}
\DeclareMathAlphabet{\mathpzc}{OT1}{pzc}{m}{it}

\setlength{\parskip}{1ex plus 0.5ex minus 0.2ex}

\AtBeginSection[]
{
\begin{frame}
	\Huge{\color{darkblue} \insertsection}
\end{frame}
}

\renewenvironment*{quote}	
	{\list{}{\rightmargin   \leftmargin} \item } 	
	{\endlist }

\definecolor{darkgreen}{rgb}{0,0.7,0}
\definecolor{darkblue}{rgb}{0,0,0.8}

\usepackage[italic]{mathastext}
\usepackage{nicefrac}
\usepackage{istgame}

\setbeamertemplate{caption}{\raggedright\insertcaption}

%\def\toprule{}
%\def\bottomrule{}
%\def\midrule{}
\usepackage{etoolbox}
\AfterEndEnvironment{description}{\vspace{9pt}}
\AfterEndEnvironment{oltableau}{\vspace{9pt}}
\BeforeBeginEnvironment{oltableau}{\vspace{9pt}}
\AfterEndEnvironment{center}{\vspace{9pt}}
\BeforeBeginEnvironment{tabular}{\vspace{9pt}}
\AfterEndEnvironment{longtable}{\vspace{-6pt}}
\usepackage{booktabs}
\usepackage{longtable}
\usepackage{array}
\usepackage{multirow}
\usepackage{wrapfig}
\usepackage{float}
\usepackage{colortbl}
\usepackage{pdflscape}
\usepackage{tabu}
\usepackage{threeparttable} 
\usepackage{threeparttablex} 
\usepackage[normalem]{ulem} 
\usepackage{makecell}
\usepackage{xcolor}
\usepackage{ulem}

\setlength\heavyrulewidth{0ex}
\setlength\lightrulewidth{0.08ex}

\aboverulesep=0ex
\belowrulesep=0ex
\renewcommand{\arraystretch}{1.2}
\ifluatex
  \usepackage{selnolig}  % disable illegal ligatures
\fi

\title{444 Lecture 8.2 - Axelrod Tournament}
\author{Brian Weatherson}
\date{}

\begin{document}
\frame{\titlepage}

\begin{frame}{Overview}
\protect\hypertarget{overview}{}
This lecture covers some of the lessons from the Iterated Prisoners'
Dilemma tournaments that Michigan professor Robert Axelrod ran in the
early 1980s.
\end{frame}

\begin{frame}{Four Papers}
\protect\hypertarget{four-papers}{}
\begin{itemize}
\tightlist
\item
  \href{https://www.jstor.org/stable/173932}{Effective Choice in the
  Prisoner's Dilemma}, \emph{Journal of Conflict Resolution} 24 (1980):
  3-25.
\item
  \href{https://www.jstor.org/stable/173638}{More Effective Choice in
  the Prisoner's Dilemma}, \emph{Journal of Conflict Resolution} 24
  (1980): 379-403.
\item
  \href{https://www.jstor.org/stable/1961366}{The Emergence of
  Cooperation among Egoists}, \emph{The American Political Science
  Review} 75 (1981): 306-318.
\item
  \href{https://www.jstor.org/stable/1685895}{The Evolution of
  Cooperation} with William Hamilton, \emph{Science} 211 (1981):
  1390-1396.
\end{itemize}
\end{frame}

\begin{frame}{The First Tournament}
\protect\hypertarget{the-first-tournament}{}
\begin{itemize}
\tightlist
\item
  Axelrod advertised the first round of his tournament, and called for
  submissions.
\item
  This was far from trivial in pre-internet days, and he only got 13
  submissions.
\item
  In the first tournament he said that \(k\) would be 100, but no one
  actually exploited that fact.
\end{itemize}
\end{frame}

\begin{frame}{The Winner}
\protect\hypertarget{the-winner}{}
Tit-for-Tat
\end{frame}

\begin{frame}{Tit-for-Tat}
\protect\hypertarget{tit-for-tat}{}
Two rules.

\begin{enumerate}
\tightlist
\item
  Play C at round 1.
\item
  In all subsequent rounds, do whatever the other player just did.
\end{enumerate}
\end{frame}

\begin{frame}{The Second Tournament}
\protect\hypertarget{the-second-tournament}{}
\begin{itemize}
\tightlist
\item
  So Axelrod wrote this up, including saying who won.
\item
  He called for more submissions, and now got 66.
\item
  Some of these were typed, some came to Ann Arbor on the huge magnetic
  disks that were used way back then.
\item
  He ran the tournament again, this time with a random number of rounds.
  \pause
\item
  And Tit-for-Tat won again.
\end{itemize}
\end{frame}

\begin{frame}{Logic and Victory}
\protect\hypertarget{logic-and-victory}{}
\begin{itemize}
\tightlist
\item
  This doesn't mean Tit-for-Tat is the best strategy.
\item
  Indeed, in each tournament it was easy in retrospect to describe
  strategies that would have beaten everyone, including TFT, if they had
  been entered.
\item
  But still, it's pretty impressive.
\end{itemize}
\end{frame}

\begin{frame}{Four Features}
\protect\hypertarget{four-features}{}
Tit-for-Tat has five striking characteristics, each of which was
positively correlated with success in the tournaments.

\begin{itemize}
\tightlist
\item
  Nice
\item
  Provocable
\item
  Forgiving
\item
  Not envious
\item
  Simple
\end{itemize}
\end{frame}

\begin{frame}{Nice}
\protect\hypertarget{nice}{}
The clearest distinction in the tournament was between strategies that
were Nice and those that were Nasty.

\begin{itemize}
\tightlist
\item
  By definition, a strategy is Nice iff it is never the first to defect.
\item
  You don't have to be very nice in the intuitive sense to count as
  Nice.
\end{itemize}
\end{frame}

\begin{frame}{Grim Trigger}
\protect\hypertarget{grim-trigger}{}
Here is one nice strategy, one Axelrod calls Grim Trigger.

\begin{enumerate}
\tightlist
\item
  Cooperate on move 1.
\item
  If the other player ever defects, defect on every subsequent move.
\end{enumerate}

This strategy did really badly; it was the worst Nice strategy in round
2. But still many Nasty strategies did worse.
\end{frame}

\begin{frame}{Nice Strategies}
\protect\hypertarget{nice-strategies}{}
\begin{itemize}
\tightlist
\item
  In the evolutionary versions of the game, there can be a tendency for
  strategies to tend towards being Nice.
\item
  Then evolution stops, because when two Nice strategies meet, the
  payout is inevitably 3k to each.
\item
  Although the best strategies are all Nice, it is how they interact
  with Nasty strategies that determines who wins.
\end{itemize}
\end{frame}

\begin{frame}{Provocable}
\protect\hypertarget{provocable}{}
\begin{itemize}
\tightlist
\item
  It's bad to get pushed around.
\item
  Nasty strategies are always looking for how much they can get away
  with.
\item
  So you want to send a clear message that defections will not be
  tolerated.
\item
  Obviously TFT does that.
\end{itemize}
\end{frame}

\begin{frame}{Forgiving}
\protect\hypertarget{forgiving}{}
\begin{itemize}
\tightlist
\item
  But you don't want to be Grim Trigger.
\item
  It's bad to be pushed around, but it's not much better to end up in
  all defect land.
\item
  You need a way back to all cooperate land.
\item
  TFT has that, though notably it isn't perfect at this.
\item
  TFT can get into CD-DC-CD-etc cycles with a bunch of strategies.
\end{itemize}
\end{frame}

\begin{frame}{Not Envious}
\protect\hypertarget{not-envious}{}
\begin{itemize}
\tightlist
\item
  In any interaction, TFT never does better than who it is playing with.
\item
  Yet it comes out first overall.
\item
  This is kind of amazing.
\item
  It just does not care at all about winning against who it is facing
  off with.
\end{itemize}
\end{frame}

\begin{frame}{Not Envious To a Fault}
\protect\hypertarget{not-envious-to-a-fault}{}
\begin{itemize}
\tightlist
\item
  Note that TFT doesn't always do that well in evolutionary games.
\item
  This is because it might take this a bit too far.
\item
  It doesn't look to exploit weaknesses in opponents.
\end{itemize}
\end{frame}

\begin{frame}{Simple}
\protect\hypertarget{simple}{}
\begin{itemize}
\tightlist
\item
  Other strategies try to figure out what their rivals are doing.
\item
  They normally get this wrong. \pause
\item
  Or they try and send complex signals.
\item
  These are usually misinterpreted. \pause
\item
  TFT keeps things simple, and doesn't lose points messing around
  looking for any edges.
\end{itemize}
\end{frame}

\begin{frame}{Variant Games}
\protect\hypertarget{variant-games}{}
\begin{itemize}
\tightlist
\item
  The most interesting variant to me is the one where a strategy only
  gets implemented with probability 0.99 on each move.
\item
  Sometimes there are performance errors.
\item
  TFT does terribly in this; it can't get out of randomly generated
  defection cycles.
\item
  In this kind of game you need to be a bit more forgiving.
\item
  But also you can try to get away with a bit more; if the other person
  will treat a defection as random, you can plan a few.
\end{itemize}
\end{frame}

\begin{frame}{Rest of Day}
\protect\hypertarget{rest-of-day}{}
\begin{itemize}
\tightlist
\item
  I'm not going to do slides/recordings about Oyun.
\item
  If you have questions about it (and it isn't obvious), come along to
  class on Monday and we'll talk through how it works.
\end{itemize}
\end{frame}

\end{document}
