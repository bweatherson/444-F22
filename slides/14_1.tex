% Options for packages loaded elsewhere
\PassOptionsToPackage{unicode}{hyperref}
\PassOptionsToPackage{hyphens}{url}
%
\documentclass[
  ignorenonframetext,
]{beamer}
\usepackage{pgfpages}
\setbeamertemplate{caption}[numbered]
\setbeamertemplate{caption label separator}{: }
\setbeamercolor{caption name}{fg=normal text.fg}
\beamertemplatenavigationsymbolsempty
% Prevent slide breaks in the middle of a paragraph
\widowpenalties 1 10000
\raggedbottom
\setbeamertemplate{part page}{
  \centering
  \begin{beamercolorbox}[sep=16pt,center]{part title}
    \usebeamerfont{part title}\insertpart\par
  \end{beamercolorbox}
}
\setbeamertemplate{section page}{
  \centering
  \begin{beamercolorbox}[sep=12pt,center]{part title}
    \usebeamerfont{section title}\insertsection\par
  \end{beamercolorbox}
}
\setbeamertemplate{subsection page}{
  \centering
  \begin{beamercolorbox}[sep=8pt,center]{part title}
    \usebeamerfont{subsection title}\insertsubsection\par
  \end{beamercolorbox}
}
\AtBeginPart{
  \frame{\partpage}
}
\AtBeginSection{
  \ifbibliography
  \else
    \frame{\sectionpage}
  \fi
}
\AtBeginSubsection{
  \frame{\subsectionpage}
}
\usepackage{amsmath,amssymb}
\usepackage{lmodern}
\usepackage{ifxetex,ifluatex}
\ifnum 0\ifxetex 1\fi\ifluatex 1\fi=0 % if pdftex
  \usepackage[T1]{fontenc}
  \usepackage[utf8]{inputenc}
  \usepackage{textcomp} % provide euro and other symbols
\else % if luatex or xetex
  \usepackage{unicode-math}
  \defaultfontfeatures{Scale=MatchLowercase}
  \defaultfontfeatures[\rmfamily]{Ligatures=TeX,Scale=1}
  \setmainfont[BoldFont = SF Pro Rounded Semibold]{SF Pro Rounded}
  \setmathfont[]{STIX Two Math}
\fi
\usefonttheme{serif} % use mainfont rather than sansfont for slide text
% Use upquote if available, for straight quotes in verbatim environments
\IfFileExists{upquote.sty}{\usepackage{upquote}}{}
\IfFileExists{microtype.sty}{% use microtype if available
  \usepackage[]{microtype}
  \UseMicrotypeSet[protrusion]{basicmath} % disable protrusion for tt fonts
}{}
\makeatletter
\@ifundefined{KOMAClassName}{% if non-KOMA class
  \IfFileExists{parskip.sty}{%
    \usepackage{parskip}
  }{% else
    \setlength{\parindent}{0pt}
    \setlength{\parskip}{6pt plus 2pt minus 1pt}}
}{% if KOMA class
  \KOMAoptions{parskip=half}}
\makeatother
\usepackage{xcolor}
\IfFileExists{xurl.sty}{\usepackage{xurl}}{} % add URL line breaks if available
\IfFileExists{bookmark.sty}{\usepackage{bookmark}}{\usepackage{hyperref}}
\hypersetup{
  pdftitle={444 Lecture 14.1 - Group Belief},
  pdfauthor={Brian Weatherson},
  hidelinks,
  pdfcreator={LaTeX via pandoc}}
\urlstyle{same} % disable monospaced font for URLs
\newif\ifbibliography
\usepackage{longtable,booktabs,array}
\usepackage{calc} % for calculating minipage widths
\usepackage{caption}
% Make caption package work with longtable
\makeatletter
\def\fnum@table{\tablename~\thetable}
\makeatother
\setlength{\emergencystretch}{3em} % prevent overfull lines
\providecommand{\tightlist}{%
  \setlength{\itemsep}{0pt}\setlength{\parskip}{0pt}}
\setcounter{secnumdepth}{-\maxdimen} % remove section numbering
\let\Tiny=\tiny

 \setbeamertemplate{navigation symbols}{} 

% \usetheme{Madrid}
 \usetheme[numbering=none, progressbar=foot]{metropolis}
 \usecolortheme{wolverine}
 \usepackage{color}
 \usepackage{MnSymbol}
% \usepackage{movie15}

\usepackage{amssymb}% http://ctan.org/pkg/amssymb
\usepackage{pifont}% http://ctan.org/pkg/pifont
\newcommand{\cmark}{\ding{51}}%
\newcommand{\xmark}{\ding{55}}%

\DeclareSymbolFont{symbolsC}{U}{txsyc}{m}{n}
\DeclareMathSymbol{\boxright}{\mathrel}{symbolsC}{128}
\DeclareMathAlphabet{\mathpzc}{OT1}{pzc}{m}{it}

\setlength{\parskip}{1ex plus 0.5ex minus 0.2ex}

\AtBeginSection[]
{
\begin{frame}
	\Huge{\color{darkblue} \insertsection}
\end{frame}
}

\renewenvironment*{quote}	
	{\list{}{\rightmargin   \leftmargin} \item } 	
	{\endlist }

\definecolor{darkgreen}{rgb}{0,0.7,0}
\definecolor{darkblue}{rgb}{0,0,0.8}

\usepackage[italic]{mathastext}
\usepackage{nicefrac}
\usepackage{istgame}

\setbeamertemplate{caption}{\raggedright\insertcaption}

%\def\toprule{}
%\def\bottomrule{}
%\def\midrule{}
\usepackage{etoolbox}
\AfterEndEnvironment{description}{\vspace{9pt}}
\AfterEndEnvironment{oltableau}{\vspace{9pt}}
\BeforeBeginEnvironment{oltableau}{\vspace{9pt}}
\AfterEndEnvironment{center}{\vspace{9pt}}
\BeforeBeginEnvironment{tabular}{\vspace{9pt}}
\AfterEndEnvironment{longtable}{\vspace{-6pt}}
\usepackage{booktabs}
\usepackage{longtable}
\usepackage{array}
\usepackage{multirow}
\usepackage{wrapfig}
\usepackage{float}
\usepackage{colortbl}
\usepackage{pdflscape}
\usepackage{tabu}
\usepackage{threeparttable} 
\usepackage{threeparttablex} 
\usepackage[normalem]{ulem} 
\usepackage{makecell}
\usepackage{xcolor}
\usepackage{ulem}

\setlength\heavyrulewidth{0ex}
\setlength\lightrulewidth{0.08ex}

\aboverulesep=0ex
\belowrulesep=0ex
\renewcommand{\arraystretch}{1.2}
\AtBeginSection[]
{
    \begin{frame}
        \frametitle{Day Plan}
        \tableofcontents[currentsection]
    \end{frame}
}
\ifluatex
  \usepackage{selnolig}  % disable illegal ligatures
\fi

\title{444 Lecture 14.1 - Group Belief}
\author{Brian Weatherson}
\date{}

\begin{document}
\frame{\titlepage}

\begin{frame}{Day Plan}
\protect\hypertarget{day-plan}{}
\tableofcontents
\end{frame}

\hypertarget{a-practical-problem}{%
\section{A Practical Problem}\label{a-practical-problem}}

\begin{frame}{Merging Experts}
\protect\hypertarget{merging-experts}{}
\begin{itemize}
\tightlist
\item
  Let's say some experts have all opined on a particular question.
\item
  And the opinions are in probabilistic form.
\item
  We are not an expert, but we want to defer to expert opinion.
\item
  Unfortunately, the experts disagree.
\item
  What should we do?
\end{itemize}
\end{frame}

\begin{frame}{A Simple Answer}
\protect\hypertarget{a-simple-answer}{}
Arithmetic Mean
\end{frame}

\begin{frame}{Formal Version}
\protect\hypertarget{formal-version}{}
\[
\Pr(p) = \frac {\sum {\Pr_i(p)}}{n}
\]

That is, you add up all the expert probabilities, and divide by the
number of experts.
\end{frame}

\begin{frame}{Three Virtues}
\protect\hypertarget{three-virtues}{}
\begin{itemize}
\tightlist
\item
  Simple.
\item
  Always gives you a probability.
\item
  Always gives you a value between the various expert opinions.
\end{itemize}
\end{frame}

\begin{frame}{Three Problems}
\protect\hypertarget{three-problems}{}
\begin{itemize}
\tightlist
\item
  May not always want to be between the experts.
\item
  Doesn't preserve independence.
\item
  Doesn't handle updating well.
\end{itemize}
\end{frame}

\hypertarget{betweenness}{%
\section{Betweenness}\label{betweenness}}

\begin{frame}{Experts With Different Evidence}
\protect\hypertarget{experts-with-different-evidence}{}
Imagine the following case.

\begin{itemize}
\tightlist
\item
  Two trials of a new medicine are being run, one in Michigan and one in
  California.
\item
  The theory behind the medicine is very speculative, and it's only
  50/50 whether it will work as intended.
\item
  The trials aren't complete, and the evidence isn't released, but you
  know each of the lead researchers, and you call them up separately.
\item
  They each say, ``This is going better than expected; I'd say it's 90\%
  likely that the drug works.''
\item
  What should your probability be that the drug works?
\end{itemize}
\end{frame}

\begin{frame}{My View}
\protect\hypertarget{my-view}{}
I think it should not be between the expert views, it should be higher.

\begin{itemize}
\tightlist
\item
  But this is a special case.
\item
  You have evidence the experts don't have.
\end{itemize}
\end{frame}

\begin{frame}{How To Correct}
\protect\hypertarget{how-to-correct}{}
When we are making a group decision, like in GroupThink, we should first
find out what everyone thinks and why, then adjust our individual
probabilities, then merge those.

\begin{itemize}
\tightlist
\item
  Possibly by arithmetic averaging.
\end{itemize}
\end{frame}

\hypertarget{independence}{%
\section{Independence}\label{independence}}

\begin{frame}{A Puzzle}
\protect\hypertarget{a-puzzle}{}
\begin{itemize}
\tightlist
\item
  You are interested in two propositions, \(p\) and \(q\), as well as
  their interactions.
\item
  There are two experts: \(A\) and \(B\).
\item
  Both experts think that \(A\) and \(B\) are probabilistically
  independent.
\item
  But they disagree on the likelihoods.
\item
  \(A\) thinks that each has probability 0.9.
\item
  \(B\) thinks that each has probability 0.1.
\end{itemize}
\end{frame}

\begin{frame}{Summary}
\protect\hypertarget{summary}{}
\begin{longtable}[]{@{}ccc@{}}
\toprule
Proposition & Expert A & Expert B \\
\midrule
\endhead
\(p \wedge q\) & 0.81 & 0.01 \\
\(p \wedge \neg q\) & 0.09 & 0.09 \\
\(\neg p \wedge q\) & 0.09 & 0.09 \\
\(\neg p \wedge \neg q\) & 0.01 & 0.81 \\
\bottomrule
\end{longtable}
\end{frame}

\begin{frame}{Arithmetic Mean}
\protect\hypertarget{arithmetic-mean}{}
\begin{longtable}[]{@{}cccc@{}}
\toprule
Proposition & Expert A & Expert B & Mean \\
\midrule
\endhead
\(p \wedge q\) & 0.81 & 0.01 & 0.41 \\
\(p \wedge \neg q\) & 0.09 & 0.09 & 0.09 \\
\(\neg p \wedge q\) & 0.09 & 0.09 & 0.09 \\
\(\neg p \wedge \neg q\) & 0.01 & 0.81 & 0.41 \\
\bottomrule
\end{longtable}

You get \(\Pr(p) = \Pr(q) = 0.5\), as seems right. But
\(\Pr(p | q) = 0.82\), which seems wrong.
\end{frame}

\hypertarget{learning}{%
\section{Learning}\label{learning}}

\begin{frame}{Two Orders of Updating}
\protect\hypertarget{two-orders-of-updating}{}
This case is a bit more complicated, but it's the most important one.

\begin{itemize}
\tightlist
\item
  There are two people, \(X\) and \(Y\), who want to learn from the
  experts.
\item
  There is only one proposition that they care about: \(p\).
\item
  There are two experts \(A\) and \(B\).
\item
  In the morning, \(X\) goes to talk to them.
\item
  They say that they'll know more in the afternoon, after it is revealed
  at midday whether \(q\) is true.
\item
  \(X\) says that they can update themselves, just give them the four
  way table.
\end{itemize}
\end{frame}

\begin{frame}{Expert Views}
\protect\hypertarget{expert-views}{}
\begin{longtable}[]{@{}ccc@{}}
\toprule
Proposition & Expert A & Expert B \\
\midrule
\endhead
\(p \wedge q\) & 0.4 & 0.3 \\
\(p \wedge \neg q\) & 0.3 & 0.2 \\
\(\neg p \wedge q\) & 0.2 & 0.1 \\
\(\neg p \wedge \neg q\) & 0.1 & 0.4 \\
\bottomrule
\end{longtable}

\(X\) takes the average of those two probabilities, and comes away with
\(\Pr(p | q) = 0.7\), and \(\Pr(p | \neg q) = 0.5\).
\end{frame}

\begin{frame}{Later That Day}
\protect\hypertarget{later-that-day}{}
\begin{itemize}
\tightlist
\item
  It becomes public knowledge whether \(q\) happens. Let's say it
  doesn't.
\item
  Now \(A\) thinks \(\Pr(p) = \frac{3}{4}\), and \(B\) thinks
  \(\Pr(p) = \frac{1}{3}\).
\item
  Now \(Y\) asks for their expert views, learns them, and averages them.
\item
  So \(Y\) ends up with \(\Pr(p) = \frac{13}{24}\).
\item
  But \(X\) ends up with \(\Pr(p) = 0.5\).
\item
  Yet they did the same thing - defer to these two experts, and make
  one's beliefs sensitive to \(q\).
\end{itemize}
\end{frame}

\begin{frame}{Order Invariance}
\protect\hypertarget{order-invariance}{}
Linear averaging has the following flaw.

\begin{itemize}
\tightlist
\item
  Learning from experts then updating gives a different answer to having
  the experts update then learning from them.
\item
  And that's true even if it is public knowledge what everyone is
  updating on, and there is complete agreement about how to update.
\item
  That seems bad, a rule for learning from experts shouldn't be order
  dependent in this way.
\end{itemize}
\end{frame}

\hypertarget{geometric-pooling}{%
\section{Geometric Pooling}\label{geometric-pooling}}

\begin{frame}{Geometric Mean}
\protect\hypertarget{geometric-mean}{}
In general the geometric mean of \(n\) numbers is:

\[
\sqrt[n]{x_1 x_2 \dots x_n}
\]

Multiply the numbers together, and take the n'th root.
\end{frame}

\begin{frame}{Geometric Mean of Probabilities}
\protect\hypertarget{geometric-mean-of-probabilities}{}
In general, if you take the geometric means of the expert's opinions
about each elemtn of a probability space, the result will \textbf{not}
be a probability.
\end{frame}

\begin{frame}{Old Example}
\protect\hypertarget{old-example}{}
\begin{longtable}[]{@{}ccc@{}}
\toprule
Proposition & Expert A & Expert B \\
\midrule
\endhead
\(p \wedge q\) & 0.81 & 0.01 \\
\(p \wedge \neg q\) & 0.09 & 0.09 \\
\(\neg p \wedge q\) & 0.09 & 0.09 \\
\(\neg p \wedge \neg q\) & 0.01 & 0.81 \\
\bottomrule
\end{longtable}

If for each of the four possibilities, your probability is the geometric
mean of the two experts, you will have probability 0.09 for each. But
those numbers add to 0.36, not to 1.
\end{frame}

\begin{frame}{Re-Normalize}
\protect\hypertarget{re-normalize}{}
So there is one extra step.

\begin{itemize}
\tightlist
\item
  You have to re-normalize.
\item
  You multiply each of the values you get at step 1 by a constant so
  that they sum to 1.
\end{itemize}
\end{frame}

\begin{frame}{Advantages}
\protect\hypertarget{advantages}{}
\begin{itemize}[<+->]
\tightlist
\item
  Preserves Independence
\item
  Order Invariant
\item
  Apparently the only function that will do this.
\end{itemize}
\end{frame}

\begin{frame}{Disadvantages}
\protect\hypertarget{disadvantages}{}
\begin{itemize}[<+->]
\tightlist
\item
  Not simple!
\item
  Still has a problem with the two drug trials case.
\end{itemize}
\end{frame}

\begin{frame}{Open Questions}
\protect\hypertarget{open-questions}{}
\begin{itemize}[<+->]
\tightlist
\item
  How to extend this to density functions.
\item
  Are there other problems? I think you get a problem with the \(XY\)
  case if one of them just asks the experts about \(p\) and the other
  asks about both \(p\) and \(q\). You end up with different answers.
\item
  What to say about the two drug trials case?
\end{itemize}
\end{frame}

\begin{frame}{Philosophical Question}
\protect\hypertarget{philosophical-question}{}
Are these two cases the same?

\begin{enumerate}
\tightlist
\item
  There are \(n\) experts, you ask each for their opinion, and then
  merge the answers.
\item
  There are \(n-1\) other experts plus you, you ask the rest of them for
  their opinion and then do the rational thing with this evidence.
\end{enumerate}

I think this is one of the biggest questions in recent epistemology.
Should members of a group treat themselves as just one more member of
the group for judgment aggregation purposes?
\end{frame}

\end{document}
