\documentclass[11pt,]{article}
\usepackage{lmodern}
\usepackage{amssymb,amsmath}
\usepackage{ifxetex,ifluatex}
\usepackage{fixltx2e} % provides \textsubscript
\ifnum 0\ifxetex 1\fi\ifluatex 1\fi=0 % if pdftex
  \usepackage[T1]{fontenc}
  \usepackage[utf8]{inputenc}
\else % if luatex or xelatex
  \ifxetex
    \usepackage{mathspec}
  \else
    \usepackage{fontspec}
  \fi
  \defaultfontfeatures{Ligatures=TeX,Scale=MatchLowercase}
    \setmainfont[]{SF Pro Text Light}
\fi
% use upquote if available, for straight quotes in verbatim environments
\IfFileExists{upquote.sty}{\usepackage{upquote}}{}
% use microtype if available
\IfFileExists{microtype.sty}{%
\usepackage{microtype}
\UseMicrotypeSet[protrusion]{basicmath} % disable protrusion for tt fonts
}{}
\usepackage[margin=1in]{geometry}
\usepackage{hyperref}
\hypersetup{unicode=true,
            pdftitle={Essay Questions},
            pdfauthor={Philosophy 444},
            pdfborder={0 0 0},
            breaklinks=true}
\urlstyle{same}  % don't use monospace font for urls
\usepackage{graphicx,grffile}
\makeatletter
\def\maxwidth{\ifdim\Gin@nat@width>\linewidth\linewidth\else\Gin@nat@width\fi}
\def\maxheight{\ifdim\Gin@nat@height>\textheight\textheight\else\Gin@nat@height\fi}
\makeatother
% Scale images if necessary, so that they will not overflow the page
% margins by default, and it is still possible to overwrite the defaults
% using explicit options in \includegraphics[width, height, ...]{}
\setkeys{Gin}{width=\maxwidth,height=\maxheight,keepaspectratio}
\IfFileExists{parskip.sty}{%
\usepackage{parskip}
}{% else
\setlength{\parindent}{0pt}
\setlength{\parskip}{6pt plus 2pt minus 1pt}
}
\setlength{\emergencystretch}{3em}  % prevent overfull lines
\providecommand{\tightlist}{%
  \setlength{\itemsep}{0pt}\setlength{\parskip}{0pt}}
\setcounter{secnumdepth}{0}
% Redefines (sub)paragraphs to behave more like sections
\ifx\paragraph\undefined\else
\let\oldparagraph\paragraph
\renewcommand{\paragraph}[1]{\oldparagraph{#1}\mbox{}}
\fi
\ifx\subparagraph\undefined\else
\let\oldsubparagraph\subparagraph
\renewcommand{\subparagraph}[1]{\oldsubparagraph{#1}\mbox{}}
\fi

%%% Use protect on footnotes to avoid problems with footnotes in titles
\let\rmarkdownfootnote\footnote%
\def\footnote{\protect\rmarkdownfootnote}

%%% Change title format to be more compact
\usepackage{titling}

% Create subtitle command for use in maketitle
\providecommand{\subtitle}[1]{
  \posttitle{
    \begin{center}\large#1\end{center}
    }
}

\setlength{\droptitle}{-2em}

  \title{Essay Questions}
    \pretitle{\vspace{\droptitle}\centering\huge}
  \posttitle{\par}
    \author{Philosophy 444}
    \preauthor{\centering\large\emph}
  \postauthor{\par}
      \predate{\centering\large\emph}
  \postdate{\par}
    \date{Due Tuesday December 17}

\usepackage{mathastext}
\usepackage{nicefrac}

\begin{document}
\maketitle

Answer \textbf{one} of the following questions in an essay of about 2500
words. The paper should engage with the relevant readings, and all
references to works of other authors should be properly cited. (We don't
particularly care which citation method you use, as long as all direct
and indirect references are properly cited.)

If you want feedback on a draft, you need to get it to Angela Sun by
\textbf{Tuesday, December 3}. You don't have to do this, but it is
strongly encouraged.

\hypertarget{question-one}{%
\section{Question One}\label{question-one}}

What does it mean to say that acquiring education provides a signal to
prospective employers, in the sense of Spence's work on job market
signalling? How plausible is the claim that the primary role of college
education is to provide such signals?

\hypertarget{question-two}{%
\section{Question Two}\label{question-two}}

Describe a situation where the preferences of each individual in a group
conflict with rights that individuals intuitively have. The situation
should be described in some detail. You should not use abstracta (like A
prefers X to Y), but instead fill in the details. And the situation
should be different from the examples described in the literature (such
as Sen's example involving Lady Chatterley's Lover).

In such a situation, what is the best outcome? Should the preferences of
everyone in the group (including the person whose rights are potentially
violated) be respected, or should the rights of the individual be
respected?

What do situations like this tell us about the best way to design
institutions for group decision making?

\hypertarget{question-three}{%
\section{Question Three}\label{question-three}}

Abraham Roth says that on Margaret Gilbert's view of shared action,
`'participants in shared activity are obligated to do their part in
it.'' What is meant by this claim about obligation? Why does Gilbert
defend it? Why does Michael Bratman think it is false, at least as a
claim about what is essential to shared action?

\hypertarget{question-four}{%
\section{Question Four}\label{question-four}}

Are there situations where a group knows that \emph{p}, but few people
in the group know that \emph{p}? Are there situations where a group
knows that \emph{p}, but no people in the group know that \emph{p}?

Defend your answers, with reference to the papers we read on group
knowledge.


\end{document}
