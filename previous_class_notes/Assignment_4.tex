\documentclass[11pt,]{article}
\usepackage{lmodern}
\usepackage{amssymb,amsmath}
\usepackage{ifxetex,ifluatex}
\usepackage{fixltx2e} % provides \textsubscript
\ifnum 0\ifxetex 1\fi\ifluatex 1\fi=0 % if pdftex
  \usepackage[T1]{fontenc}
  \usepackage[utf8]{inputenc}
\else % if luatex or xelatex
  \ifxetex
    \usepackage{mathspec}
  \else
    \usepackage{fontspec}
  \fi
  \defaultfontfeatures{Ligatures=TeX,Scale=MatchLowercase}
    \setmainfont[]{SF Pro Text Light}
\fi
% use upquote if available, for straight quotes in verbatim environments
\IfFileExists{upquote.sty}{\usepackage{upquote}}{}
% use microtype if available
\IfFileExists{microtype.sty}{%
\usepackage{microtype}
\UseMicrotypeSet[protrusion]{basicmath} % disable protrusion for tt fonts
}{}
\usepackage[margin=1in]{geometry}
\usepackage{hyperref}
\hypersetup{unicode=true,
            pdftitle={Assigment 4},
            pdfauthor={Philosophy 444},
            pdfborder={0 0 0},
            breaklinks=true}
\urlstyle{same}  % don't use monospace font for urls
\usepackage{longtable,booktabs}
\usepackage{graphicx,grffile}
\makeatletter
\def\maxwidth{\ifdim\Gin@nat@width>\linewidth\linewidth\else\Gin@nat@width\fi}
\def\maxheight{\ifdim\Gin@nat@height>\textheight\textheight\else\Gin@nat@height\fi}
\makeatother
% Scale images if necessary, so that they will not overflow the page
% margins by default, and it is still possible to overwrite the defaults
% using explicit options in \includegraphics[width, height, ...]{}
\setkeys{Gin}{width=\maxwidth,height=\maxheight,keepaspectratio}
\IfFileExists{parskip.sty}{%
\usepackage{parskip}
}{% else
\setlength{\parindent}{0pt}
\setlength{\parskip}{6pt plus 2pt minus 1pt}
}
\setlength{\emergencystretch}{3em}  % prevent overfull lines
\providecommand{\tightlist}{%
  \setlength{\itemsep}{0pt}\setlength{\parskip}{0pt}}
\setcounter{secnumdepth}{0}
% Redefines (sub)paragraphs to behave more like sections
\ifx\paragraph\undefined\else
\let\oldparagraph\paragraph
\renewcommand{\paragraph}[1]{\oldparagraph{#1}\mbox{}}
\fi
\ifx\subparagraph\undefined\else
\let\oldsubparagraph\subparagraph
\renewcommand{\subparagraph}[1]{\oldsubparagraph{#1}\mbox{}}
\fi

%%% Use protect on footnotes to avoid problems with footnotes in titles
\let\rmarkdownfootnote\footnote%
\def\footnote{\protect\rmarkdownfootnote}

%%% Change title format to be more compact
\usepackage{titling}

% Create subtitle command for use in maketitle
\providecommand{\subtitle}[1]{
  \posttitle{
    \begin{center}\large#1\end{center}
    }
}

\setlength{\droptitle}{-2em}

  \title{Assigment 4}
    \pretitle{\vspace{\droptitle}\centering\huge}
  \posttitle{\par}
    \author{Philosophy 444}
    \preauthor{\centering\large\emph}
  \postauthor{\par}
      \predate{\centering\large\emph}
  \postdate{\par}
    \date{Due 4 October, 2019}

\usepackage{gensymb}
\usepackage{nicefrac}
\usepackage{caption}
\usepackage{istgame}
\usepackage{mathastext}

\begin{document}
\maketitle

\hypertarget{questions-one-six}{%
\section{Questions One-Six}\label{questions-one-six}}

A thief wants to steal a particular diamond owned by the city council.
The security firm guarding the diamond knows that the thief might strike
tonight. They have to decide whether to put in Basic security (B) or
Enhanced security (E). The thief has to decide whether to Steal (S) or
Not steal (N) the diamond. The thief can evade basic security, but will
be caught by enhanced security. Here is the payoff table for the two
parties (with thief payouts first, since she's row).

\begin{longtable}[]{@{}rcc@{}}
\toprule
& B & E\tabularnewline
\midrule
\endhead
S & 1, 0 & -5, 2\tabularnewline
N & 0, 1 & 0, 0\tabularnewline
\bottomrule
\end{longtable}

Intuitive, the thief wants the diamond, doesn't want to get caught, and
cares way more about getting caught that the diamond. The security firm
doesn't want the bad reputation of the diamond been stolen on their
watch, or the cost of running Enhanced security, but they would like the
reward for catching the thief.

The following questions are all about the mixed strategy Nash
equilibrium of this game. (Answers should be decimal, not fractions, and
accurate to two decimal places.)

\begin{enumerate}
\def\labelenumi{\arabic{enumi}.}
\tightlist
\item
  What is the probability of B?
\item
  What is the probability of E?
\item
  What is the probability of S?
\item
  What is the probability of N?
\item
  What is firm's expected return?
\item
  What is thief's expected return?
\end{enumerate}

\hypertarget{questions-seven-nine}{%
\section{Questions Seven-Nine}\label{questions-seven-nine}}

A tough-on-crime faction on the city council wants to prevent theft by
increasing the punishment for being caught stealing. If they had their
way, the payout in the upper-right cell (i.e., SE), would be -10, 2. If
this change was made, how would the new equilibrium compare to the old
equilibrium?

\begin{enumerate}
\def\labelenumi{\arabic{enumi}.}
\setcounter{enumi}{6}
\tightlist
\item
  Would firm's expected payout go up, go down, or be unchanged?
\item
  Would thief's expected payout go up, go down, or be unchanged?
\item
  Would the probability that the diamond gets stolen go up, go down, or
  be unchanged?
\end{enumerate}

\hypertarget{questions-ten---eleven}{%
\section{Questions Ten - Eleven}\label{questions-ten---eleven}}

The tough-on-crime faction is defeated, but there is a new dispute over
exactly what we should do to the payout in the top right cell. There are
three options

\begin{itemize}
\tightlist
\item
  Option one: leave it unchanged.
\item
  Option two: change it to -10, 2, as the tough-on-crime faction wanted.
\item
  Option three: change it to -5, 3, i.e., increase the reward for
  catching the thief.
\end{itemize}

Happily for the security firm, a nice bit of regulatory capture means
that they control the deciding vote. So this is now a dynamic game. Firm
will choose option one, two or three, and then choose Basic or Enhanced.
Thief will find out whether one, two or three was chosen, then choose
Steal or Not.

The following questions concern the subgame perfect equilibrium of this
game.

\begin{enumerate}
\def\labelenumi{\arabic{enumi}.}
\setcounter{enumi}{9}
\tightlist
\item
  What is firm's expected payout?
\item
  What is the probability that the diamond gets stolen?
\end{enumerate}


\end{document}
