\documentclass[11pt,]{article}
\usepackage{lmodern}
\usepackage{amssymb,amsmath}
\usepackage{ifxetex,ifluatex}
\usepackage{fixltx2e} % provides \textsubscript
\ifnum 0\ifxetex 1\fi\ifluatex 1\fi=0 % if pdftex
  \usepackage[T1]{fontenc}
  \usepackage[utf8]{inputenc}
\else % if luatex or xelatex
  \ifxetex
    \usepackage{mathspec}
  \else
    \usepackage{fontspec}
  \fi
  \defaultfontfeatures{Ligatures=TeX,Scale=MatchLowercase}
    \setmainfont[]{SF Pro Text Light}
\fi
% use upquote if available, for straight quotes in verbatim environments
\IfFileExists{upquote.sty}{\usepackage{upquote}}{}
% use microtype if available
\IfFileExists{microtype.sty}{%
\usepackage{microtype}
\UseMicrotypeSet[protrusion]{basicmath} % disable protrusion for tt fonts
}{}
\usepackage[margin=1in]{geometry}
\usepackage{hyperref}
\hypersetup{unicode=true,
            pdftitle={Non-Cooperative Signalling},
            pdfauthor={Philosophy 444},
            pdfborder={0 0 0},
            breaklinks=true}
\urlstyle{same}  % don't use monospace font for urls
\usepackage{longtable,booktabs}
\usepackage{graphicx,grffile}
\makeatletter
\def\maxwidth{\ifdim\Gin@nat@width>\linewidth\linewidth\else\Gin@nat@width\fi}
\def\maxheight{\ifdim\Gin@nat@height>\textheight\textheight\else\Gin@nat@height\fi}
\makeatother
% Scale images if necessary, so that they will not overflow the page
% margins by default, and it is still possible to overwrite the defaults
% using explicit options in \includegraphics[width, height, ...]{}
\setkeys{Gin}{width=\maxwidth,height=\maxheight,keepaspectratio}
\IfFileExists{parskip.sty}{%
\usepackage{parskip}
}{% else
\setlength{\parindent}{0pt}
\setlength{\parskip}{6pt plus 2pt minus 1pt}
}
\setlength{\emergencystretch}{3em}  % prevent overfull lines
\providecommand{\tightlist}{%
  \setlength{\itemsep}{0pt}\setlength{\parskip}{0pt}}
\setcounter{secnumdepth}{0}
% Redefines (sub)paragraphs to behave more like sections
\ifx\paragraph\undefined\else
\let\oldparagraph\paragraph
\renewcommand{\paragraph}[1]{\oldparagraph{#1}\mbox{}}
\fi
\ifx\subparagraph\undefined\else
\let\oldsubparagraph\subparagraph
\renewcommand{\subparagraph}[1]{\oldsubparagraph{#1}\mbox{}}
\fi

%%% Use protect on footnotes to avoid problems with footnotes in titles
\let\rmarkdownfootnote\footnote%
\def\footnote{\protect\rmarkdownfootnote}

%%% Change title format to be more compact
\usepackage{titling}

% Create subtitle command for use in maketitle
\providecommand{\subtitle}[1]{
  \posttitle{
    \begin{center}\large#1\end{center}
    }
}

\setlength{\droptitle}{-2em}

  \title{Non-Cooperative Signalling}
    \pretitle{\vspace{\droptitle}\centering\huge}
  \posttitle{\par}
    \author{Philosophy 444}
    \preauthor{\centering\large\emph}
  \postauthor{\par}
      \predate{\centering\large\emph}
  \postdate{\par}
    \date{23 October, 2019}

\usepackage{mathastext}
\usepackage{nicefrac}

\begin{document}
\maketitle

\hypertarget{basic-game}{%
\section{Basic Game}\label{basic-game}}

\begin{itemize}
\tightlist
\item
  Two states of the world - sender knows them, receiver doesn't. Call
  them `High' and `Low'. (Or H/L.) Assume for now that High is
  marginally more probable than Low.
\item
  Two possible signals/messages that sender can send - call them
  `Difficult' and `Easy' for reasons that we will get to. (Or D/E).
\item
  Two possible actions receiver can take - call them Risky and Safe (Or
  R/S).
\end{itemize}

In words, here are the payoffs:

\begin{itemize}
\tightlist
\item
  Sender pays 0 to perform Easy, but a cost \(c > 0\) to perform
  Difficult. This is subtracted from whatever their ultimate payout is.
  (We will complicate this clause in a bit.)
\item
  Receiver gets a payout of 0 for doing Safe.
\item
  If Receiver plays Risky, they get 1 if High, and -1 if Low.
\item
  Sender gets 1 (minus whatever cost they incurred at round 1) if Risky,
  and 0 (minus whatever cost they incurred at round 1) if Low.
\end{itemize}

In table form, here are the payouts. First, here are the payouts for
High.

\begin{longtable}[]{@{}rcc@{}}
\toprule
& Risky & Safe\tabularnewline
\midrule
\endhead
Difficult & \(1-c, 1\) & \(-c, 0\)\tabularnewline
Easy & \(1, 1\) & \(0, 0\)\tabularnewline
\bottomrule
\end{longtable}

Now, here are the payouts for Low.

\begin{longtable}[]{@{}rcc@{}}
\toprule
& Risky & Safe\tabularnewline
\midrule
\endhead
Difficult & \(1-c, -1\) & \(-c, 0\)\tabularnewline
Easy & \(1, -1\) & \(0, 0\)\tabularnewline
\bottomrule
\end{longtable}

Let's try to work through what the equilibria are for different values
of \(c\).

\begin{itemize}
\tightlist
\item
  First, assume \(c < 1\).
\item
  Assume that Receiver will do different things if Difficult or Easy.
\item
  Then Sender will do whatever makes Receiver do Risky, whether it is D
  or E. So no separating equilibrium.
\item
  Now assume Receiver will do the same thing if Difficult or Easy.
\item
  Then of course Sender will do Easy and get that payout, whatever it
  is.
\item
  Again, no separating equilibrium.
\end{itemize}

\newpage

What does produce a separating equilibrium is if the cost is different
in High and Low. Change the `penalty' for playing Difficult so that in
High, the penalty is \(c_1 < 1\), and in Low, the penalty is
\(c_2 > 1\). Now we get the following equilibrium.

\begin{itemize}
\tightlist
\item
  Sender does Difficult if High, Easy if Low.
\item
  Receiver does Risky if Difficult, Safe if Easy.
\end{itemize}

(Aside: You also get one other really weird equilibrium where

\begin{itemize}
\tightlist
\item
  Sender always does Easy.
\item
  Receiver does Risky if Easy, Safe if Difficult.
\end{itemize}

This is mathematically interesting - and hence interesting to me -
because it is not just a Nash equilibrium and subgame perfect, but also
satisfies all the extra criteria developed in chapters 11 and 12 to rule
out intuitively absurd equilibria like this one. But it is hard to see
how it has any real-world relevance - it doesn't look like it could
naturally evolve, for example - and I'll ignore it from here on.)

So we get a separating equilibrium. And maybe we get a model of some
fascinating real-world things. In most of these cases, Sender has
something like a continuum of choices from Easy to Difficult, and
Receiver has a continuum from Risky to Safe. But it arguably helps to
look at the binary case first, and maybe we can generalise that to the
real-world example.

Two caveats before we start the examples.

In practice, biologists seem happy to use this procedure - think about
the binary model and then generalise to the continuous case - while
economists prefer to start with the continuous case. My intuitions are
normally with the economists, but here I'm acting like a biologist.

And these are possible models of what we see. In every case, I'm going
to eventually raise worries for the model. But I want to have them on
the table.

\hypertarget{example-one-tail-feathers}{%
\section{Example One: Tail-Feathers}\label{example-one-tail-feathers}}

Male peacocks have very colorful tails. On the face of it, this doesn't
look like it serves any purpose in either collecting food or avoiding
becoming food. (Quite the opposite in fact.) But maybe we should think
of it as a move in a signalling game.

\begin{itemize}
\tightlist
\item
  Sender is the male, choosing whether to have a normal tail (Easy) or a
  colourful tail (Difficult). `Choosing' here is misleading - it's less
  misleading to say their genes choose.
\item
  Sender is either Strong (that's High) or Weak (that's Low).
\item
  It's resource-intensive to produce (and preserve) a colorful tail, but
  it's more costly for Weak than for Strong peacocks.
\item
  Receiver is a female, choosing a mate. They prefer Strong to Weak -
  since they want better genes for their children. (Again, it's hard to
  say this is what the individual female wants - better to say there are
  evolutionary advantages to acting as if that's what she wants.)
\item
  So perhaps an equilibrium is Strong have colorful tails, Weak don't,
  and females who have a choice prefer males with colorful tails.
\end{itemize}

\hypertarget{example-two-stotting}{%
\section{Example Two: Stotting}\label{example-two-stotting}}

Stotting is where a quadruped leaps into the air, with legs relatively
stiff. Stotting is common among young animals in various species. But
the really odd thing is that among some gazelles, it only happens when a
predator is nearby. And why they do this is a bit of a mystery. It
doesn't seem that efficient as a means of propulsion. And revealing
one's location this dramatically doesn't seem like a good tactic in
predator avoidance. But maybe it's a move in a signalling game. In this
game, the payouts are slightly changed. The gazelle is the sender, and
the predator is the receiver, and the predator's payoffs are going to be
different from the standard game.

\begin{itemize}
\tightlist
\item
  High in this case is that the gazelle is strong (High means things are
  good from the predator's perspective.) Low is that the gazelle is
  weak.
\item
  Stotting is Difficult; Not-stotting is Easy.
\item
  Chasing this particular gazelle is Risky; leaving it is Safe.
\item
  But here we change the payoffs. Here receiver/predator gets -1 for
  doing Risky in High, and 1 for doing Risky in Safe. Otherwise the game
  is the same.
\end{itemize}

Again, there is a separating equilibria.

\begin{itemize}
\tightlist
\item
  Gazelle stotts if and only if they are strong.
\item
  Predator chases if and only if they don't see stotting.
\end{itemize}

And so there is an advantage to stotting - you don't get chased - even
though holding fixed the state of the world and the behavior of the
predator, stotting is a cost with no benefit. It doesn't help you get
away - it helps the predator not choose to attack.

\hypertarget{example-three-university}{%
\section{Example Three: University}\label{example-three-university}}

According to recent research from the San Francisco Federal Reserve,
here are the average hourly wages for Americans by educational level as
of 2015. (I don't think the numbers have changed much since.) I've also
added a column for what percentage of the workforce each of these groups
are. Source:
\url{https://www.frbsf.org/economic-research/files/wp2016-17.pdf}

\begin{longtable}[]{@{}ccc@{}}
\toprule
Education & Wage & Ratio\tabularnewline
\midrule
\endhead
No degree & \$13.56 & 7.7\%\tabularnewline
High school degree & \$17.98 & 25.6\%\tabularnewline
Some college & \$21.59 & 27.8\%\tabularnewline
Undergrad degree & \$30.93 & 24.7\%\tabularnewline
Graduate degree & \$39.48 & 14.3\%\tabularnewline
\bottomrule
\end{longtable}

What could explain the fact that college graduates earn almost 75\% more
per hour than high school graduates? There are two obvious
possibilities.

\begin{enumerate}
\def\labelenumi{\arabic{enumi}.}
\tightlist
\item
  Universities impart lots of valuable skills, and employers are
  rationally responding to this value we add by paying more for more
  valuable employees.
\item
  It's a selection effect - the people who come to college were more
  valuable before they came here, and employees are rationally
  responding to that underlying fact.
\end{enumerate}

These aren't exclusive, but let's pretend for now we're going to assume
one of them is decisive, and we're trying to figure out which it is.
There are two things missing in explanation 2.

First, why do all these smart people choose to go to college? On the one
hand, it is mostly fun. On the other hand, it's expensive, and there are
a lot of other fun things you could do with the money. If it's all about
the climbing walls, you could join the fanciest gym in the country for a
fraction of the cost. If it's about meeting new people, you could go
backpacking around Europe. If it's about intellectual stimulation, you
could take a gap year or four and spend your days reading and listening
to educational podcasts.

Second, why do employers look for college degrees as the signal of who
they will pay high wages to? Why don't they simply ask to see your offer
letters? If it's just a selection effect, then you can see who has been
selected in as soon as the offer letters go out - and that should be
enough to make employers happy. But it's not - they want degrees not
just offer letters. (This seems really obvious, but I think it's kind of
a striking fact about the modern world, and one that doesn't get enough
attention in a lot of debates.)

So we need a model that explains why we don't see, for example, valuable
employees taking their offer letters and backpacking around Europe with
Kindles and podcasts for their spare time. Spence's signalling model
provides such an explanation.

\begin{itemize}
\tightlist
\item
  Sender is the college graduate. They are either High - i.e., valuable
  to employers, or Low - not so valuable.
\item
  Receiver is the high wage employer. They can do the Risky thing - hire
  this person - or the Safe thing - not hire them.
\item
  Going to college is Difficult. But - and this is the crucial thing -
  on this model it is more Difficult for Low than for High. All those
  calculus classes are really unpleasant. But they are more unpleasant
  for Low - so much so that \(c_2 > 1\).
\item
  So the separating equilibrium is High goes to college and Low hits the
  beach/workforce. Then high-wage employers hire college graduates only.
\item
  And all this happens even though college is a pure cost to everyone
  who goes there. Employer doesn't get any reward for hiring graduates -
  they get same reward for hiring High grads as High non-grads. And
  holding all else fixed, every young person is better off skipping
  college than going.
\end{itemize}

Four questions for you to discuss.

\begin{enumerate}
\def\labelenumi{\arabic{enumi}.}
\tightlist
\item
  What aspects of this model seem to resemble the world you (as in
  literally you personally) find yourself in?
\item
  What aspects of it seem to differ from the world in significant ways?
\item
  What empirical data would make you think this model was right in some
  important way?
\item
  What empirical data would make you think this model was wrong in some
  important way?
\end{enumerate}


\end{document}
